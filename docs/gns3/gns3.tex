\documentclass[12pt]{article}
\usepackage{geometry}
\geometry{a4paper, total={170mm,257mm},left=20mm, top=20mm,}
\usepackage[colorlinks=true,linkcolor=blue,urlcolor=black]{hyperref}
\usepackage{bookmark}
\usepackage[autostyle, english = american]{csquotes}
\begin{document}
\begin{titlepage}
\title {Running Labtainers from within GNS3}
\maketitle

\vspace{2.0in}
This document was created by United States Government employees at 
The Center for Cybersecurity and Cyber Operations (C3O) at the Naval Postgraduate School NPS. 
Please note that within the United States, copyright protection is not available for any works created  
by United States Government employees, pursuant to Title 17 United States Code Section 105.   
This document is in the public domain and is not subject to copyright. 
\end{titlepage}
\tableofcontents
\newpage
\section {Introduction}
This document describes the integration of GNS3 with Labtainers.


\section{Cloning and installing GNS3 and Labtainers}
Get the gns3 branch from Labtainers:
\begin{verbatim}
git clone https://github.com/mfthomps/labtainers.git
git checkout gns3 
\end{verbatim}
\noindent Define the {\tt LABTAINERS\_DIR} environment variable to point to the Labtainers directory.
Create the capinout executable:
\begin{verbatim}
cd $LABTAINER_DIR/tool-src/capinout
./mkit.sh
\end{verbatim}

Get the GNS3 server from the forked github repo:
\begin{verbatim}
git clone --single-branch --branch labtainers https://github.com/mfthomps/gns3-server.git
cd gns3-server
sudo python3 setup.py install
\end{verbatim}

Get the GNS3 GUI from the gns3 github repo:
\begin{verbatim}
git clone --single-branch --branch 2.1 https://github.com/mfthomps/gns3-gui.git
git checkout tags/v2.1.21
cd gns3-gui
sudo python3 setup.py install
\end{verbatim}

Run the server from a terminal with LABTAINER\_DIR defined.

\begin{verbatim}
gns3server --local --log /tmp/gns3log
\end{verbatim}

Run the gui:
\begin{verbatim}
    gns3
\end{verbatim}

\section{Updating local git repos}
Note below that your Labtainers repo uses the "gns3" branch -- and your GNS3 repo uses the "labtainers" branch.
From the local GNS3  repo:
\begin{verbatim}
git pull origin labtainers
\end{verbatim}

From the local Labtainers repo:
\begin{verbatim}
git pull origin gns3
\end{verbatim}

\section{Updating remote repo}
The GNS3 repo:
\begin{verbatim}
git push --set-upstream origin labtainers
\end{verbatim}

\noindent The Labtainer repo:
\begin{verbatim}
git push --set-upstream origin gns3
\end{verbatim}


\section{Porting a Labtainers lab}
This example illustrates porting the telnetlab from Labtainers to run in the GNS3 environment.
It is assumed the gns3-server has been started.
This workflow can be further automated to automatically generate network connections.  Or to fully define the project
file based on predefined templates.

Go to the {\tt \$LABTAINER\_DIR/scripts/gns3} directory:
\begin{verbatim}
cd $LABTAINER_DIR/scripts/gns3
\end{verbatim}

Create modified Docker images for the lab:
\begin{verbatim}
./noNet.py telnetlab
\end{verbatim}

View the Labtainer network topology.
\begin{verbatim}
./showNet.py telnetlab
\end{verbatim}
\noindent Use the GNS3 gui to define GNS3 Docker container templates for each of the 
images you created, and to then use those templates to create a GNS3 instance of 
the lab.
\begin{itemize}
	\item Start the gui, e.g., run {\tt gns3}
	\item Open a new project, assigning the same name as the Labtainers lab, e.g., ``telnetlab''
	\item Use {\tt Browse all devices / Add applicance template} to add the new container images created
		via the {\tt noNet.py} command.  Accept all defaults, except set the number of ``Adapaters'' to
		the quantity displayed using {\tt showNet.py} for each component.
	\item Drag each container image from the list on the left of the GUI onto the workspace.  (Do not try to fix
		the component names, that will be done later).
	\item Use the gui's {\tt Add a Link} function to connect each component per the output of {\tt showNet.py}.
	\item Run the {\tt ./genNet.py telnetlab telnetlab} command to generate GNS3 network connection files, and to
		fix the component names.  You will not see the revised names until you restart the lab.
	\item Stop the GUI and restart it.  Then test the lab.
\end{itemize}

\section{History}
This section describes steps taken to create the Labtainers version of GNS3.
The steps outlined here are not intended to be repeated by developers.

The gns3 server was forked into https://github.com/mfthomps/gns3-server.git

The branch "2.1" was then cloned the tag set to tag v2.1.21:
\begin{verbatim}
git clone --single-branch --branch 2.1 https://github.com/mfthomps/gns3-server.git
git checkout tags/v2.1.21
\end{verbatim}

\noindent A new ``labtainers'' branch was defined for the gns3 fork:
\begin{verbatim}
    git tag v1.2.1-labtainers
    git checkout -b labtainers
    git push --set-upstream origin labtainers
\end{verbatim}

The node.py script was altered to increase the timeout to account for parameterization of
containers.  This should be revisited to be perhaps less crude.

Install qtcreator
Open the {\tt gns3-gui/gns3/gui/main\_window.ui} file

To add icons:
\begin{itemize}
   Add to resources/images or wherever.
   Edit resources.qrc
   scripts/build\_pyqt.py --ressources
   Use qtcreator to identify the icon in "resources"
\end{itemize}
\end{document}
