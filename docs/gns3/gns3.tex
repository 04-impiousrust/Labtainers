\documentclass[12pt]{article}
\usepackage{geometry}
\geometry{a4paper, total={170mm,257mm},left=20mm, top=20mm,}
\usepackage[colorlinks=true,linkcolor=blue,urlcolor=black]{hyperref}
\usepackage{bookmark}
\usepackage[autostyle, english = american]{csquotes}
\begin{document}
\begin{titlepage}
\title {Running Labtainers from within GNS3}
\maketitle

\vspace{2.0in}
This document was created by United States Government employees at 
The Center for Cybersecurity and Cyber Operations (C3O) at the Naval Postgraduate School NPS. 
Please note that within the United States, copyright protection is not available for any works created  
by United States Government employees, pursuant to Title 17 United States Code Section 105.   
This document is in the public domain and is not subject to copyright. 
\end{titlepage}
\tableofcontents
\newpage
\section {Introduction}
This document describes the integration of GNS3 with Labtainers.
The goal of this integration is to allow lab exercises developed for Labtainers to run within the
GNS3 environment.   From the student's perspective, they see the GNS3 GUI and select their lab as a standard GNS3 project.
Labtainer components will be visible in the GNS3 workspace along with any other components defined in the lab, 
e.g., GNS3 routers.  Pressing the "play" button starts that lab, and any Labtainers terminal windows that were
defined for the lab will open as the components come online.

All components within the running lab are known to GNS3 as standard GNS3 nodes.  Those that happen to be based on
Labtainers containers will call-back into Labtainers modules, e.g., to open new terminals, or perform configuration steps
during startup.  All component networking is via GNS3 network management.

The lab development workflow is summarised as follows:
\begin{itemize}
	\item Develop a lab using standard Labtainers tools and techniques, ignoring true routers, switches and such.
	\item Run a script that creates GNS3-Labtainers instances of the lab's Docker images
	\item Use GNS3 to lay out the topology of the newly created images, along with other devices such as routers.
	\item Run a script to automatically configure the GNS3 network configurations to match those defined for Labtainers
	\item Restart GNS3 and test the lab
	\item The new "lab" then consists of the GNS3 project directory and the Labtainers lab directory.
\end{itemize}

\section {Requirements}
It is assumed your development system is a recent Linux distribution.  You will require these packages:
\begin{itemize}
	\item pthon3-setuptools
	\item python3-dev
	\item python3-pyqt5
	\item g++
\end{itemize}
\noindent Install GNS3 from their website in order to get the ubridge software property configured.
\begin{verbatim}
https://docs.gns3.com/1QXVIihk7dsOL7Xr7Bmz4zRzTsJ02wklfImGuHwTlaA4/index.html
\end{verbatim}
Add yourself to the ubridge group and logout and in to get permissions set per GNS3 guidance.

\section{Cloning and installing GNS3 and Labtainers}
\subsection{Labtainers repo (optional)}
This subsection is optional, and is intended to give access to GNS3 updates to Labtainers directly
from github without waiting for formal Labtainers releases.  

Get the gns3 branch from Labtainers:
\begin{verbatim}
git clone https://github.com/mfthomps/labtainers.git
cd labtainers
git checkout gns3 
\end{verbatim}
\noindent Define the {\tt LABTAINERS\_DIR} environment variable to point to the Labtainers directory. Also define \$PATH to include '/bin' and \$LABTAINER\_DIR/scripts/designer/bin
(Log out and in, or create a new bash shell to take effect.)
Create the capinout executable (Ignore the warning messages):
\begin{verbatim}
cd $LABTAINER_DIR/tool-src/capinout
./mkit.sh
\end{verbatim}
\noindent Run the docker install script for your operating system located in \$LABTAINERS\_DIR/setup\_scripts to install docker and other packages to run Labtainers. Ex. For Ubuntu developers run install-docker-ubuntu.sh.

\vspace{5mm}

\noindent
Run any Labtainers lab from the labtainer-student directory to test that it works and to capture a Labtainers email identifier.

\subsection{GNS3 Repos}
Get the GNS3 server from the forked github repo.  First change your directory to where you want the repos to
exist.  Then:
\small
\begin{verbatim}
git clone --single-branch --branch labtainers https://github.com/mfthomps/gns3-server.git
cd gns3-server
sudo python3 setup.py install
\end{verbatim}

Change directory to where you want the repo, and then get the GNS3 GUI from the gns3 github repo:
\small
\begin{verbatim}
git clone --single-branch --branch labtainers  https://github.com/mfthomps/gns3-gui.git
cd gns3-gui
sudo python3 setup.py install
\end{verbatim}
\normalsize


\subsubsection{Updating local git repos}
Note below that your Labtainers repo uses the "gns3" branch -- and your GNS3 repo uses the "labtainers" branch.
From the local GNS3  repo:
\begin{verbatim}
      git pull origin labtainers
\end{verbatim}

From the local Labtainers repo:
\begin{verbatim}
git pull origin gns3
\end{verbatim}

After pulling, always run {\tt sudo python3 setup.py install} to get the latest 
build of GNS3. Do this as well after making local edits to the source code to see changes in the build.

\subsubsection{Updating remote repo}
This step is only for use by Labtainers framework developers.
The GNS3 repo:
\begin{verbatim}
git push --set-upstream origin labtainers
\end{verbatim}

\noindent The Labtainer repo:
\begin{verbatim}
git push --set-upstream origin gns3
\end{verbatim}

\section{Starting GNS3}
Run the server from a terminal with LABTAINER\_DIR defined.

\begin{verbatim}
      gns3server --local --log /tmp/gns3log
\end{verbatim}
\noindent The server does not display to stdout.  Tail the /tmp/gns3log to see status.

\noindent Run the gui from a different terminal or tab:
\begin{verbatim}
    gns3
\end{verbatim}

\section{Porting a Labtainers lab}
This example illustrates porting the telnetlab from Labtainers to run in the GNS3 environment.
It is assumed the gns3-server has been started.
This workflow can be further automated to automatically generate network connections.  Or to fully define the project
file based on predefined templates.

These steps assume the Labtainer Docker images exist on the machine, e.g., you've peformed a rebuild.py.  If not, run the Labtainers lab
to cause them to be pulled. 

\bigskip
If there is a logo that you wish to appear on the GNS3 display, put that file in 
\begin{verbatim}
      <lab>/config/logo.png
\end{verbatim}

\noindent Clicking on the resulting logo will display the text found in {\tt <lab>/config/about.txt}.
\bigskip
Go to the {\tt \$LABTAINER\_DIR/scripts/gns3} directory:
\begin{verbatim}
      cd $LABTAINER_DIR/scripts/gns3
\end{verbatim}

Create modified Docker images for the lab:
\begin{verbatim}
      ./noNet.py telnetlab
\end{verbatim}
\noindent \textbf{NOTE:} Make note of the container image names displayed.  You will use these when adding appliance templates below.

View the Labtainer network topology.
\begin{verbatim}
      ./showNet.py telnetlab
\end{verbatim}
\noindent Use the GNS3 gui to define GNS3 Docker container templates for each of the 
images you created, and to then use those templates to create a GNS3 instance of 
the lab.
\begin{itemize}
	\item Start the gui, e.g., run {\tt gns3}.  
	
	{\bf Note:} Starting the gns3 program with the \-\-student flag, e.g., 
	\begin{verbatim}	
	  gns3 -s 
	\end{verbatim}		
	will cause the GUI to hide toolbars and widgets that students should not interact with. The container state after quitting gns3 will also persist. Without the student flag a lab's containers state wil reset upon pressing start once.
	\item Open a new project, assigning the same name as the Labtainers lab, e.g., ``telnetlab''
	\item Use {\tt Browse all devices / Add appliance template} to add the new container images created
		via the {\tt noNet.py} command (DO NOT select the original container image names, e.g., having ``student'' as a suffix).  
Accept all defaults, except set the number of ``Adapaters'' to
		the quantity displayed using {\tt showNet.py} for each component.
	\item Drag each container image from the list on the left of the GUI onto the workspace.  (Do not try to fix
		the component names, that will be done later).
	\item Use the gui's {\tt Add a Link} function to connect each component per the output of {\tt showNet.py}. If a network has more than 2 componenets, add an ethernet hub and connect the components to the ethernet hub.
	\item Close GNS3 and gns3-server. Run the {\tt ./genNet.py telnetlab telnetlab} command to generate GNS3 network connection files, to set the logo, and to
		fix the component names.  You will not see the revised names until you restart the lab.
	\item Start the GUI and test the lab.
\end{itemize}
Changes made to Labtainers container images will be picked up by GNS3 the next time the GUI is started,
assuming you re-run the noNet.py command.  There is no need to redefine appliance templates.

\section{Remote access to containers}
See the \textit{Lab Designer User Guide} section on ``Remote access and control of Labtainers'' for information 
on remote management and remote access to containers within a lab.

\section{GNS3 interface changes for Labtainers}
Once the lab is defined for use in GNS3, it can be run by opening the associated project file.  The following features were added to the GNS interface:
\begin{itemize}
	\item Labtainers containers will be parameterized and initialized when the container is started.  It requires a student email address in
		{\tt ~/.local/share/labtainers/}
	\item Terminal windows defined for the Labtainers lab will be opened.
	\item Double clicking on a Labtainers component will open another Labtainers window similar to the {\tt moreterm.py} function.
	\item The ``check work'' button will perform automated assessment similar to the Labtainers {\tt checkwork} function.
	\item Lab manuals are displayed by clicking the ``Lab Manual'' button (question mark)
	\item The state of Labtainers containers is maintained after GNS3 is terminated.  A new fresh instance of the lab can be generated
		by pressing the "Restart Lab" button.
	\item All Labtainers containers support X11 applications using the host system X11 server.
	\item Starting the gns3 program with the \-\-student flag will cause the GUI to hide toolbars and widgets that students should not interact with.
        \item Right clicking on a Labtainer node will display a pop-up menu including an option to insert a thumb drive.  This will cause
              a {\tt sudo mount} command to be run on the container, with arguments provided in start.config THUMB\_VOLUME configuration
              value for that container.
        \item Use of the {\tt -s} option when starting gns3 will cause any cloud endpoint nodes to be hidden from the student,
along with any links to them.  It will also hide any Labtainer containers having {\tt HIDE YES} in the start.config.
\end{itemize}
\subsection{Thumb drive simulation example}
Create a volume image file, e.g., by including the following in the container fixlocal.sh script:
\begin{verbatim}
    sudo mkdir /var/stuff
    sudo chown ubuntu:ubuntu /var/stuff
    cd /var/stuff
    dd if=/dev/zero of=usb.img bs=1k count=1k
    mkfs.ext2 -F usb.img
    sudo mkdir /media/usb1
    sudo mount -o loop usb.img /media/usb1
    sudo chown ubuntu:ubuntu /media/usb1
    echo “First file created” > /media/usb1/file1
    echo “Second file created” > /media/usb1/file2
    echo “Third file” > /media/usb1/file3
    sudo umount /media/usb1
\end{verbatim}
\noindent Alternately, the image can be created outside of labtainers and copied into the lab, e.g., via {\tt \_system/var}.

Then place the arguments to {\tt mount} in the start.config file for the appropriate container, e.g.,:
\begin{verbatim}
        THUMB_VOLUME -o loop /var/stuff/usb.img /media/usb1
\end{verbatim}
\noindent Then, when the user right clicks on the component, and selects ``insert thumb drive'', the volume will be mounted.

\section{History}
This section describes steps taken to create the Labtainers version of GNS3.
The steps outlined here are not intended to be repeated by developers.

The gns3 server was forked into https://github.com/mfthomps/gns3-server.git

The branch "2.1" was then cloned the tag set to tag v2.1.21:
\small
\begin{verbatim}
      git clone --single-branch --branch 2.1 https://github.com/mfthomps/gns3-server.git
      git checkout tags/v2.1.21
\end{verbatim}

\normalsize
\noindent A new ``labtainers'' branch was defined for the gns3 fork:
\begin{verbatim}
       git tag v1.2.1-labtainers
       git checkout -b labtainers
       git push --set-upstream origin labtainers
\end{verbatim}

The node.py script was altered to increase the timeout to account for parameterization of
containers.  This should be revisited to be perhaps less crude.

\bigskip
To add icons: 
\begin{itemize}
	\item Add icon to {\tt gns3-gui/resources/images}
	\item Add  <file>images/yourIconFileName</file> to the list in resources.qrc located in {\tt gns3-gui/resources}.
	\item Run gns3-gui/scripts/build\_pyqt.py --ressources
	\item Install qtcreator. 
	\item Open the {\tt gns3-gui/gns3/gui/main\_window.ui} file with qtcreator.
	\item Edit an object from the Action Editor bar, and click on the button with 3 dots on the 'Icon' row. 
	\item Select your icon from the 'images' section, press OK, and press OK again.
	\item After saving your changes, run gns3-gui/scripts/build\_pyqt.py --ressources.
	\item Run 'sudo python3 setup.py install' in the gns3-gui directory.
\end{itemize}
Added lab manual button and check work button.
\section{Notes}
The startup file is provided to start\_window.py, and that is provided to loadPath, and that results in
a call to Topology.loadProject.  The nodes and links are created in graphics\_view.py
\end{document}
