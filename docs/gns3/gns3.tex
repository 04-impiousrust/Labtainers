\documentclass[12pt]{article}
\usepackage{geometry}
\geometry{a4paper, total={170mm,257mm},left=20mm, top=20mm,}
\usepackage[colorlinks=true,linkcolor=blue,urlcolor=black]{hyperref}
\usepackage{bookmark}
\usepackage[autostyle, english = american]{csquotes}
\begin{document}
\begin{titlepage}
\title {Running Labtainers from within GNS3}
\maketitle

\vspace{2.0in}
This document was created by United States Government employees at 
The Center for Cybersecurity and Cyber Operations (C3O) at the Naval Postgraduate School NPS. 
Please note that within the United States, copyright protection is not available for any works created  
by United States Government employees, pursuant to Title 17 United States Code Section 105.   
This document is in the public domain and is not subject to copyright. 
\end{titlepage}
\tableofcontents
\newpage
\section {Introduction}
This document describes the integration of GNS3 with Labtainers.
The goal of this integration is to allow lab exercises developed for Labtainers to run within the
GNS3 environment.   From the student's perspective, they see the GNS3 GUI and select their lab as a standard GNS3 project.
Labtainer components will be visible in the GNS3 workspace along with any other components defined in the lab, 
e.g., GNS3 routers.  Pressing the "play" button starts that lab, and any Labtainers terminal windows that were
defined for the lab will open as the components come online.

All components within the running lab are known to GNS3 as standard GNS3 nodes.  Those that happen to be based on
Labtainers containers will call-back into Labtainers modules, e.g., to open new terminals, or perform configuration steps
during startup.  All component networking is via GNS3 network management.

The lab development workflow is summarised as follows:
\begin{itemize}
	\item Develop a lab using standard Labtainers tools and techniques, ignoring true routers, switches and such.
	\item Run a script that creates GNS3-Labtainers instances of the lab's Docker images
	\item Use GNS3 to lay out the topology of the newly created images, along with other devices such as routers.
	\item Run a script to automatically configure the GNS3 network configurations to match those defined for Labtainers
	\item Restart GNS3 and test the lab
	\item The new "lab" then consists of the GNS3 project directory and the Labtainers lab directory.
\end{itemize}

\section {Requirements}
It is assumed your development system is a recent Linux distribution.  You will require these packages:
\begin{itemize}
	\item pthon3-setuptools
	\item python3-dev
	\item python3-pyqt5
\end{itemize}
\noindent Install GNS3 from their website in order to get the ubridge software property configured.
\begin{verbatim}
https://docs.gns3.com/1QXVIihk7dsOL7Xr7Bmz4zRzTsJ02wklfImGuHwTlaA4/index.html
\end{verbatim}
Add yourself to the ubridge group and logout and in to get permissions set per GNS3 guidance.

\section{Cloning and installing GNS3 and Labtainers}
\subsection{Labtainers repo (optional)}
This subsection is optional, and is intended to give access to GNS3 updates to Labtainers directly
from github without waiting for formal Labtainers releases.  

Get the gns3 branch from Labtainers:
\begin{verbatim}
git clone https://github.com/mfthomps/labtainers.git
git checkout gns3 
\end{verbatim}
\noindent Define the {\tt LABTAINERS\_DIR} environment variable to point to the Labtainers directory.
(Log out and in, or create a new bash shell to take effect.)
Create the capinout executable:
\begin{verbatim}
cd $LABTAINER_DIR/tool-src/capinout
./mkit.sh
\end{verbatim}
\noindent Ignore the warning messages.
Run any Labtainers lab from the labtainer-student directory to test that it works and to capture a
Labtainers email identifier.

\subsection{GNS3 Repos}
Get the GNS3 server from the forked github repo.  First change your directory to where you want the repos to
exist.  Then:
\small
\begin{verbatim}
git clone --single-branch --branch labtainers https://github.com/mfthomps/gns3-server.git
cd gns3-server
git checkout tags/v2.1.21
sudo python3 setup.py install
\end{verbatim}

Change directory to where you want the repo, and then get the GNS3 GUI from the gns3 github repo:
\small
\begin{verbatim}
      git clone --single-branch --branch 2.1 https://github.com/mfthomps/gns3-gui.git
      cd gns3-gui
      git checkout tags/v2.1.21
      sudo python3 setup.py install
\end{verbatim}
\normalsize
\noindent Run the server from a terminal with LABTAINER\_DIR defined.

\begin{verbatim}
      gns3server --local --log /tmp/gns3log
\end{verbatim}
\noindent The server does not display to stdout.  Tail the /tmp/gns3log to see status.

\noindent Run the gui from a different terminal or tab:
\begin{verbatim}
    gns3
\end{verbatim}

\section{Updating local git repos}
Note below that your Labtainers repo uses the "gns3" branch -- and your GNS3 repo uses the "labtainers" branch.
From the local GNS3  repo:
\begin{verbatim}
      git pull origin labtainers
\end{verbatim}

From the local Labtainers repo:
\begin{verbatim}
git pull origin gns3
\end{verbatim}

\section{Updating remote repo}
This step is only for use by Labtainers framework developers.
The GNS3 repo:
\begin{verbatim}
git push --set-upstream origin labtainers
\end{verbatim}

\noindent The Labtainer repo:
\begin{verbatim}
git push --set-upstream origin gns3
\end{verbatim}


\section{Porting a Labtainers lab}
This example illustrates porting the telnetlab from Labtainers to run in the GNS3 environment.
It is assumed the gns3-server has been started.
This workflow can be further automated to automatically generate network connections.  Or to fully define the project
file based on predefined templates.

These steps assume the Labtainer Docker images exist on the machine, e.g., you've peformed a rebuild.py.  If not, run the Labtainers lab
to cause them to be pulled. 

\bigskip
If there is a logo that you wish to appear on the GNS3 display, put that file in 
\begin{verbatim}
      <lab>/config/logo.png
\end{verbatim}
\noindent Clicking on the resulting log will display the text found in {\tt <lab>/config/about.txt}.
\bigskip
Go to the {\tt \$LABTAINER\_DIR/scripts/gns3} directory:
\begin{verbatim}
      cd $LABTAINER_DIR/scripts/gns3
\end{verbatim}

Create modified Docker images for the lab:
\begin{verbatim}
      ./noNet.py telnetlab
\end{verbatim}

View the Labtainer network topology.
\begin{verbatim}
      ./showNet.py telnetlab
\end{verbatim}
\noindent Use the GNS3 gui to define GNS3 Docker container templates for each of the 
images you created, and to then use those templates to create a GNS3 instance of 
the lab.
\begin{itemize}
	\item Start the gui, e.g., run {\tt gns3}
	\item Open a new project, assigning the same name as the Labtainers lab, e.g., ``telnetlab''
	\item Use {\tt Browse all devices / Add appliance template} to add the new container images created
		via the {\tt noNet.py} command.  Accept all defaults, except set the number of ``Adapaters'' to
		the quantity displayed using {\tt showNet.py} for each component.
	\item Drag each container image from the list on the left of the GUI onto the workspace.  (Do not try to fix
		the component names, that will be done later).
	\item Use the gui's {\tt Add a Link} function to connect each component per the output of {\tt showNet.py}.
	\item Run the {\tt ./genNet.py telnetlab telnetlab} command to generate GNS3 network connection files, and to
		fix the component names.  You will not see the revised names until you restart the lab.
	\item Stop the GUI and restart it.  Then test the lab.
\end{itemize}
Changes made to Labtainers container images will be picked up by GNS3 the next time the GUI is started,
assuming you re-run the noNet.py command.  There is no need to redefine appliance templates.

\section{History}
This section describes steps taken to create the Labtainers version of GNS3.
The steps outlined here are not intended to be repeated by developers.

The gns3 server was forked into https://github.com/mfthomps/gns3-server.git

The branch "2.1" was then cloned the tag set to tag v2.1.21:
\small
\begin{verbatim}
      git clone --single-branch --branch 2.1 https://github.com/mfthomps/gns3-server.git
      git checkout tags/v2.1.21
\end{verbatim}

\normalsize
\noindent A new ``labtainers'' branch was defined for the gns3 fork:
\begin{verbatim}
       git tag v1.2.1-labtainers
       git checkout -b labtainers
       git push --set-upstream origin labtainers
\end{verbatim}

The node.py script was altered to increase the timeout to account for parameterization of
containers.  This should be revisited to be perhaps less crude.

\bigskip
Install qtcreator
Open the {\tt gns3-gui/gns3/gui/main\_window.ui} file

To add icons:
\begin{itemize}
\item Add to resources/images or wherever.
\item Edit resources.qrc
\item scripts/build\_pyqt.py --ressources
\item Use qtcreator to identify the icon in "resources"
\end{itemize}
Added lab manual button "?" and check work button.
\end{document}
