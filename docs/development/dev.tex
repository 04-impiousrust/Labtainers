The framework has not yet been adapted to use Linux package managers.
Currently, scripts are run from a workspace directory and python
paths are managed relatively between scripts.  

Student scripts, e.g., start.py, run from the trunk/scripts/labtainer-student directory.
That directory also contains the labutils.py, which contains most of the framework
functions.

When a student container is first started "docker exec" is used
to run parameterize.sh on the container.

That script also invokes hookBash.sh, which adds the bash
sdtin/stout capturing hook, and adds the startup.sh call
into the .profile.

The startup.sh scripts differ between instructor and student.  The latter
displays instructions.txt.  The former runs grading.
The startup.sh uses a lock to control which
terminal displays the instructions or grading.  In practice instruction
display and the running of the instructor.py script is done by
an xterm that explicitly runs the startup.sh directly, and thus
the startup.sh invoked by profile.sh typically never does anything.
HOWEVER... the startup.sh invoked by student will source a student_startup.sh if present.

Regression testing of grading functions is performed by labtainer-instructor/regress.py.
Expected results are stored in the labtainer/testsets directory.

\section{Developer Software Prerequisits}
\begin {itemize}
\item Subversion
\item Latex (texlive-full)
\end {itemize}


\section{Getting Labtainers from Subversion}
svn co https://tor.ern.nps.edu/svn/proj/labtainer
Change directory to trunk/setup-scripts and run ./build-docs.sh to build the PDF lab
manuals so that you can reference the manuals while you test or otherwise reference
existing labs.  (Please follow the lab manual and report discrepancies!)
Then run ./pull-all.sh to pull all the baseline images (so that your running of 
existing labs is more akin to what students and instructors do so we can better test that).
is more akin t
