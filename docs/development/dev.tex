The framework has not yet been adapted to use Linux package managers.
Currently, scripts are run from a workspace directory and python
paths are managed relatively between scripts.  

Student scripts, e.g., start.py, run from the trunk/scripts/labtainer-student directory.
That directory also contains the labutils.py, which contains most of the framework
functions.

When a student container is first started "docker exec" is used
to run parameterize.sh on the container.

That script also invokes hookBash.sh, which adds the bash
sdtin/stout capturing hook, and adds the startup.sh call
into the .profile.

The startup.sh scripts differ between instructor and student.  The latter
displays instructions.txt.  The former runs grading.
The startup.sh uses a lock to control which
terminal displays the instructions or grading.  In practice instruction
display and the running of the instructor.py script is done by
an xterm that explicitly runs the startup.sh directly, and thus
the startup.sh invoked by profile.sh typically never does anything.
HOWEVER... the startup.sh invoked by student will source a student_startup.sh if present.

Regression testing of grading functions is performed by labtainer-instructor/regress.py.
Expected results are stored in the labtainer/testsets directory.

\section{Developer Software Prerequisits}
\begin {itemize}
\item Subversion
\item Latex (texlive-full)
\end {itemize}


\section{Getting Labtainers from Subversion}
svn co https://tor.ern.nps.edu/svn/proj/labtainer
Change directory to trunk/setup-scripts and run ./build-docs.sh to build the PDF lab
manuals so that you can reference the manuals while you test or otherwise reference
existing labs.  (Please follow the lab manual and report discrepancies!)
Then run ./pull-all.sh to pull all the baseline images (so that your running of 
existing labs is more akin to what students and instructors do so we can better test that).

\section{Testing and Running Existing Labs}
There are situations where you will run an existing lab, e.g., to test it, or to 
observe some example.  When running labs, please refer to the lab manuals
so that they get reivewed and tested by different people.  Also, please first delete
the lab using trunk/setup\_scripts/removelab.sh to ensure that you are running the latest
version of the published lab.  If you find the lab to be broken, e.g., missing a file, please
attempt to run "rebuild.py" on the lab.  Report these findings to the lab author.  And always
run removelab.sh after you have run an existing lab via rebuild.py.  Again, the goal is to
force ourselves to run the distributed labs unless we have specific reasons to do otherwise.


\section{Automation and Distributions}
The mkdist.sh script runs on a Linux VM hosted on windows, and creates the distribution tar 
and copies it into a shared folder.  From that folder, it is copied to the 
\\my.nps.edu@SSL\DavWWWRoot\webdav\c30-staging\document\_library" and then "Publish to Live" is 
performed on the Liferay site.
Two prepackaged VMs are maintained: one for VirtualBox, and one for VMWare.  Each include
their respective guest additions.  The VMs are maintained on a Linux system using command line
utilities, e.g., VBoxManage.  The VMs are rigged to update labtainers, including a pull of
baseline images, on each boot until the first lab is commenced.  Scripts named "export*" are
used to created the appliance files.  The scripts re-import into test images, which must be
manually tested.  The WinSCP script pushes new applicance images to the CyberCIEGE download
directory on the C3O web server.  (Wine and WinSCP must be installed on the Linux host that
manages the VMs.

\section {Race condition on checklocal.sh output}
If an mynotify.py event causes an output from checklocal.py, that may conflict with
concurrent output from checklocal.py resulting from some program/script running.  In 
theory, the program/script should complete its run of checklocal before the program/script
actually gets to access the file that triggers a mynotify watch.  So, the latter's output
to the timestamped file is appended.  Further, the mynotify.py looks for an existing timestamped
file, and if not found, looks for one from the previous second.  This hack is an attempt to
keep the outputs merged.  It will fail if the access does not happen within a second of the
program start.  See the acl lab.

\section {installation sizes}
An initial install, including the base images, requires about 4GB.  Installing a larger lab,
e.g., snort, requires an additional 1GB.  Running bufoverflow added 22M.

\section {temporal logic considerations}
When evaluating results from logfiles containing timestamps use FILE\_TS or FILE\_TS\_REGEX
to ensure you get timestamped values for only matching records. Reliance on goals.config to
matchany can result in timestamped results that don't corrolate to the desired record. 

\section {parameterizing the start.config}
Is difficult.  The current parameterization features only affect containers, and leave no
persistent trail.  Thus, several students could share a computer and, via "redo.py", each 
perform that same lab with parameterization maintained for each student.  There is only one
start.config per Labtainers installation.   But... the start.config is only used during
docker create container.  So that could be driven from a copy that is parameterized as needed?
Except, the parameterized values are available for assessment.
