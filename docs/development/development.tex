\documentclass[12pt]{article}
\usepackage{geometry}
\geometry{a4paper, total={170mm,257mm},left=20mm, top=20mm,}
\usepackage[colorlinks=true,linkcolor=blue,urlcolor=black]{hyperref}
\usepackage{bookmark}
\usepackage{listings}
\usepackage{enumitem}
\usepackage[autostyle, english = american]{csquotes}
\usepackage{courier}
\usepackage{mdframed}
\begin{document}
\begin{titlepage}
\title {Labtainer Framework Development Guide}
\maketitle

\vspace{2.0in}
This document was created by United States Government employees at 
The Center for Cybersecurity and Cyber Operations (C3O) at the Naval Postgraduate School NPS. 
Please note that within the United States, copyright protection is not available for any works created  
by United States Government employees, pursuant to Title 17 United States Code Section 105.   
This document is in the public domain and is not subject to copyright. 
\end{titlepage}
\tableofcontents
\newpage
\section {Introduction}
This document is intended for use by developers who maintain the
Labtainer framework.  It does not address lab creation, which is 
covered in the \textit {Labtainers Lab Designer User Guide}.

\begin{flushleft} 
{\bf Note:} 
The Labtainer framework is based in Linux and is accessed through the Linux Terminal. 
\end{flushleft} 


\section{Developer Software Prerequisites}
\begin {itemize}
\item {\bf git}
\item {\bf make}
\item {\bf g++}
\item {\bf Latex} (texlive-full)
\item {\bf Docker (Community Edition)} [See {\bf Docker Installation} Section]
\item {\bf pip3} {\tt apt-get install python3-pip}
\item {\bf dateutil} {\tt pip3 install py-dateutil}
\end {itemize}
\section{Getting Labtainers from Github}

In the Linux Terminal change working directory into the directory you want to store Labtainers.
\begin{flushleft} Run this in the terminal: \end{flushleft}
\begin{center} {\tt git clone https://github.com/mfthomps/Labtainers.git} \end{center} 

\section{Setting up the Development Environment}
\begin {itemize}
\item Disable any auto-updates on your machine as this may interfere with 'apt-get' requests you may have during development.
\item Modify your ~/.bashrc file.
    \begin{enumerate}
    \item Add {\bf Labtainer\_DIR} as another global variable and set its value as the path to the {\tt  /Labtainers} directory. 
    \item Modify the \$PATH to include {\tt ./bin} and {\tt \$LABTAINER\_DIR/scripts/designer/bin}.
    \item In summary, your ~/.bashrc should include something like this:
	    \lstset{basicstyle=\footnotesize\ttfamily,
		    breaklines=true
	    	    framextopmargin=50pt
		    frame=single,
		    language=bash}
	    \begin{mdframed}
	    \begin{lstlisting}
if [[ ":$PATH:" != *":./bin:"* ]]; then
  export LABTAINER_DIR=$HOME/Labtainers
  export PATH="${PATH}:./bin:$LABTAINER_DIR/scripts/designer/bin"
fi
	    \end{lstlisting}
	    \end{mdframed}
    \item cd into \$LABTAINER\_DIR/setup\_scripts:
	    \begin{itemize}
    		\item Run pull-all.py to get all base docker images.
    		\item Run build-docs.sh to build the lab manuals for all labs.
	    \end{itemize}
    \item cd into \$LABTAINER\_DIR/tool-src/capinout and run mkit.sh
    \end{enumerate}
\end {itemize}


\section{Docker Installation}
For full and convenient installation of Docker and setting of Docker privileges, run  'install-docker-ubuntu.sh' in 'setup\_scripts', assuming you are developing in Ubuntu. {\bf Note:} Make sure to run the script as user (not sudo), so that your user can be added to the Docker group.\\ 
    
\noindent Reboot system, so that user receives Docker privileges.

    \begin {itemize}
    \item If on a different Linux distribution look in the same folder for your corresponding distribution (CentOS, Debian, Fedora). 
    \item If your Linux distribution is none of these, please view Docker's webpage documentation to learn how to install Docker and set Docker privileges on your machine.
    \item {\bf NOTE:} These install-docker scripts include the installation of other packages outside Docker that are necessary for building labs. 
    \end {itemize}

\section{Testing and Running Existing Labs}
When running labs, the goal is to force ourselves to run the distributed labs unless we have specific reasons to do otherwise.
\begin{enumerate}[label=\Alph*]
	\item )   To ensure that you are running the latest version of the published lab, first delete the current version of the lab using setup\_scripts/removelab.sh.  

	\item )   If you find the lab to be broken, e.g., missing a file, please attempt to run "rebuild.py" on the lab. rebuild.py will ouput a log of issues. Report these findings to the lab author.  

	\item )   Always run removelab.sh after you have run an existing lab via rebuild.py.  

	\item )   Please review the lab's manual very closely. This is so that both the lab itself and the lab's manual can receive feedback for improvement. 
\end{enumerate}

\section{Overview of Labtainer Elements}
The Labtainer framework implementation is primarily python scripts.  A number of the 
top level scripts share functions found in scripts/labtainer-student/{\bf bin}/labutils.py.  The 
top level scripts are organized as follows:

\begin{itemize}
	\item {\bf Student} \begin{itemize}
\item Student {\tt labtainers} (start) and {\tt stoplab} -- In the labtainers-student/bin directory, these run on the 
Linux host and manage the pulling, starting and stopping of containers.  They also coordinate
collection of student artifacts.
\item Student container scripts -- In the labtainers-student/{\bf lab\_bin} directory, these execute on
containers, e.g., to hook bash and parameterize containers.
	\end{itemize}

	\item {\bf Instructor} \begin{itemize}
\item Instructor {\tt gradelab} and {\tt stopgrader}-- Push student artifacts onto grader container and get assessment results.
\item Instructor container scripts -- perform grading functions.
	\end{itemize}

	\item {\bf Developer} \begin{itemize}
\item Developer building -- rebuild.py in labtainers-student/bin and labtainers-instructor/bin.
\item Publishing labs -- labtainers/distrib/publish.py
\item Base Labtainer images -- scripts/designer/bin, create and publish the base images.
	\end{itemize}	

	\item {\bf Other} \begin{itemize}
\item VM appliances -- //host\_scripts, update and publish VM appliances as OVA files for 
VirtualBox and VMWare.
\item Regression testing of grading functions is performed by labtainer-instructor/regress.py.
Expected results are stored in the labtainer/testsets directory.
\item Regression testing of labs and grading combined: scripts in testsets/bin; data sets
are not distributed, they are in labtainer/simlab/<labname>
       \end{itemize}

\end{itemize}

\section {Control Flow}
Student scripts, e.g., start.py, run from the scripts/labtainer-student directory.
That directory also contains the bin/labutils.py, which contains most of the framework
functions.

When a student container is first started "docker exec" is used
to run parameterize.sh on the container.

That script also invokes hookBash.sh, which adds the bash
sdtin/stout capturing hook, and adds the startup.sh call
into the .profile.

The startup.sh uses a lock to control which
terminal displays the instructions or grading.  In practice most
instructions are now pdf files.
The startup.sh invoked by student will source a student\_startup.sh if present.

\section{Automation and Distributions}
The Labtainer framework is distributed via the c3o website as a tar file, or, optionally a
VM applicance (both VMWare and VirtualBox).  The Docker images are distributed via the Docker Hub.

The labtainer/distrib/mkdist.sh script runs on a Linux VM hosted on windows or Linux, and creates the distribution tar 
and copies it into a shared folder.  The mk-devel.sh script makes the developers version of the tar.
From that shared folder, the two tar files are copied to the 
\begin{verbatim}
\\my.nps.edu@SSL\DavWWWRoot\webdav\c30-staging\document\_library" 
\end{verbatim}
\noindent and then "Publish to Live" is 
performed on the Liferay site.

The distributions are created from a local git repo, i.e., it does not pull a new repo.  The scripts first highlights
changes that have not yet been locally committed.  The user must press q to continue, potentially ignoring local changes.

Two prepackaged VMs are maintained: one for VirtualBox, and one for VMWare.  Each include
their respective guest additions.  The VMs are maintained on a native Linux system using command line
utilities, e.g., VBoxManage.  The VMs are rigged to update labtainers, including a pull of
baseline images, on each boot until the first lab is commenced.  Scripts named "export*" are
used to created the appliance files.  The scripts re-import into test images, which must be
manually tested.  The WinSCP script pushes new applicance images to the CyberCIEGE download
directory on the C3O web server.  (Wine and WinSCP must be installed on the Linux host that
manages the VMs.

New baseline images are created using scripts/designer/bin/create\_all.sh.  Note its comment
about deleting all docker images first.  When new baselines are created, use the labtainer-scripts
on the native Linux system to update the VM appliances so they contain the latest baseline images.
After the VM starts and updates the baseline images, use:
\begin{verbatim}
sudo dd if=/dev/zero of=/emptyfile bs=1M
sudo rm -fr /emptyfile
\end{verbatim}
\noindent to zero unused space and then run
\begin{verbatim}
./poweroffVB.sh
./compact.sh
\end{verbatim}
\noindent to compact the VM image.  Then export it:
\begin{verbatim}
./exportVB.sh
\end{verbatim}
\noindent This will create the appliance OVA image, and will create a test
VM from that appliance.  The test VM will start.  Use that to run ad-hoc
tests.

Do the same for vmware.

Then push the images to the web server

The appliances automatically update the baselines and the Labtainer scripts on boot, so there
is only really advantage to doing this for baseline changes, since they take a while to download.
After running the poweron/poweroff scripts, then run the exportVM.sh to 

\section {installation sizes}
An initial install, including the base images, requires about 4GB.  Installing a larger lab,
e.g., snort, requires an additional 1GB.  Running bufoverflow added 22M.

\section {Maintenance notes}
When working on a significant change, do so in a distinct branch:
\begin{itemize}
\item Create a new branch: {\tt git checkout -b newbranch}
\item Make changes 
\item Push changes in progress {\tt git push --set-upstream origin newbranch}
\item Pull new branch from other repo -- first checkin all local changes or stash. 
\begin{itemize}
\item {\tt git pull}
\item {\tt git checkout newbranch}
\end{itemize}
\item After testing, merge to branch into master:
\begin{itemize}
\item Checkout master: {\tt git checkout master}
\item merge: {\tt git merge newbranch}
\end{itemize}
\end{itemize}

\section {Notes}
\subsection {Race condition on checklocal.sh output}
If an mynotify.py event causes an output from checklocal.py, that may conflict with
concurrent output from checklocal.py resulting from some program/script running.  In 
theory, the program/script should complete its run of checklocal before the program/script
actually gets to access the file that triggers a mynotify watch.  So, the latter's output
to the timestamped file is appended.  Further, the mynotify.py looks for an existing timestamped
file, and if not found, looks for one from the previous second.  This hack is an attempt to
keep the outputs merged.  It will fail if the access does not happen within a second of the
program start.  See the acl lab.

\subsection {temporal logic considerations}
When evaluating results from logfiles containing timestamps use FILE\_TS or FILE\_TS\_REGEX
to ensure you get timestamped values for only matching records. Reliance on goals.config to
matchany can result in timestamped results that don't corrolate to the desired record. 

\subsection {parameterizing the start.config}
A copy of the parameterized version of start.config is placed into 
labtainer-student/.tmp/<lab>/.  This ensures that subsequent runs of the lab always
have the same psuedo random values.

\subsection {Packaging}
The framework has not yet been adapted to use Linux package managers.
Currently, scripts are run from a workspace directory and python
paths are managed relatively between scripts.  

\subsection{Todo}
Change smoke test to look for email in expected results and set that as the email before starting a lab.
Validation should catch results.config naming of non-existent container.

\subsection{Docker problems}
The check\_nets.py tests for problems that sometimes crop up in Docker.  These include Linux routes defined
on the host for container networks that no longer exist.  And, loopback devices that are not properly deleted?

\end{document}
