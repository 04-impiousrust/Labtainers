\documentclass[12pt]{article}
\usepackage{geometry}
\geometry{a4paper, total={170mm,257mm},left=20mm, top=20mm,}
\usepackage[colorlinks=true,linkcolor=blue,urlcolor=black]{hyperref}
\usepackage{bookmark}
\usepackage{listings}
\usepackage{enumitem}
\usepackage[autostyle, english = american]{csquotes}
\usepackage{courier}
\usepackage{mdframed}
\begin{document}
\title {Labtainer Instructor Modules (IModules)}
\maketitle

\section{Introduction and Purpose}
This guide describes how instructors can add content to Labtainers.
Instructors extend Labtainers with new labs or customized versions of existing
labs by defining IModules and directing their students to enable the IModules within
their individual Labtainers instances. 
(Or, instructors can enable IModules in VMs, and direct students to use those).
The scope of instructor-generated extensions can range from modified lab manuals 
to new Labtainer exercises.  The Labtainers framework provides tools
to assist instructors in creating and publishing these extensions. 

\section{Labtainer distribution strategy}
To understand how IModules are distributed, it is helpful to first review the general
Labtainers distribution strategy.  A Labtainers installation, (e.g., the initial content of a Labtainers VM 
appliance, or the results of installing from the distribution),  includes the scripts and configuration files 
needed to run all Labtainers exercises.  The installation initially only includes a small number of 
Docker container images that provide the core of container images for each of the labs.   
When a student first starts a given lab, the framework retrieves all Docker image layers 
required for that lab.  These layers are retrieved from the Docker Hub, and build upon the core images
present in the initial distribution.   The scripts and configuration files are 
published as a tar archive on the Labtainers website.  Whenever a Labtainers installation is updated, 
the archive is retrieved from the website and used to update the installation.

Files needed to create Docker images are typically not distributed in Labtainers distributions, but are 
installed when the user runs the update-designer script.  These files are drawn from a separate 
tar archive on the Labtainers website.  

\section{Imodule distribution strategy}
Instructors place archives on a web server and student
instances of Labtainers retrieve those archives from the web server while retrieving other
Labtainer updates.  When creating new labs, instructors publish the lab Docker images to
DockerHub, where they'll be retrieved by the framework when students run that lab.
While the publishing of extensions does not depend on any particular 
source control system, supporting tools that simplify archive creation are built around git.  

Archives published by instructors are tar files that include only changed and new files,
relative to the Labtainers baseline.  Inclusion of unchanged (relative to the 
Labtainers baseline) files is discouraged, as is publishing only deltas from previous
IModule publications.  Put another way, an IModule will contain any 
and all files
necessary for running, (not building), all new labs -- or to modify existing labs, 
relative to the Labtainers baseline as defined by the GitHub master repository.

Support tools simplify creation of IModule tar files through use of git attributes.
Instructors who chose not to use git are responsible for creating a tar of selected
files -- which may be trivial, e.g., if the IModule consists of lab manual modifications
or new lab guides.  Paths within tar files will be relative to the labtainers/lab
directory.  For example, a revised telnet-lab manual would have the path:
\begin{verbatim}
telnet-lab/docs/telnet-lab.pdf
\end{verbatim}
\noindent Note the modified source, e.g., docx files, need not be included in the IModule
archive, though the support tools do include them.

Typically, each participating instructor will publish a single archive (i.e., a tar file)
at a publically accessible URL specific to the instructor or institution. The URL 
is distributed to students and entered into their Labtainers 
instance using the {\tt imodule} command \footnote{The full URL is published because many
web hosting systems, e.g., box.com make it impossible to construct URLs from relative paths}.  
For example, if the instructor publishes
at \url{https://myschool/mystuff/labtainers/imodule.tar}, the students would each issue
this command to Labtainers:
\begin{verbatim}
imodule myschool/mystuff/labtainers/imodule.tar
\end{verbatim} 

The student labs will be updated to include those IModules the next time the student runs
either {\tt update-labtainer.sh} or {\tt imodule -u}.

IModule support tools rely on instructor contributions existing in local git repositories.
The tools do not reference remote repositories.  IModule repositories have no relationship
to the main Labtainers repository, and should be managed within Labtainer
distributions rather than within local repo copies of the main Labtainer repository.   \footnote{In general,
instructors and lab designers are encourage to work from Labtainer distributions rather
than repos pulled from the Labtainers repo at GitHub to avoid git repository conflicts.}

\subsection {Examples}  
These examples assume the instructor is working from a Labtainers distribution, e.g., one
of the VM appliance.
\subsubsection {Modify a lab manual for the telnet-lab}
In this example, the instructor wants his or her students to work with a customized version
of the telnet-lab manual.  
\begin{itemize}
\item Change directory to labtainer/labs
\item Initialize the git archive: 
\begin{verbatim}
git init  
\end{verbatim}
\noindent (Do this only once, no need to repeat for each IModule.)
\item Add the original Labtainer file as the baseline: 

\begin{verbatim}
git add telnet-lab/docs/telnet-lab.docx
\end{verbatim}
\item Edit the telnet-lab/docs/telnet-lab.docx file
\item Commit your change: 
\begin{verbatim}
    git commit telnet-lab/docs/telnet-lab.docx
\end{verbatim}
\end{itemize}

This change has no effect on any Docker container, so we need only generate the 
updated tar:
\begin{verbatim}
    create-imodules.sh
\end{verbatim}

\noindent Then publish the imodule.tar to the website.
\subsubsection{Create a new lab}
In this example, the instructor wants to create a new lab for use by his or her students.
This example assumes the instructor has created a DockerHub registry that is publicly accessible.
\begin{itemize}
\item Change directory to labtainer/labs
\item Initialize git archive: git init  (Do this only once, no need to repeat for each IModule.)  
\item Create the lab per the Lab Designer User Guide, for this example, we assume the lab is my-new-lab.
\item Include the name of your DockerHub registry the lab config/start.config file.
\item Complete development and testing of the lab, e.g., build a SimLab test.
\item While in the my-new-lab directory, run {\tt cleanlab4svn.py} to remove temporary files that should not be under source control.
\item While in the lab directory (parent of my-new-lab), add the lab to source control:
\begin{verbatim}
    git add my-new-lab
    git commit my-new-lab -m "Adding an IModule"
\end{verbatim}
\item Publish the lab container images: 
\begin{verbatim}
    ./publish.py -d -l my-new-lab
\end{verbatim}
\noindent This will rebuild the lab container images and publish them to your DockerHub registry.  Note the {\tt -d} option directs the 
function to publish to the DockerHub registry named in your lab start.config file.  Otherwise, it will try to publish to a test registry.
Use of test registries is optional, and are described in the \textit{Lab Designer User Guide}.
\item Generate the updated IModule tar:
\begin{verbatim}
    create-imodules.sh
\end{verbatim}
\item Then publish the imodule.tar to your website.
\end{itemize}


\end{document}
