\documentclass[12pt]{article}
\usepackage{geometry}
\geometry{a4paper, total={170mm,257mm},left=20mm, top=20mm, }
\usepackage[colorlinks=true,linkcolor=blue,urlcolor=black]{hyperref}
\usepackage{bookmark}
\usepackage[autostyle, english = american]{csquotes}
\begin{document}
\title {Labtainer Instructor Guide\vspace{-4ex}}
\maketitle

\section {Introduction}
This manual is intended for use by instructors who assign and/or grade
labs using Labtainers.
Labtainers assume you have a Linux system, e.g., a virtual machine.  Refer to
in Appendix A of the \underline{Labtainer Student Guide} for installation of VirtualBox and a Linux system.
Note that any Linux system can be used as long as it supports Docker.

Labtainers provide a consistent execution environment for performing
laboratory exercises, and can simulate the execution of several different
computers interconnected via virtual networks.   

\subsection{Obtaining Labtainers and installing Docker}
The Labtainer framework is distributed as a tarball from:
\url{https://my.nps.edu/web/cisr/labtainers}
Untar this file into a permanent directory on your Linux system,
e.g., \verb ~/home.  From the directory into which you untarred the
tarball:
\begin{verbatim}
   cd labtainer
   ./install-labtainer.sh
\end{verbatim}

This script will install the latest version of Docker and packages required
by the Labtainer framework.  It will cause your Linux host to reboot when it
completes.
Note that older Linux distributions, e.g., Ubuntu 14.* lack the
\textit{realpath} package, which should be installed prior to using Labtainers.

After the Linux host reboots, open a terminal to your Linux host and
change directory to wherever you untarred the tarball.

\subsection{Assigning a Lab}
Student instructions for using Labtainers are in the \underline{Labtainer Student Guide}.  Available labs are
listed via the {\tt labtainer} script:
\begin{verbatim}
    cd labtainer/trunk/scripts/labtainer-instructor
    labtainer
\end{verbatim}

\subsection{Assessing a Lab}
Collect all of the lab zip files from each student into the Labtainer xfer directory, which
is typically at
\begin{verbatim}
    ~/labtainer_xfer/<labname>
\end{verbatim}
All labs are assessed from the same Labtainer instructor workspace:
\begin{verbatim}
    labtainer/trunk/scripts/labtainer-instructor
\end{verbatim}
\noindent Start the lab containers to initiate automated assessment of a lab:
\begin{verbatim}
    labtainer <labname>
\end{verbatim}
\noindent You will see a summary of what is assessed for the named lab, and the framework
will then start the containers, one of which will perform automated assessment.

One of the resulting virtual terminals, titled ``GOAL\_RESULTS'' will display the assessment results.
These same results can also be generated by running this command in a virtual terminal:
\begin{verbatim}
    instructor.py
\end{verbatim}
\noindent This will generate the grading output file and display the file name.  In
addition to the grading output, a directory is created that will contain each student's
home directory content.  And container-specific artifacts are found in:
\begin{verbatim}
    <student_dir>/<container>/.local/result
\end{verbatim}

\noindent While running the lab, if you require more virtual terminals, use:
\begin{verbatim}
    moreterm.py <labname> <container>
\end{verbatim}
\noindent Where \textit{container} is the host name of the computer (container) to which to attach the terminal.

\noindent When you are finished, or wish to stop working, type:
\begin{verbatim}
    stoplab
\end{verbatim}
The stoplab command will terminate the containers.  This is recommended to avoid network
name conflicts when assessing other labs.

\end{document}
