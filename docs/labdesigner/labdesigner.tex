\documentclass[12pt]{article}
\usepackage{geometry}
\geometry{a4paper, total={170mm,257mm},left=20mm, top=20mm,}
\usepackage[colorlinks=true,linkcolor=blue,urlcolor=black]{hyperref}
\usepackage{bookmark}
\usepackage[autostyle, english = american]{csquotes}
\begin{document}
\begin{titlepage}
\title {Labtainer Lab Designer User Guide}
\maketitle

\vspace{2.0in}
This document was created by United States Government employees at 
The Center for Cybersecurity and Cyber Operations (C3O) at the Naval Postgraduate School NPS. 
Please note that within the United States, copyright protection is not available for any works created  
by United States Government employees, pursuant to Title 17 United States Code Section 105.   
This document is in the public domain and is not subject to copyright. 
\end{titlepage}
\tableofcontents
\newpage
\section {Introduction}
This manual is intended for use by lab designers wanting
to create or adapt cybersecurity labs to use the Docker
container-based lab framework known as ``Labtainers''.
The Labtainer framework is designed for use with computer and network security
laboratory exercises targeting Linux environments, and it is built around 
standard Linux Docker containers.  A Labtainer exerciese may include multiple 
networked components, all running locally on a student's computer, but without
the performance degredation associated with running multiple virtual machines.

While most Labtainer exercises focus on exploring concepts via the Linux command line -- GUI based
applications, e.g., browsers and Wireshark are also supported. 

\subsection {Benefits of Labtainers}

Deploying cybersecurity labs using this framework
provides three primary benefits:

\begin{enumerate}
\item The lab execution environment is controlled and consistent
across all student computers regardless of the Linux distribution
and configuration present on individual student computers.  
This allows each lab designer to control
which software packages are present, the versions of libraries and
specific configuration settings, e.g., /etc file values. These configurations
may vary between labs, and they may vary between multiple containers in
a single lab.

\item Assessment of student lab activity can be automated through a
set of configuration files that identify expected results, thus
relieving lab instructors from having to individually review detailed lab
results.

\item Labs may be automatically ``parameterized'' for each student such that
students cannot easily copy results from another student or from internet
repositories.  
\end{enumerate}

Labtainers provide the advantages of a consistent
execution environment without requiring
an individual Virtual Machine (VM) per lab, and without requiring all labs to be adapted for
a common Linux execution environment.   These benefits can be realized 
whether or not labs are configured for automatic assessment, 
or are parameterized for each student.

Exercises that include multiple networked computers illustrate an advantage 
of using containers over VMs, namely, containers require significantly less resources
than do VMs.  A student laptop that struggles to run two or more VMs can readily 
run multiple containers simultaneously, as shown in this 50 second demonstration: \url{https://youtu.be/JDV6jGF3Szw} 

Lab designers enhance labs to include automated assessment using directives built into the famework 
For example, 10 rather simple directives can evaluate the following question:

``Was there any
single iptables configuration during which the student used nmap to demonstrate that:
\begin{itemize}
\item The remote workstation could reach the HTTPS port but not the SQL port, and,
\item The local workstation could reach the HTTPS port and the SQL port.''
\end{itemize}

\section {Overview of the student environment and workflow}
Labtainers support laboratory exercises designed for Linux environments,
ranging from interaction with individual programs to labs that include
what appear to be multiple components and networks.  Students see and interact with Linux
environments, primarily via bash shell commands. In general, the Labtainer
framework implementation is not visible to the student, and the Linux
environment as seen by the student is not augmented to support the framework.

Labtainers are intended for use on individual student computers.
The computer utilized by a student must include the Linux operating system, e.g.,
as a single VM.  This Linux operating system, referred to herein
as the \textit{Linux host}, can be any distribution and version
which supports Docker.  Students download and expand a tarball, and run
an installation script as described in the \textit{Labtainer Student Guide}
(This tarball may someday be replaced by standard Linux distribution packages,
e.g., Debian and/or RPM packages.)  

It is suggested that the student's Linux host be a virtual machine that is
not used for purposes requiring trust.  Software programs contained in cybersecurity lab
exercises are not, in general, trusted.  And while Docker containers provide namespace
isolation between the containers and the Linux host, the containers run as privileged.

Students initiate any and all labs from a
single workspace directory on the Linux host.
To perform a specific Labtainer exercise, the student runs a \textit{start.py} command from
the Labtainer workspace, naming the lab exercise.  This results in one or more
containers starting up along with corresponding virtual terminals via which the 
student will interact with the containers.  These virtual terminals typically
present a bash shell.  Each container appears to the student as a separate
computer, and these computers may appear to be connected via one or more networks.  

When a student starts a given exercise for the first time, the framework fetches
Docker images from the Docker registry.  Docker manages container images as a set of
layers, providing efficient storage and retrieval of images having common components.
The initial Labtainer installation step pulls a few baseline images (about 1.5 GB) from 
the public
Docker registry, known as the \textit{Docker hub}.  Images for specific labs are pulled
from the Docker hub by downloading only those additional layers reqiured by that lab, and
which had not been previously pulled from the hub.  This is transparent to
the student, other than waiting for downloads to complete.

After the student performs the lab exercise, artifacts from the container
environments are automatically collected into an archive, (a zip file), that appears on
the student's Linux host.  The student forwards this archive file to the instructor,
e.g., via email or a learning management system (LMS).  The instructor collects student archive files into a common
directory on his or her own Linux host, and then issues a command that starts
the instructor container(s) for that lab.  This results in automated assessment of student lab
activity, (if the lab is designed for that), and creation of an environment
in which the instructor can review the work of each student.

Many cybersecurity lab exercises are assessed through use of reports in which students
describe their activities and answer specific questions posed by the instructor.  Labtainers
are intended to augment, rather than supplant this type of reporting.  The framework includes
mechanisms for automating the collection of student lab reports into the artifact archive files
that are collected by instructors. 

\section {Obtaining the labtainer development kit}
Installation of Labtainers is described in in the \textit{Labtainer Student Guide},  
which also includes instructions for installing an Ubuntu VM (if you do not already have a Linux system),
and the Labtainer framework.  Our website also distributes pre-packaged VM appliances that already have
Labtainers installed. Labtainers will work with any Linux
distribution that supports Docker containers.  If you already have Docker installed on a Linux system, 
reference the Student Guide for other dependencies. 

The difference between the development kit and the standard Labtainer distribution is primarily
just the lab definition files, which are withheld from the general distribution for efficiency.
If you have a Labtainer installation, you can get the developer files by going to your
labtainers directory, e.g., {\tt ~/labtainers} and running {\tt ./update-designer.sh}
You may then want to logout and login again, or run a new {\tt bash} shell because that script
sets some environment variables.

It is suggested that you periodically run that update script to get the latest lab definition files,
and to update framework software.   

\section {Creating new labs}
\label{sec:new_labs}
The most challenging and critical part of designing a new cybersecurity lab
is the design of the lab itself, i.e., identifying learning objectives and
organizing exercises to achieve those objectives.  The Labtainer framework
does not specifically address any of that.  Rather, the framework is intended
to allow you to focus more time on the design of the lab and less time on mitigating and
explaining system administration and provisioning burdens you are placing on 
students and instructors.

Typical steps for developing a new lab are:
\begin{enumerate}
\item Give the lab a name and create its computers using the {\tt new\_lab\_setup.py} script;
\item Choose the starting baseline configuration for each computer and add software packages
within a Dockerfile;
\item Define networks and connections to the lab computers in the lab's {\tt start.config} file.
\item Populate the user's HOME directory and system directories with lab-specific files.
\end{enumerate}
The remainder of this section covers the fist step and provides an example.  The 
following section \ref{execution environment}, covers the other three
steps.  After a lab is created, you can then optionally parameterize it per section \ref{parameterize} and/or
define criteria for automated assessment per section \ref{assessment}

\subsection{Create the first lab computer}
Labtainer exercises each have their own
directory under the ``labs'' directory in the project repository.
The first step in creating a new lab within the framework is to create
a directory for the lab and then cd to it.  The directory name will be the name
used by students when starting the lab.  It must be all lower case and not contain spaces.
\begin{verbatim}
    cd $LABTAINER_DIR/labs
    mkdir <new lab name>
    cd <new lab name>
\end{verbatim}

\noindent After the new lab directory is created, run the ``new\_lab\_setup.sh'' script.
\footnote {The {\tt \$LABTAINER\_DIR} will have been defined in your .bashrc
file when you installed Labtainers.  It should point to the {\tt labtainers/trunk}
directory.  You may need to start a new {\tt bash} shell to inherit the environment
variable.}

\begin{verbatim}
    new_lab_setup.py
\end{verbatim}
This will create a set of template files that you can then customize
for the new lab.  These template files are referenced in the discussion
below.
The result of running {\tt new\_lab\_setup.py} is a new labtainer lab that can be immediately run.  
While this new lab will initially only present you with a bash shell to an
empty directory on a Linux computer, it is worth testing the lab to understand the workflow.

\subsection{Testing the new lab}
Once a new lab directory is created, and the new\_lab\_setup.py has been run, then 
you can test the new, (currently empty) lab.  All student labs are launched from the
labtainer-student directory.  Lab development workflow is easiest if at least two
terminals or tabs are used, one in the new lab directory, and one in the labtainer-student
directory.  So, open a new tab or window, and then:

\begin{verbatim}
    cd $LABTAINER_DIR/scripts/labtainer-student
\end{verbatim}
Then start the container using the:

\begin{verbatim}
    rebuild.py [labname] 
\end{verbatim}
command, where labname is the name of the lab you just created.  

The rebuild.py command will remove and recreate the lab containers
each time the script is run.  And it will rebuild the container images if any of their configuration 
information has changed.  \footnote{The build process may generate warnings in red text, some of which are expected.  
These include an unreferenced ``user'' variable and the lack of apt-utils if apt-get is used to install packages in 
Dockerfiles.}  This is often necessary when building and testing new labs, to ensure the
new environment does not contain artifacts from previous runs.
The progress of the build, and error messages can be viewed in 
the labtainer.log file.  While developing, it is generally a good idea to tail this log in
a separate terminal:
\begin{verbatim}
    tail -f labtainer.log
\end{verbatim}

Note the ``rebuild.py'' command is not intended for use by students, they would use the ``start.py'' command.  
The rebuild.py utility compares file modification dates to Docker image creation dates to determine if
a given image needs to be rebuilt.  \footnote{rebuild.py will miss file deletion.  Thus, if files are deleted, you must
force the rebuild using the {\tt -f} option at the end of the rebuild.py command.}

Stop the containers with 
\begin{verbatim}
    stop.py 
\end{verbatim}
When you stop the container, a path to saved results is displayed.
This is the zip file that the student will forward to the instructor.

To test adding a ``hello world'' program to the new labtainer, perform the following steps:
\begin{itemize}
\item From the new lab directory window, cd \verb!$LABTAINER_DIR/labs/[labname]/[labname]!
\item Create a ``hello world'' program, e.g., in python or compiled C.
\item From the labtainer-student window, run rebuild.py [labname]
\end{itemize}
    
You should see the new program in the container's
home directory.  If you run the program from the container, and then stop the container
with stop.py, you will see the stdin and stdout results of the program within the
saved zip file.

The ``hello world'' program was placed in \verb!$LABTAINER_DIR/labs/[labname]/[labname]!.
The seemingly redundant ``labname'' directories are a naming convention in which the
second directory names one of potentially many containers.  In this simple example,
the lab has but one container, whose name defaults to the lab name.

The following sections describe how to further alter the lab execution environment seen by 
the student.

\subsection {Multiple containers}
The {\tt new\_lab\_setup.sh} script can be used to create additional containers for use
in the lab.  For example, from your new lab directory:
\begin{verbatim}
    new_lab_setup.sh -a joe_computer
\end{verbatim}
\noindent will create a second container for your lab,
named ``joe\_computer''.  If you again run the rebuild.py script, you will see two virtual
terminals, each connected to one of your two independent computers. 

Use 
\begin{verbatim}
    new_lab_setup.sh -h
\end{verbatim}
\noindent to view the operations available in that script.
The following sections describe how to configure the execution environments on your components,
and how to define virtual networks connected to the components.

\section {Defining the lab execution environment}
\label{execution environment}
A given lab typically requires some set of software packages, and some
system configuration, e.g., network settings, and perhaps some lab-specific
files.  It can include multiple containers, each appearing as distinct
computers connected via networks.  The execution environment seen by a
student when interacting with one of these ``computers'' is therefore defined
by the configuration of the associated container. 

Software packages are defined in each container's Dockerfile, described in 
the subsection below. That is followed by subsection \ref{start.config} describing network definitions,
(and other computer attributes) in the start.config file.  The remaining subsections then
described populating the user HOME directory and system directories, and methods for starting 
system services and miscellanious envrionment settings. 

\subsection {Docker files}
A default Labtainer-specific Dockerfile is placed in the new lab's ``dockerfiles'' 
directory when the new lab is created.  And additional Dockerfiles are added when the
{\tt new\_lab\_setup.sh -a} script adds computers to the lab. We use standard Docker file syntax, which is described at 
\url{https://docs.docker.com/engine/reference/builder/}

Simple labs should be able to use the default Dockerfile copied by the 
new\_lab\_setup.py script.  That Dockerfile refers to a base Labtainer
image that contains the minimum set of Linux packages necessary to 
host a lab within the framework.  The default
execution environment builds off of a recent Ubuntu image.

\noindent Each container has its own Dockerfile within the 
\begin{verbatim}
   $LABTAINER_DIR/labs/[labname]/dockerfiles
\end{verbatim}
\noindent directory.  The naming convention for dockerfiles is
\begin{verbatim}
    Dockerfile.[labname].[container_name].[role]
\end{verbatim}
where role is either ``student'' or ``instructor''.  The framework will use the student Dockerfile to
create the instructor container unless a distinct instructor file is present (e.g., if the designer
wishes the instructor to have unique software packages.) 

The first line of each Dockerfile identifies the baseline Labtainer image to be pulled from the Docker Hub.
The initial default image is a basic Ubuntu system with a minimal set of packages.  To use an
alternate image having additional networking packges (e.g., tcpdump, xinetd, sshd), change the first line to:
\begin{verbatim}
FROM mfthomps/labtainer.network
\end{verbatim}
\noindent Other alternate images include:
\begin{itemize}
\item labtainer.centos -- A CentOS server with systemd and the true ``init'' initial process.
\item labtainer.lamp -- A CentOS server with Apache, Mysql and PHP, (the LAMP stack)
\item labtainer.firefox -- An Ubuntu container with the Firefox browser.
\item labtainer.wireshark -- The labtaienr.network with wireshark added.
\item labtainer.java -- An Ubuntu container with the Firefox browser and the open JDK.
\end{itemize}
Refer to the Dockerfiles in {\tt \$LABTAINER\_DIR/scripts/designer/base\_dockerfiles} to see which
software packages are included within each baseline image. 

The Dockerfile is used to add packages to your container, e.g., 
\begin{verbatim}
RUN apt-get update && apt-get install -y some_package
\end{verbatim}

You will also see ``ADD'' commands in the Docker file that populate the container
directories with lab-specific files such as described in section \ref{home files}.

Next, you must also describe your containers within the \textit{start.config} file as described below.

\subsection{Container definitions in start.config}
\label{start.config}
Most single container labs can use the automatically generated start.config file
without modification.  Adding networks to containers and defining users other than the
default "ubuntu" user requires modification of the start.config file.
The following describes the major sections of that configuration file.  Most of the configuration
entries can be left alone for most labs.
\begin{itemize}
\item GLOBAL\_SETTINGS -- Beneath this keyword, the following values must be defined:

\begin{itemize}
\item GRADE\_CONTAINER [container name] -- All lab containers are available to the instructor while assessing student labs.
This setting identifies which of the lab containers will host automated grading functions.
\item HOST\_HOME\_XFER [dir name] --  Identifies the host directory via which to transfer student artifacts, relative to 
the home directory.  For students, this is where the zip files of their results end up.  For instructors, this is
where zip files should be gathered for assessment.
\item LAB\_MASTER\_SEED [seed] -- The master seed string for this lab.  It is combined with the student email
address to create an instance seed that controls parameterization of individual student labs.
\item REGISTRY [registry] -- The id of the Docker Hub registry that is to contain the lab images.
See \ref{publishing} for details on the use of this keyword.
\item COLLECT\_DOCS [yes/no] -- Optional directive to collect lab/docs content as part of student artifacts.
These are then available to the instructor in the labtainer\_xfer/[lab]/docs directory.  Also see \ref{instructions}.
\end{itemize}

\item NETWORK [network name] -- One of these sections is require for each network within the lab.  In addition to
providing a name for the network, the following values are defined:

\begin{itemize}
\item MASK [network address mask] -- The network mask, e.g., 172.25.0.0./24
\item GATEWAY [gateway address] -- The IP address of the network gateway used by Docker to communicate with the
host.  Please note that to define a different network gateway for the component, you should 
use the {\tt set\_default\_gw.sh}.  This GATEWAY field should not name the IP of any of your other components.  
See the the {\tt routing\_basics2} lab for examples.
\end{itemize}

\item CONTAINER [container name] -- One of these sections is required for each container in the lab.
These sections are automatically created by the {\tt new\_lab\_setup.py} script.  In addition to 
naming the container, the following values are defined: 

\begin{itemize}
\item TERMINALS [quantity] -- The number of virtual terminals to open and attach to this 
container when a lab starts.  If missing, it defaults to 1. Terminal titles are set to the 
bash shell prompt. A value of 0 suppresses creation of a terminal, and a value of -1 prevents
the student from attaching a terminal to the container. 
\item TERMINAL\_GROUP [name] -- All virtual terminals within the same group are organized as
tabs within a single virtual terminal.  Terminal group names can be arbitrary strings.
\item XTERM [title] [script] -- The named script is executed in a virtual terminal with the
given title.   The system will change to the user's home directory prior to executing the
script.  The script should be placed in container \_bin directory, i.e.,
\begin{verbatim}
    $LABTAINER_DIR/labs/[labname]/[container]/_bin
\end{verbatim}
\noindent If the title is ``INSTRUCTIONS'', no script is necessary and the instructions.txt file
in the container home directory will be displayed.
\item USER [user name] -- The user name whose account will be accessed via the virtual terminals. 
This defaults to ``ubuntu.''
\item PASSWORD [password] -- The password for the user name whose account will be accessed via the virtual terminals. 
This defaults to the user name defined above.
\item network name [ip address] -- Network address assignments for each network (defined via a NETWORK section), 
that is to be connected to this container.  A separate line should be entered for each network.  The given ip address 
can optionally include a MAC address assignment as a suffix following a colon, e.g., "172.25.0.1:2:34:ac:19:0:2".
\item SCRIPT [script] -- Optional script to provide to the Docker create command, defaults to ``bash''.  This must be set to
``NONE'' for CentOS-based labs (and likely fedora[test that]).
\item ADD-HOST [host:ip] -- Optional addition to the /etc/host file, a container may have multiple ADD-HOST entries.
\item X11 [YES/NO] -- Optional, defaults to NO.  If YES, the container mounts the TCP socked used by the hosts X11 server,
enabling the container to run applications with GUIs, e.g., browsers or wireshark.  See sql-inject as an example.  See the
Notes section at the end of this manual for tips on using Firefox and Wireshark.

\end{itemize}
\end{itemize}
  
A simple example of a two-container lab with suitable network settings in the start.config file can be found in 
\begin{verbatim}
    $LABTAINER_DIR/labs/telnetlab
\end{verbatim}
Entries in the start.config file can be parameterized as described in section \ref{parameterize}, e.g., to allocate
random IP addresses to components.


\subsection {Lab-specific files in the student's home directory}
\label{home files}
Files that are to reside in the student's \$HOME directory are placed in the 
new lab container directory.  For example, if a lab is to include a source code file, that
should be placed in the lab container directory. The file will appear in the student's
home directory within the container when the container starts.  The lab container
directory is at:  

\begin{verbatim}
    $LABTAINER_DIR/labs/[labname]/[container name]
\end{verbatim}
The container name in labs with a single container matches the labname by default.

All files and directories in the lab container directory will be copied to the student's HOME
directory except for the \_bin and \_system directories.
Each initial Dockerfile from the templates include this line:
\begin{verbatim}
    ADD $labdir/$lab.tar.gz $HOME
\end{verbatim}
to accomplish the copying. Except as noted below, Dockerfiles should not include any other ADD commands
to copy files to the HOME directory.
\subsubsection{Large or numerous files in the home directory} \label{large files}
If there are large sized, or a high quantity of files that are to be placed relative to a 
container home directory, those should be placed into a ``home\_tar'' directory at:
\begin{verbatim}
    $LABTAINER_DIR/labs/[labname]/[container_name]/home_tar/
\end{verbatim}
\noindent Use of this technique prevents these files from being collected as student artifacts, which
otherwise include copies of everything relative to the home directory.  This
can save considerable time and space, e.g., on the instructor's computer that must collect
all student artifacts.
The individual files should exist in the home\_tar directory, and the framework automatically
creates the tar file for transfer to the Docker image, (and will do so if an existing tar file
is older than any file in the directory).


\subsection{Lab-specific system files}
All files in the
\begin{verbatim}
    $LABTAINER_DIR/labs/[labname]/[container name]/_system
\end{verbatim}
directory will be copied to their corresponding paths relative to the root directory.
For example, configuration files for /etc should appear in \_system/etc/.

The initial Dockerfile from the templates include this line:
\begin{verbatim}
    ADD \$labdir/sys_\$lab.tar.gz /
\end{verbatim}
\noindent to accomplish the copying. 
If a lab contains a large quantity of system files, or large files, those
can be placed into the directory named:
\begin{verbatim}
    $LABTAINER_DIR/labs/[labname]/[container name]/sys_tar
\end{verbatim}
either as individual files, or in a ``sys.tar'' archive.  In the former case,
the framework will automatically create the sys.tar file.  This technique 
can save time in building lab images becauase the files do not need to be 
archived for each build.  

In general, files modified and maintained by the designer should go into the
\_system directory while static system files should go into the sys\_tar directory.

\subsection {System services}
The Dockerfile ``ENTRYPOINT'' command can be used to start a system service.  The general Docker 
model is that a single Docker container runs a single service, with logging being forwarded to 
the host.  Labtainers disregards this model because our goal is to make a container look more like a Linux
system rather than a conformant Docker container.  The Ubuntu-based Labtainer Dockerfiles include an
ENTRYPOINT command that launches a \textit{faux\_init} script that starts rsyslog, (so that system logs
appear in /var/log), and runs rc.local.  The network configuration of the baseline Dockerfile also starts xinetd,
which will then fork services, e.g., the sshd, per the /etc/xinet.d/ configuration files.  The faux\_init script
can be extended (or replaced) as needed by the lab.  The example telnetlab illustrates the use of this script
to provide network services.

The CentOS-based Labtainer containers, exemplified by centos-log, do not use the \textit{faux\_init}, rather they
use a Docker image built to run systemd and the /usr/sbin/init process.  The Dockerfile template includes
commented options for CentOS.  The config/start.config file must also include the 
\begin{verbatim}
    SCRIPT NONE
\end{verbatim}
\noindent line for the container.  The centos-log also provides an example of forcing 
the student to login using the traditional
login program, controlled by its bin/student\_startup.sh script.   


\subsection {Instructions for Students} \label{instructions}
Lab instructions for students can be displayed in a virtual terminal by placing an
``instructions.txt'' file within the home directory of one of the containers.  Refer to existing
labs for conventions.  Additionally, textual message can be displayed to the student before any 
of the lab virtual terminals are created.  Any text within the 
\begin{verbatim}
   $LABTAINER_DIR/labs/[labname]/docs/read_first.txt
\end{verbatim}
\noindent file will be displayed on the Linux host in the terminal in which the student
starts the lab.  Any ``LAB\_MANUAL'' string in that file will be replaced with the full path
to a [labname].pdf file within that same docs directory.  And ``LAB\_DOCS'' is replaced by the
path to the lab docs directory.  One intended use is to prompt the
student to open a PDF lab manual and perhaps read parts of it prior to continuing with the lab.
Another intended use is to display the path to a reporting template that a student is to use
for answering lab-specific questions and note taking.  If the name of the symbols are prefaced
by ``file://'', then the paths will display as links that can be opened via a right click.
If the start.config file includes ``COLLECT\_DOCS YES'', the content of the lab/docs directory will be
included with the student artifacts and available extracted into the intstrutor's 
labtainer\_xfer/[lab]/docs directory.

\subsection {Running programs in Virtual Terminals}
Programs can be started automatically within virtual terminals using two methods.
The first is the ``XTERM'' directive in the container section in the start.config file
described in \ref{start.config}.  That is intended for programs whose results are displayed
within the virtual terminal.  The second method is intended for user authentiation and
for starting GUI based programs
that will use the Linux host Xserver.  If a file exists at:
\begin{verbatim}
    $LABTAINER_DIR/labs/[labname]/[container name]/_bin/student_start.sh
\end{verbatim}
it will be executed from each virtual terminal created for the container.
See the sql-inject lab and the centos-log lab examples.


\subsection{Final lab environment fixup}
The initial environment encountered by the student is further refined using
the optional \_bin/fixlocal.sh script.  The framework executes
this script the first time a student starts the lab container.  For example,
this could be used to compile lab-specific programs afer they have been parameterized,
(as described below in \ref{parameterize}).  Or this script could perform final configuration adjustments
that cannot be easily performed by the Dockerfile.  These scripts are per-container
and reside at:
\begin{verbatim}
    $LABTAINER_DIR/labs/[labname]/[container name]/_bin
\end{verbatim}

\subsection{Automatic copying files from containers to the host}
This feature no longer has an intended use, but it is available if you have one.
Files are identified within
\begin{verbatim}
    $LABTAINER_DIR/labs/[labname]/config/files_to_host.config
\end{verbatim}
\noindent with a format of ``container:filename''.  Any named files within the home directory of
the named container will be copied to the host computer into a directory named by the lab, relative
to the Labtainer working directory.

\section{Parameterizing a lab}
\label{parameterize}
This section describes how to individualize the lab for each student to discourage
sharing of lab solutions.  This is achieved by defining symbols within source 
code or/and data files residing on lab containers. \footnote{An exception is the
start.config file, which can be parameterized as described in the syntax description.}  
The framework will replace these symbols with randomized values
specific to each student.  The config/parameter.config file identifies the files, and
the symbols within those files that are to be modified.  A simple example can be found in 
\begin{verbatim}
    $LABTAINER_DIR/labs/formatstring/formatstring/config/parameter.config
\end{verbatim}

That configuration file causes the string ``SECRET2\_VALUE'' within the file:
\begin{verbatim}
     /home/ubuntu/vul_prog.c
\end{verbatim}
to be replaced with a hexidecimal representation of a random value
between 0x41 and 0x5a, inclusive.

This symbolic replacement occurs when the student first starts the lab container,
but before the execution of the \_bin/fixlocal.sh script.  Thus, in the formatstring
lab, the executable program resulting from the fixlocal.sh script will be specific
to each student (though not necessarily unique).

Symbolic parameter replacement operations are defined within the config/parameter.config file.
Each line of that file must start with a \verb!"<parameter_id> : "!, which is any unique string, and
is followed by one of the following operations:

\begin{verbatim}
 RAND_REPLACE : <filename> : <symbol> : <LowerBound> : <UpperBound>
   Replace a symbol within the named file with a random value within a given
   range.  The random value generator is initialized with the lab instance
   seed.

     where: <filename> - the file name (file must exist) where <symbol> is 
                         to be replaced.  The file name is prefixed 
                         with a container name and a ":", (the container 
                         name is optional for single-container labs).  
                         This may be a list of files, delimited by semicolons. 
                         A file name of "start.config" will cause symbols
                         in the lab's start.config file to be replaced, e.g.,
                         to randomize IP addresses.
           <symbol> - the string to be replaced
           <LowerBound> and <UpperBound> specifies the lower and upper bound
                                       to be used by random generator
   example:

     some_parameter_id : RAND_REPLACE: client:/home/ubuntu/stack.c 
         : BUFFER_SIZE : 200 : 2000
     (all one line) will randomly replace the token string "BUFFER_SIZE" found in
     file stack.c on the mylab.client.student container with a number ranging from 
     200 to 2000

 RAND_REPLACE_UNIQUE : <filename> : <symbol> : <LowerBound> : <UpperBound>
   Identical to RAND_REPLACE, except randomly selected values are never resused
   within any given upper/lower bound range.  This is intended for use on IP
   addresses, e.g., 198.18.1.WEB_IP.  It is suggested that random ranges be selected
   such that they do not intersect any non-random address allocations.
 
 HASH_CREATE : <filename> : <string>
   Create or overwrite a file with a hash of a given string and the lab 
   instance seed.
     where: <filename> - the file name that is to contain the resulting hash.
                         The file name is prefixed with a container name 
                         and a ":", (the container name is optional for 
                         single-container labs).  
                         This may be a list of files, delimited by semicolons 
                         The file name is is optionall, (in cases of a single
                         container).  This may be a 
                         list of files, delimited by semicolons.
            <string> -   the input to a MD5 hash operation (after concatenation 
                         with the lab instance seed)
                       
                   
   example:
     some_parameter_id : HASH_CREATE : client:/home/ubuntu/myseed 
        : bufferoverflowinstance
   A file named /home/ubuntu/myseed will be created (if it does not exist), 
   containing an MD5 hash of the lab instance seed concatentated with the 
   string 'bufferoverflowinstance'.
 
 HASH_REPLACE : <filename> : <symbol> : <string>
   Replace a symbol in a named file with a MD5 hash of a given string 
   concatenated with the lab instance seed.
     where: <filename> - the file name (file must exist) where <symbol> is 
                         to be replaced.  The file name is prefixed 
                         with a container name and a ":", (the container 
                         name is optional for single-container labs).  
                         This may be a list of files, delimited by semicolons. 
            <symbol> - a string that will be replaced by the hash
            <string> - a string contatenated with the lab instance seed and hashed

     example:
       some_parameter_id HASH_REPLACE : client:/root/.secret : 
           ROOT_SECRET : myrootfile
     The string "ROOT_SECRET" in file /root/.secret will be replaced with an MD5 hash
     of the concatenation of the lab instance seed and "myrootfile".
\end{verbatim}

The parameter\_id fields may be referenced during the automated grading function, described below
in section \ref{goals.config}. 

Parameterizing occurs subsequent to system "boot", but prior to running the fixlocal.sh script.
The ubuntu based images /sbin/faux\_init.sh  script includes logic to delay the starting of 
selected services until after parameterization occurs (see commented sections at the end of that script).

Parameterizing of the start.config file occurs prior to Docker container creation.  The framework 
modifies a copy of the file stored in /tmp/start.config and uses that when assigning attributes to containers,
e.g., IP addresses. Currently only IP addresses within the start.config can parameteterized (e.g., not user names).

\subsection{Simple Parameterization for Checking Own-work}
The simplest, though by no means robust, strategy for ensuring students
have turned in their own work, (vice getting a zip file from a friend and simply
changing the name of the file), is to individualize some file on one of the containers,
and then check that file and the archive file names during grading.  The framework does
this automatically and reports on any student archive that does not seem to have
originated from a Labtainer initiated with that student's email address.


\section{Automated assessment of student labs}
\label{assessment}
This section describes how to configure a lab for automated assessment of student work.
Note the framework does not require automated assessment, e.g., the
``results'' of a lab may consist entirely of a written report submitted by the student.
Support for automated collection of written reports is described in \ref{instructions}
and the use of COLLECT\_DOCS in the start.config file.

The goal of automated assessment is to provide instructors with some confidence that 
students performed the lab, and to give instructors insight into which parts
of a lab students may be having difficulty with.  The automated assessment functions are
not intended to standardize each student's approach to a lab, rather the goal is to permit
ad-hock exploration by students.  Therefore, lab designer should consider ways to identify
evidence that steps of a lab were performed rather than trying to identify everything a student
may have done in the course of the lab.

Automated assessment is achieved by first generating artifact files while the student works.  That
is described in the first subsection below.  Next, artifacts within those files are identified
as described in section \ref{results.config}.  The values of the resulting artifacts are then
compared to expected values, as per section \ref{goals.config}.

\subsection{Artifact files}
\label{artifact files}
The files from which artifacts are derived include persisent data, such as system logs and 
{\tt .bash\_history}, as well as 
timestamped snapshots of transitory data such as stdout of a program.  Lab designers can also generate
customized artifacts in response to student actions using scripts that automatically execute when selected
programs or utilities are executed -- or when selected files are accessed.   The following paragraphs
describe how these artifacts are generated.

The Labtainer framework use of timestamps allows designers to express temporal 
relationships between artifacts, and thus between events.  For example, the designer can determine if
two distinct artifacts were part of the same stdout stream.  Or if artifacts in the stdout stream from one 
program were occuring during the invocation of a different program that generated other specific artifacts.  
The framework also can incorporate
timestamps from standard log file formats, e.g., syslog, allowing the designer to determine if some logfile
entry occurred during the invocation of a program whose stdout stream contains selected artifacts.
As a more concrete example, the use of timestamps allows the designer to determine that a spcific web log
record occurred during invocation of some program that produced a specific artifact.

\subsubsection{Capturing stdin and stdout}
\label{stdin and stdout}
Each time the student invokes a selected program or utility, the 
framework captures copies of standard input and standard output, (stdin and stdout) into timestamped file sets.
This is transparent to the student.  (Also see the following section for capturing
program output other than stdout.)  These timestamped file sets, selected system logs,  and everything relative to
the student's home directory, are automatically packaged when the student completes the lab.
These packages of artifacts are then transferred to the instructor, (e.g., via email or a LMS), and 
ingested into the instructor's system where lab assessment occurs. Timestamped stdin and stdout files
are captured in \texttt{\$HOME/.local/result}

By default, stdin and stdout for all non-system programs is captured, e.g., the results of an ``ls'' command
are not captured.  The stdin and stdout of system programs will be captured if the program
names appear at the beginning of a line in the \textit{treataslocal} file at
\begin{verbatim}
    $LABTAINER_DIR/labs/[labname]/[container name]/_bin/treataslocal
\end{verbatim}
\noindent The basename of the treataslocal entries are compared to the basename of each command.
\footnote{In other words, if the treataslocal entry is: \texttt{usr/bin/nmap}, the path leading to nmap is ignored.}
Starting of services can be monitored through use of treataslocal entries having a ``.service'' suffix,
e.g., {\tt httpd.service} would generate timestamped artifact files whenever http was started (or restarted)
using systemctl or /etc/init.d.  See section \ref{time delimeter services} for the intended use of this feature.

Non-system programs can be excluded from stdin/stdout capturing by including their names in
a ``ignorelocal'' file in that same directory.  \footnote{These should not include path information, just the program name.}

\subsubsection{Capturing program file output}
\label{program output}
Sometimes program file output is of interest to automated assessment, e.g., the program
may not have useful stdout.
The treataslocal entries can include optional output file identifiers that
cause timestamped copies of specified files to be made whenever the named program terminates.
If program file output from local programs is to be captured in timestamp files (in addition
to the stdout and stdin), simply include those program names in the treataslocal file.
These output file identifiers are of the form:
\begin{verbatim}
program    delim_type:delim_value
        where delim_type is one of:
            starts -- the output file name is derived from the 
            substring following the given delim_value within the 
            command line typed by the student.  For example, 
            "dd starts:of=" for a command line of 
            "dd in=myfs.img of=newfile" would yield an output 
            file name of "newfile".
     
            follows -- the output file name is the command line 
            token following the given delim_value.  For example, 
            "myprogram follows:myprogram" for a command line of
            "myprogram outfile" would yield "outfile" as the output 
             file name.

            file -- the delim_value is the output file name
\end{verbatim}
\noindent The resulting timestamped files are located with the stdin and stdout files in .local/result

\subsubsection{Bash History}
The framework collects all student bash history into the \texttt{\$HOME/.bash\_history} and \newline
\texttt{/root/.bash\_history} files.  These files are
available for reference as an artifact file.  

\subsubsection{System logs}
All files referenced in the {\tt results.config} file, (described below in section \ref{results.config}
will be collected into the artifact archive.  

\subsubsection{Capturing information about the environment}
\label{precheck}
Some labs require the student to alter system configuration settings,
e.g., using the sysctl command to affect ASLR. A \textit{precheck.sh} script in:
\begin{verbatim}
    $LABTAINER_DIR/labs/[labname]/[container name]/_bin
\end{verbatim}
is intended to contain whatever commands are necessary to record the 
state of the system at the time a program was invoked.  The stdout of
the precheck.sh script is recorded in a timestamped \textit{precheck.stdout}
file.  The timestamp of this file will match the timestamp of the stdin and
stdout artifacts associated with the command that caused precheck.sh to run.
The precheck.sh is passed in the name of the program as an argument, thereby
allowing the designer to capture different environment information for different commands.

As another example, consider the file-deletion lab \textit{precheck.sh} script.
It mounts a directory, lists its content, and unmounts it.  This all occurs 
transparently to the student, and, in this example, helps confirm a specific file
was in fact deleted at the time of issuing a command to recover deleted content from
the volume.

In other situations, you may wish to capture environment information when selected
commands are executed, even though you have no interest in stdin or stdout of those
commands.  For example, imagine you want to capture the file permissions of /usr/bin/tcpdump
whenever that command is executed.  This can be achieved by including /usr/bin/tcpdump in a
list within a file at:
\begin{verbatim}
    $LABTAINER_DIR/labs/[labname]/[container name]/_bin/forcecheck
\end{verbatim}
\noindent and then include \texttt{ls -l /usr/bin/tcpdump} in the precheck.sh script.
Note that the \textit{forcecheck} list of programs must include the full path name.
The \textit{forcecheck} file can be used instead of a \textit{treataslocal} file entry
for those cases where stdin and stdout are not required for goal assessment.  An example of the use of
forcecheck can be found in the \textit{capabilities} lab.  

\subsubsection{Capturing file access events}
\label{notify}
File creation, reading and modification events can be recorded using a combination of
a {\tt notify} file and an optinoal {\tt notify\_cb.sh} script at:
\begin{verbatim}
    $LABTAINER_DIR/labs/[labname]/[container name]/_bin/
\end{verbatim}
\noindent The {\tt notify} file will name directory or file paths and the access modes of interest, one entry
per line, having this format:
\begin{verbatim}
    <file_path> <mode> [output file]
\end{verbatim}
\noindent
where the {\tt file\_path} is the absolute path to the file of interest, and {\tt mode} is 
one of the following:
\begin{itemize}
\item {\tt CREATE} Assumes the path is to a directory.  This will capture any file or directory
creation within the named directory.
\item {\tt ACCESS} will capture any read of the file named by the path.
\item {\tt MODIFY} will capture any write to the file named by the path.
\end{itemize}
\noindent The optional {\tt output file} will be used for the timestamped filename of the
output from the event (instead of the default {\tt notify.stdout.YYMMDDHHMM}
\noindent Each time an event occurs matching a criteria specified in the {\tt notify} file,
the {\tt notify\_cb.sh} script is invoked (if it exists), passing in the given path and access mode.  
The script is also provided with the most recent command issued by either the container user, or the
root account (whichever is more recent).
If there is no {\tt notify\_cb.sh} script, then the output consists of the file\_path, the most recent
command, and the associated user (e.g., root).
See the {\tt acl} lab for an example.

The output from the notify event is captured in timestamped files, just as
those resulting from events described in Section \ref{precheck}.  If the optional output file
provided in the {\tt notify} list is given as {\tt precheck}, then
events resulting from program invocation, e.g., due to use of
a {\tt forcecheck} file, can be recorded in the very same timestamped file as events resulting
from a {\tt notify} file.  In such cases, output from the former will preceed output from the latter
within the file.  The framework will append the {\tt notify} output to any timestamped {\tt precheck.stdout}
file that was created up to two seconds prior to the {\tt notify} event.  Inclusion of both outputs into
one timestamped file allows the designer to identify events that occured as part of a single program
invocation.  Again, see the {\tt acl} lab for an example.

\subsection{Artifact result values}
\label{results.config}
The automated assessment functions encourage labs to be organized into a set of distinct ``goals''.
For each goal, the lab designer identifies one or more specific fields or attributes of artifact files that
could be compared to ``expected'' values.  These lab-specific artifacts are identified within the
\texttt{instr\_config/results.config file}.  Artifact files are identified in terms of:
\begin{enumerate}
\item The program that was invoked
\item Whether the artifact is in stdin or stdout or is program output (prgout) as descrbed in section \ref{program output}
\item An explicit file name, either as an absolute path or relative to the user HOME directory.  These are intended
to be persistent log files, e.g., syslogs.
\end{enumerate}

One or more properties of each artifact file are assigned symbolic names, referred to herein as \textit{results}, which 
are then referenced in the goals.config file to assess whether results match expected values.  Directives within the
results.config file assign each result a value having one of three types:
\begin{itemize}
\item Boolean, e.g., did an artifact file contain a specific string or regular expression?
\item String, e.g., the third space-delimited token on the first line containing the string "Audience says:"
\item Numeric such as the quantity of lines in an artifact file, or the quantity of occurances of a string in an artifact file.
\end{itemize} 

There are typically multiple instances of each result, each with its own associated timestamp. The framework automatically
associates timestamps with results, thereby allowing the designer to express temporal relationships between results
as introduced in section \ref{artifact files} The timestamp associated with any given result will be derived from different
sources depending on the nature of the results.config directive:
\begin{itemize}
\item The timestamp of the artifact file.  For example, each stdout artifact file name includes a timestamp reflecting when the
program was invoked, (and its corresponding stdin file contains an entry reflecting when the program terminated).
\item A timestamped entry from a log file, e.g., an entry in a web log, that matches criteria specified in the results.config
directive.
\item The invocation times of {\tt time\_delimiter} programs, syntactically associated with system log artifact files.
This allows designers to temporally group syslog results that were generated between changes to system configurations
as defined by invocation of the {\tt time\_delimiter} program, e.g., a script that alters the routing table.  See section 
\ref{time delimeter} for additional information. 
\end{itemize}

Directives within the results.config file each have the following format:

\begin{verbatim}
  <result> = <file_id> : <field_type> : <field_id> [: <line_type> : <line_id>]
      where:
            result -- The symbolic name of the result, which will be referenced in 
                       the goals configuration file.  It must be alphanumeric, 
                       underscores permitted.
            file_id -- Identifies a single file, or the set of files to be parsed.  
                  The format of this id is:
                     [container_name:]<prog>.[stdin | stdout | prgout] | 
                           [container_name:]file_path[:time_delimiter]
                   where <prog> is a program or utility name whose stdin, stdout,
                   or program output (prgout) artifacts will include timestamps.  
                   The optional container_name identifies the container hosting 
                   the file.  Labs with a single container can omit this qualifier.  
                   Alternately, an explicit 
                   file_path is intended for log files of services that persist 
                   across multiple student operations.  If the given path is not 
                   absolute, it is relative to the container user's home directory. 
                   The wildcard character '*' can be used in place of <prog>,
                   i.e., *.stdin is for all stdin artifacts and *.stdout is for all 
                   stdout artifacts.  The optional time_delimiter qualifier is 
                   explained further below.
            field_type - Optional, defaults to "TOKEN", possible values include:
               TOKEN      -- Treat the line as space-delimited tokens
               PARENS     -- The desired value is contained in parenthesis
               QUOTES     -- The desired value is contained in parenthesis
               SLASH      -- The desired value is contained within slashes, 
                             e.g., /foo/
               LINE_COUNT -- The quantity of lines in the file. Remaining fields 
                             are ignored.
               CHECKSUM   -- The result value is set to the md5 checksum 
                             of the file.
               CONTAINS   -- The result value is set to TRUE if the file 
                             contains the string represented in field_id.
               FILE_REGEX -- The result value is set to TRUE if the file 
                             contains the regular expression represented in field_id.
               STRING_COUNT--The result value is set to the quantity of
                             occurances of the string represented in field_id.
               PARAM      -- The result value is set to nth parameter
                             (0 is the program name), provided in the 
                             program invocation.  
                             
               SEARCH     -- The result is assigned the value of the search 
                             defined by the given field_id, which is treated as an
                             expression having the syntax of pythons parse.search 
                             function.  E.g., "frame.number=={:d}" would 
                             yield the frame number.
               GROUP      -- Intended for use with "REGEX" line types, the 
                             result is set to the value of the regex group 
                             number named by the field_id.  Regular expressions
                             and their groups are processed using the python
                             re.search semantics.
                             
                              
            field_id -- An integer identifying the nth occurance of the field type.
                        Alternately may be "LAST" for the last occurance of the 
                        field type, or "ALL" for the entire line (which causes the 
                        field type to be ignored).  Or if field_type is SEARCH, the
                        field_id is treated as the search expression. 
                        If field_type is "CONTAINS", the 
                        remainder of the line is treated as a string to be 
                        searched for.  If field_type is "PARAM", the field_id is
                        the 1-based index of the parameter whose value is to be 
                        assigned, and no other fields should be present.
                        If field_type is "CHECKSUM", no other field is required.
            line_type - Identifies how the line is to be identified, values include:
                LINE           --  The line_id is an integer line number 
                                   (starting at one).  Use of this to identify 
                                   lines is discouraged since minor lab changes 
                                   might alter the count.
                STARTSWITH     --  the line_id is a string.  This names the 
                                   first occurrence of a line that starts with 
                                   this string. 
                HAVESTRING      -- The line_id is a string.  This names the 
                                   first occurrence of a line that contains the 
                                   string.
                REGEX           -- The line_id is a regular expression.  This names the 
                                   first occurrence of a line that matches the regular
                                   expression.  Also see the "GROUP" field_type.
                NEXT_STARTSWITH -- the line_id is a string.  This names the 
                                   line preceeding the first occurrence of a line 
                                   that starts with this string. 
                HAVESTRING_TS   -- Intended for use with log files that have 
                                   timestamped entries.  Each entry containing
                                   the string identified in line_id will have
                                   its result stored as a timestamped value
                                   as if it came from a timestamped stdout or 
                                   stdin file.  See the snort lab for an example.
                REGEX_TS        -- Similar to HAVESTRING_TS, but with REGEX semantics,
                                   including optional use of the GROUP field_type.
            line_id - See line_type above.
\end{verbatim}

\subsubsection{Converting artifact file formats}
Some artifact file formats are not easily referenced by results.config directives.
For example, a browser history file in the .sqlite format is binary.  Such files
can be processed into a more convenient form through use of a script at:
\begin{verbatim}
   $LABTAINER_DIR/labs/[lab]/instr_config/pregrade.sh
\end{verbatim}
\noindent Modify or expand on the default pregrade.sh script.
In general, the pregrade.sh script is expected to extract or convert
data from an artifact file, and write it into a new file in the .local/results
directory of the container.   

\subsection{Evaluating results}
\label{goals.config}
Results of student lab activity are assigned symbolic names by the results.config file
as described above.  These results are then referenced in the goals.config to evaluate whether
the student obtained expected results.  Most lab goals defined in the goals.config file
will evaluate to TRUE or FALSE, with TRUE reflecting that the student met the defined goal.
In addition to these binary goals, the designer can capture and report on quantities of events,
e.g., the number of times a student ran a specific program.
Once evaluated, a goal may determine the value of subsequent goals within the goals.config file, 
i.e., through use of boolean expressions and temporal comparisons between goals.  The evaluated
state of each goal can then contribute to an overall student assessment.

Student results may derive from multiple invocations of the same program or system utility.  
The framework does not discourage students from continuing to experiment and explore aspects of the 
exercise subsequent to obtaining the desired results.  In general, Labtainer assessment determines if the student
obtained expected results during any invocation of a program or system utility, or during a time period
delineated by timestamp ranges described in section \ref{results.config}.  \footnote{In those cases 
where the student is required to obtain the expected results during the final invocation of a program, 
the \textit{matchlast} goal type may be specified as described below.}

The goals.config file contains directives, each of which assigns a value to a symbolic name referred
to as the {\tt goal\_id}.  Each goal\_id may have multiple instances of timestamped values, with their
associated timestamp ranges inherited from results.  Examples of assigning values to a goal\_id include:
\begin{itemize}
\item A goal\_id is automatically created for each boolean result from the results.config file.  The timestamps
are directlly inherited from the results.

\item The value of a specific result is compared (e.g., do two strings match?) to a literal expected value.
A boolean goal\_id value is generated for each referenced result's timestamp.

\item The value of a specific result is compared to a parameterized value
generated from the student email address as described in section \ref{parameterize}.
A boolean goal\_id value is generated for each referenced result's timestamp.

\item Timestamps and boolean values of two different goal\_id's are compared.  For example, ``was
a TRUE value for {\tt result A} generated while a TRUE value for {\tt result B} was being generated?''  
A boolean goal\_id is generated for each
timestamp range of {\tt result B} within which falls at least one {\tt result A} timestamp.

\item A boolean expression consisting of multiple {\tt goal\_id}'s and boolean operators such as OR, AND, NOT\_AND.
A boolean goal\_id is generated for each timestamp range for which there is an instance of every goal\_id named
in the expression.
\end{itemize}

The following syntax defines each goal within the goals.config file.  While the syntax
may appear complex, most goals can be expressed simply as can be seen in section \ref{examples} and
in the Labtainer exercises distributed with the framework.

\begin{verbatim}
<goal_id> = <type> : [<operator> : <resulttag> : <answertag> | <boolean_expression> 
            | goal1 : goal2 | <resulttag> | value : subgoal_list]
  Where: 
    <goal_id> - An identifer for the goal. It must be alphanumeric 
                (underscores permitted).
    <type> - must be one of the following:
         matchany'    - Results from all timestamped sets are evaluated.
                         If the answertag names a result, then both that 
                         result and the resulttag must occur in the same 
                         timestamped set.  The 'matchany' goals are treated 
                         as a set of values, each timestamped based on the 
                         timestamp of the reference resulttag.
                      
         matchlast   - only results from the latest timestamped set are 
                         evaluated.
         matchacross - The resulttag and answertag name results.  The 
                         operator is applied against values in different 
                         timestamped sets.  For example, a "string_diff" 
                         operator would require the named results to have 
                         at least two distinct values in different 
                         timestamped sets.  Note: 'matchacross' cannot 
                         be used within the boolean expression defined below.
         boolean     - The goal value is computed from a boolean expression 
                         consisting of goal_id's and boolean operators, ("and", 
                         "or", "and_not", "or_not", and "not"), and parenthisis 
                         for precedence.  The goal_id's must be from goals defined 
                         earlier in the goals.config file, or boolean results
                         from results.config.  The goal evalutes to 
                         TRUE if the boolen expression evaluates to TRUE for any
                         of the timestamped sets of goal_ids, (see the 'matchany' 
                         discussion above).  The goal_id's cannot include any 
                         "matchacross" goals.
         count_greater The goal is TRUE if the count of TRUE subgoals in the 
                         list exceeds the given value.  The subgoals are 
                         summed across all timestamps.  The subgoal list is 
                         comma-separated within parenthesis.
         time_before -   Both goal1 and goal2 must be goal_ids from previous 
                         matchany, or boolean values from results.config
                         A timestamped goal is created for each goal2 
                         timestamped instance whose timestamp is proceeded 
                         by a goal1 timestamped instance. The goal for that
                         timestamp will be TRUE if the goal2
                         instance is TRUE, and at least one of the goal1
                         instances is TRUE. These timestamped goals can
                         then be evaluated within boolean goals.
         time_during - Both goal1 and goal2 must be goal_ids from previous 
                         matchany goal types, or boolean values from 
                         results.config.  Timestamps include a start and end 
                         time, reflecting when the program starts and when it 
                         terminates.  A timestamped goal is created for each
                         goal2 range that encompasses a goal1 timestamp.
                         The goal for that timestamp will be TRUE if the
                         goal2 instance is TRUE, and at least one goal1 instance
                         is TRUE. These timestamped goals can then be
                         evaluated within boolean goals.
         execute     - The <operator> is treated as a file name of a script to 
                         execute, with the resulttag and answertag passed to the 
                         script as arguments.  The resulttag is expected to be 
                         one of the symbolic names defined in the results.config 
                         file, while the answertag is expected to be a literal 
                         value or the symbolic name in the parameters.config file 
                         Note: the answertag cannot be a symbolic name from 
                         results.config 
         count       - If the remainder of the line only includes a resulttag,
                         then the goal value is assigned the quanity of 
                         timestamped files containing the given resulttag.  
                         Otherwise the goal value is assigned the
                         quantity of timestamped files having results
                         that satisfy the given operator and arguments.
         value       - The goal value is assigned the given resulttag value from
                         the most recent timestamped file that satisfies the resulttag.
 
<operator> - the following operators evaluate to TRUE as described below:
   string_equal -  The strings derived from <answertag> and <resulttag>
                     are equal.
   string_diff -   The strings derived from <answertag> and <resulttag>
                     are not equal.
   string_start -  The string derived from <answertag> is at the start of 
                     the string derived from <resulttag>.
           example:  answertag value = 'MySecret'
                     resulttag value = 'MySecretSauceIsSriracha'
   string_end -    The string derived from <answertag> is at the end of
                     the string derived from <resulttag>.
           example:  answertag value = 'Sriracha'
                     resulttag value = 'EatMoreFoodWithSriracha'
   integer_equal - Integers derived from <answertag> and <resulttag>
                     are equal.
   integer_greater - The integer derived from <answertag> is greater than
                       that derived from <resulttag>.
   integer_lessthan- The integer derived from <answertag> is less than
                       that derived from <resulttag>
   <executable_file> - If the type is 'execute' then <operator> is a filename of 
                       an executable.
             
<resulttag>  -- One of the symbolic names defined in the results.config file.
                The value is interpreted as either a string or an integer,
                depending on the operator as defined above.  Alternately, 
                for integer operators within matchany types,  this
                may be an arithmetic expression within parentheses.  For example,
                "(frame_number-44)".
                
<answertag>  -- Either a literal value (string, integer or hexidecimal), or a 
                symolic name defined in the results.confg file or the 
                parameters.config file:
 
                answer=<literal>    -- literal string, integer or hex value 
                                       (leading with 0x), interpretation depending 
                                       on the operator as described above.
                result.<symbol>     -- symbol from the results.config file
                parameter.<symbol>  -- symbol from the parameters.config file
                parameter_ascii.<symbol> -- same as above, but the value parsed as 
                                       an integer or hexidecimal and converted to an 
                                       ascii character.

\end{verbatim}
         Note that values derived from the parameters.config file are assigned the same values as
         were assigned when the lab was parameterized for the student.

\subsubsection{Assessment Report}
Evaluation of student results occurs on an instructor container, via a script named {\tt instructor.py}, which runs
automatically when the instructor starts the lab.  The script can also be run manually, e.g., to
test changes and additions to grading configuartion files.  It must be run from the HOME directory
on the container identified in the start.config for the lab as being the {\tt GRADE\_CONTAINER}
By convention, all goals and boolean results whose symbolic names are not prefaced with an
underscore ({\tt\_}), will have corresponding entries in the assessment report, located in 
the home directory in a file named {\tt <lab name>.grades.txt>}


\subsection{Assessment examples}
\label{examples}
The following examples illustrate some typical assessment operations as they would
be defined in the results.config and goals.config files.

\subsubsection{Do artifact files contain specific strings?}
Consider the labs/formatstring/instr\_config/results.config file for a few examples.  
The first non-comment line defines a result having the symbolic name ``\_crash\_sig'':
\begin{verbatim}
   _crash_sig = vul_prog.stdout : CONTAINS : program exit, segmentation
   _crash_smash = vul_prog.stdout : CONTAINS : *** stack smashing detected
\end{verbatim}
\noindent This result is TRUE for each timestamped stdout file resulting from running 
the vul\_prog program in which the file contains the string ``program exit, segmentation''.
The goals.config includes this goal:
\begin{verbatim}
   crash =  boolean : ( _crash_smash or _crash_sig )
\end{verbatim}
\noindent The value of the crash goal is TRUE if either result is TRUE.

\subsubsection{Compare value of a field from a selected line in an artifact file}
Again refrence the labs/formatstring/instr\_config/results.config file.  The third non-comment line
defines a result having the symbolic name ``origsecret1value'':  
\begin{verbatim}
   origsecret1value = vul_prog.stdout : 6 : STARTSWITH : The original secrets:
   newsecret1value = vul_prog.stdout : 6 : STARTSWITH : The new secrets:
\end{verbatim}
\noindent The timestamped results are
found by looking at stdout from the ``vul\_prog'' program, and finding the first line that starts with:
``The original secrets:''.  The result is assigned the value of the sixth space-delimited 
token in that line.  The ``newsecret1value`` assignment is similar.  The goals.config file includes:
\begin{verbatim}
modify_value = matchany : string_diff : newsecret1value : result.origsecret1value
\end{verbatim}
\noindent , which will be TRUE if any of the vul\_prog stdout files include a
``newsecret1value'' that differs from its ``oldsecret1value``.

\subsubsection{My desired artifacts are not in stdin or stdout, the program outputs a file}
See section \ref{program output}

\subsubsection{Distinguish log file entries generated before and after configuration changes}
\label{time delimeter}
When log files are named in results.config files, you can qualify the log file name
with the name of program whose invocation serves as a {\tt time\_delimiter} introduced
in section \ref{results.config}.
The {\tt time\_delimiter} identifies some
monitored program whose start times will be used to organize the log file into
a set of timestamped results.  This differs from use of HAVESTRING\_TS and
REGEX\_TS in that those store results as descrete timestamped values for each
timestamp found in the log file.  The {\tt time\_delimiter} timestamp values are
based on the start times of the monitored program.  The intended use is to
group results from programs whose actions result in entries in the log file.
This can be useful for grouping system log entries based on system configuration
changes (i.e., accomplished via the {\tt time\_delimiter} program). 

Consider a lab that directs students to alter iptables on a component using the
/etc/rc.local script.  The student is required to demonstrate a desired iptables
configuration by running nmap on various other components.  The instructor wants 
to confirm that some set of expected stdout from nmap running on different
components all occurred within a single
configuration of iptables, delimited by the running of rc.local.  In other words,
the student cannot succeed by altering iptables between invocations of nmap on
different components.  In this example, the {\tt file\_path} would name the iptables
log, and the {\tt time\_delimiter} would name rc.local.  In order to track invocations
of rc.local, we would add it to the {\tt treataslocal} file described in section \ref{stdin and stdout}.

Then, if the lab results.config were:
\begin{verbatim}
    _iplog = outer_gw:/var/log/ulog/syslogemu.log:rc.local : \
             CONTAINS : IPTABLES DROPPED
    _remote_nmap_443 = remote_ws:nmap.stdout : CONTAINS : 443/tcp open  https
    _remote_nmap_sql = remote_ws:nmap.stdout : CONTAINS : 3306/tcp open  mysql
    _local_nmap_443 = ws1:nmap.stdout : CONTAINS : 443/tcp open  https
    _local_nmap_sql = ws1:nmap.stdout : CONTAINS : 3306/tcp open  mysql
\end{verbatim}

\noindent The {\tt \_iplog} result would then have up to N+1 timestamped instances,
where N is the quantity of times that rc.local was executed. The first possible
timestamp would have a starting time of zero and an ending time of the very first
invocation of rc.local.  The nmap results would each have timestamps corresponding
to their times of execution.  Note the nmap results include results from two different
computers, ws1 and remote\_ws.

A goals.config file of:
\begin{verbatim}
    remote_nmap_443 = time_during : _remote_nmap_443 : _iplog
    remote_nmap_sql = time_during : _remote_nmap_sql : _iplog
    local_nmap_443 = time_during : _local_nmap_443 : _iplog
    local_nmap_sql = time_during : _local_nmap_sql : _iplog
    remote_correct = boolean : ((remote_nmap_443 and_not remote_nmap_sql) \
                                and local_nmap_443 and local_nmap_sql)
\end{verbatim}
\noindent would generate sets of nmap goals grouped into timestamps corresponding to the
{\tt \_iplog} results.  The {\tt remote\_correct} boolean expression could then be read as: 
``Was there any
single iptables configuration during which the student used nmap to demonstrate that:
\begin{itemize}
\item The remote workstation could reach the HTTPS port but not the SQL port, and,
\item The local workstation could reach the HTTPS port and the SQL port.
\end{itemize}

\subsubsection{Delimiting logs by starting services}
\label{time delimiter services}
Another example of the use of {\tt time\_delimiter} log file qualifiers is a web server,
and its corresponding httpd log.  It may be desired to group log entries generated during a
single configuration of the web server, delimited by the starting of the web server, e.g., via
{\tt sudo systemctl restart httpd}.  Here, our {\tt time\_delimiter} program is the use of systemctl
to start or restart the httpd.  Services are named in the {\tt treataslocal} file by giving them a
suffix of ``.service'', e.g.,
\begin{verbatim}
    httpd.service
\end{verbatim}
\noindent and that same name is used for our {\tt time\_delimiter}, e.g.,
\begin{verbatim}
   web_log = vuln-site:/var/www/csrflabelgg.com:httpd.service : CONTAINS : GET / HTTP/
\end{verbatim} 
\noindent will create time stamp ranges delimited by the starting of the web server.  Results from
other programs, (or other results derived from the web log), could then be similarly group using 
the {\tt time\_during} operation, e.g.,
\begin{verbatim}
   _some_goal = time_during : other_result : web_log
   _some_other_goal = time_during : yet_another_result : web_log
   success = boolean (_some_goal and _some_other_goal) 
\end{verbatim}
The success goal would only then be TRUE if the two goals each occurred during a single instance
of the web server configuration, as delimited by use of {\tt systemctl restart httpd}.

\section{Networking}
Most networking is simply a matter of defining networks and assigning them to containers
as described in \ref{start.config}.  

In addition to networks properties defined in the
start.config file, each container \texttt{/etc/host} file includes a ``my\_host entry'' that names
the host Linux.  By Docker default, each container includes a default gateway that
leads to the Linux host.  This allows students to scp files to/from the container and host.
It also allows the student to reach external networks, e.g., to fetch additional packages in
support of student exploration.

In many instances, the lab designer will want to define a different default route for a 
container.  Each container includes a \textit{set\_default\_gw.sh}
script that can be added to the \textit{/etc/rc.local} (or \textit{fixlocal.sh}) file to redefine the default gateway.
This script will automatically retain a route table entry so that the student can reach the ``my\_host''
address.  Additionally, those baseline images include a \textit{togglegw.sh} script that the student
can use to toggle the default gateway between one that leads to the host, and one defined for the lab.
This allows students to add packages on components having lab-specific default gateways.

\subsection{Realistic Network Routing and DNS}
Some labs will strive to represent realistic networking environments, 
e.g., several networked components including gateways and DNS servers.
To achieve that, you must override Docker, which automatically sets 
the container's /etc/resolv.conf file to use the
host system DNS resolution.  This is in addition to the default routes described
above.  While convenient, these mechanisms can distract and confuse students, particularly
when routing and DNS resolution are central to the point of the exercise, (e.g.,
a DNS cache poisoning lab).

These Docker defaults can be easily overridden to present a more realistic networking
environment.  A worked example of such a topology can be seen in the \textit{routing-basics} lab.
This lab includes the following properties that can be reproduced in other labs:
\begin{itemize}
\item Default routes to gateway components.
\item DNS definitions in /etc/resolv.conf that name gateway components.
\item Use of \texttt{iptables} in gateway components to implement NAT.
\item A hidden ISP component that exchanges network traffic with the host Linux system,
thereby allowing all visible components to include routing, DNS and iptables entries that do not
expose virtual networking tricks.  See section \ref{Networking Notes} for additional information.

\end{itemize}

\section{Building, Maintaining and Publishing Labs} \label{publishing}
Typically, when a Labtainer is started, the container's associated Docker images are pulled from
the Docker Hub if they are not already local on the Linux host.  When building and editing labs,
the designer desires to run images reflecting recent changes that have been made.  The framework
includes logic to identify dependencies within containers whose image content has changed, 
and it will rebuild those images, (using the Docker build command).  The framework will only 
rebuild those images that have changed.  The designer can force the rebuild of all images within
a lab by appending the ``-f'' switch to the end of the ``rebuild.py'' command.  That switch is not
intended for routine use because it wastes time and masks errors in our dependency logic.

If you build a new Labtainer exerciese, the container images will not be on the Docker Hub unless you put
them there.  If they are not on the hub, each student's computer will rebuild your lab when they go to run it.
While this is fully functional, the build time may distract from performance of the lab.  If you
create your own public repository on the Docker Hub (https://hub.docker.com/), you can populate that
with your lab(s) by setting the ``REGISTRY\_ID'' value in the start.config file for the lab(s). You
would then use the distrib/publish.py script to build, tag and push your lab to your registry.

\subsection{NPS Development Operations}
When building lab images at NPS, please set the LABTAINER\_NPS environment variable to "YES", e.g.,
\begin{verbatim}
   export LABTAINER_NPS=YES
\end{verbatim}
This will force packages to be retrieved from the local NPS mirrors (centosmirror.uc.nps.edu or
ubuntumirror.uc.nps.edu).  Refer to section \ref{package sources} for additional information.

Labs must be checked into the Subversion repository in order to be distributed.  After creating and testing
the lab, use the scripts/designer/bin/cleanlab4svn.py script to remove temporary files that do not belong in 
svn.  Use the publish.py script (described above) to publish the lab containers.
The distrib/mkdist.sh script is used by NPS to create the distribution tar file.  This script relies on
an internal NPS subversion repository as the source to the Labtainer scripts and labs.  Use the mk-devel-distrib.py script
to publish the developer configuration of the tar file.  The distrib/publish.py script is used to rebuild and 
publish all of the Labtainer exercieses managed by NPS.

The publish.py script will only rebuild labs that have changed.  After pushing a new lab container
image to the Docker hub, the script deletes the image from the local system.  The intent is to
ensure that future testing of the lab is done on the authoritative copy, i.e., from the hub.

Labtainer base images are built and published from the scripts/designer/bin directory.  Prior to publishing
baseline images, it is suggested that all local images be purged from the development machine, e,g.,
\begin{verbatim}
    /trunk/setup_scripts/destroy-docker.sh
\end{verbatim}
\noindent This will ensure the new baseline images to not incorporate layer remnants.

All new images should be first built and pushed onto the test registry, i.e., using the 
{\tt ./publish\_image.sh <image> -t}

\subsection{Alternate registry for testing}
If the environment variable {\tt TEST\_REGISTRY}, is set to YES, labs to be pulled and pushed
into an alternate registry defined in the trunk/config/labtainer.config file test\_registry entry entry.
It is easy to set up a registry (it is a container!), \url{https://docs.docker.com/registry/deploying/}

\subsection{Reuse of large file sets}
\label{manifest}
The use of ``sys\_tar'' described in \ref{large files} facilitates sharing of common
baselines of large or numerous files.  New labs can incorporate tar files from existing
labs through the use of ``external-manifest'' files, (see the xsite/victim/home\_tar as
an example).   The syntax of the external-manifest is shown below, and it may contain
multiple entries, one per line:
\begin{verbatim}
lab:container
\end{verbatim}
\noindent Where ``lab'' is the name of the lab, and ``container'' is the name of the container
whose tar file is to be included.

The framework will include content of tar archives referenced within these files
when creating an archive for the new lab.  This allows the sys\_tar to include lab-specific files
as well as files from other labs.  Designers should avoid adding duplicate tar files to the SVN repository.
This will avoid duplication of the files when a new distribution is created.

\subsection{Package Sources for apt and yum}
\label{package sources}
Labtainer base images include configuration files to use local NPS mirrors when creating derivative
images.  The original apt or yum sources are restored to an image if it is built without an environment
variable of {\tt LABTAINER\_NPS=YES}  The original sources are also restored when any container is first
run.  See the baseline Labtainer Dockerfiles in {\tt trunk/scripts/designer/base\_dockerfiles} to understand
how the sources files are manipulated.

The {\tt apt\_source} entry in the trunk/config/labtainer.config file will set the {\tt \$apt\_source}
environment variable in a Dockerfile, and this can be used by lab designers to force image builds to use
alternate sources.  By default, the value of the variable is ``archive.ubuntu.com''.  This hostname can be 
overridden via the trunk/config/labtainer.config file apt\_source entry, and having the following in your dockerfile:
\begin{verbatim}
RUN sed -i s/archive.ubuntu.com/$apt_source/ /etc/apt/sources.list
\end{verbatim}

\subsection{Locale Settings}
The locale settings, (e.g., used when interpreting character encodings) are set to en\_US.utf-8 
as can be seen in 
\begin{verbatim}
trunk/scripts/labdesigner/base_dockerfiles/Dockerfile.labtainer.base
\end{verbatim}
Similar Dockerfile entries in new or existing labs can provide alternate locale settings.

\section{Suggestions for Developers}
The result and goals configuration files can be revised and tested within a
running instructor container.  This saves time because you do not need to rebuild
the container for each iteration of the development of configuration files.  However,
be sure to scp the configuration files from the container to your host Linux system.

Most result and goal assessment can occur once you have generated a suitable sample of
expected student artifacts.  In other words, adding new goal does not typically require
that you go back and re-perform student actions.  Exceptions to this are:

Use {\tt TERMINAL\_GROUPS} in the start.config file to organize terminals if you have more
than a few.  Otherwise the student will spend time trying to find each terminal.

\begin{enumerate}
\item Adding new system commands to a ``treataslocal'' file;
\item Identifying new system files to be parsed as results.  For example, results in a log
file will not be collected unless that log file has been named in the results.config file.
\end{enumerate}

\section{Limitations}
The labtainers framework limits labs to the Linux execution environment.
However, a lab designer could prescribe the inclusion of a separate
VM, e.g., a Windows system, and that VM could be networked with the Linux
VM that hosts the Docker containers.  Future work would be necessary to include
artifacts from the Windows system within the framework's automated assessment
and parameterization.

Within the Ubuntu-based labs, the process tree of the initial Linux process will
not look like a typical Linux system \textit{init} process.  Within containers that have no
services, the initial process, i.e., process ID 1, will be a bash shell.
Containers having services and logging will have an initial process that is
the script that launches the services, e.g., the \textit{faux\_init} script.  However,
other process tree's will appear as they
do in Linux system, and this includes inetd services.

The CentOS-based labs have the true /usr/sbin/init process and systemd structure.

The available Docker network drivers do not permit IP address overlap between virtual networks.
For example, you cannot define two 192.168.1.0/24 LANs.

Inquisitive students will see evidence of artifact collection.  Home directories
on containers includes a \texttt{.local} directory that includes Labtainer scripts that manage
capturing and collection of artifacts, and that directory contains the stdin and
stdout files generated by student actions. Additionally, when the student starts a process
that will have stdin and stdout captured, the student will see extra processes within
that process tree, e.g., the \texttt{tee} function that generates copies of those data streams.
All of the containers share the Linux kernel with the Linux host.  Changes to
kernel configuration settings, e.g., enabling ASLR, will be visible across all
of the containers.

\section{Notes} \label{Notes}
\subsection{Firefox}
The labtainer.firefox image includes a /var/tmp/home.tar 
which is expanded into the user home directory when parameterize.sh is run.
This tar includes a profile in .mozilla that avoids firefox starting with its 
welcome pages and privacy statements.  The labtainer.firefox image includes a 
customized /usr/bin/firefox that starts the browswer in a new instance so it does share existing browsers. 
The {\tt about:config}
was altered to disabled insecure field warnings for the labs that do not use SSL connections to web servers.

\subsection{Wireshark}
Wireshark will not run as root in Labtainer containers.  
The wireshark installion in the labtainer.wireshark image is configured to not
require root to collect network packets:

When using the wireshark image, after the existing 
\begin{verbatim}
RUN adduser $user_name sudo
\end{verbatim}

\noindent add:

\begin{verbatim}
RUN adduser $user_name wireshark
\end{verbatim}

\subsection{Elgg}
The xsite/vuln-site/myelgg.sql file needs to be loaded for elgg to run.  First edit it to
change xsslabelgg.com to your site name (two changes).  Copy the sys\_tar/var/www/xsslabelgg.com/elgg
to your new lab.  Note the elgg/views/default/output files have been modified to permit cross site scripting.
\subsection{Host OS dependencies}
On rare occations, performance of a lab may depend on the host Linux OS.  An example is some
kernel tuning parameters viewed and set via sysctl are not visible within containers on eariler versions
of Ubuntu.  If your lab has such OS dependencies, you can check the OS and warn the student/instructor via a script
named "hostSystemCheck.py" placed withe the \_bin directory of any of the lab's containers.  This script shall
return the value '0' if dependencies are met, and '1' if dependencies are not met.  In the latter case, the 
startup.py will prompt the use to continue or abort.  Your script should explain the situation to the student.
An example of such a script is in the labs/tcpip/server/\_bin directory.
\subsection{Login Prompts}
See the centos-log lab for an example of a lab that prompts users to login to the virtual terminal.
In particular, you will need the {\tt bin/student\_startup.sh} script, and the {\tt \_system/sbin/login} program and the {\tt \_system/etc/login.defs} and {\tt securetty} files.

\subsection{Networking Notes}
\label{Networking Notes}
\subsubsection{Traffic mirroring}
Send copies of traffic from one ethernet port to another using the iptables TEE operation, e.g.,
\begin{verbatim}
    iptables -t mangle -A PREROUTING -i eth1 -j TEE --gateway 172.16.3.1
\end{verbatim}
\noindent will send copies of all incoming traffic on eth1 to the component with address 172.16.3.1.
Note that gateway must be a next hop, or you will have to configure the nexthop to forward it further.
This is useful for IDS labs, e.g., snort.  Mirroring all incoming traffic into a component will let you
reconstruct TCP sessions within that component.  Mirroring output from components is not always reliable.
Besides potential for duplicat traffic, Docker networks seem to sometimes gratuitously replace destination
addresses with those of the Docker network gateway, i.e., the gateway to the host.

\subsubsection{DNS}
Install bind9 in dockerfile.  Add zone files to /etc/bind and db files to ../\_system/var/cache/bind/.
Add reference to the /etc/bind/named.conf.local as seen in local-dns/\_bin/fixlocal.sh

\subsubsection{Overridding Docker routing and DNS}
Realistic network topologies require components to have /etc/resolv.conf and routing
table entries that do not depend on Docker gateways and related magic.  However, at some point you may
want components to be able to reach the outside world.  If you've fiddled resolv.conf and routing,
you likely broke the default Docker method for doing this.  One solution is to 
define an \textit{isp} component that has a default gateway and resolv.conf as Docker defines them.  Then
route all traffic and DNS queries to that (making use of dnsmasq and your own resolv.conf entries).  
Note you will also have to set up your own NAT on that ISP component.  See the dmz-example lab
ISP component .local/bin/fixlocal.sh as a worked example of a simple NAT setup.

As a worked example, the dmz-example lab components (other than the ISP), typically use the .local/bin/fixlocal.sh script to 
delete the Docker-generated route:
\begin{verbatim}
    sudo route del -host 172.17.0.1
\end{verbatim}
\noindent And the fixlocal.sh also replaces the resolv.conf entry with either a local DNS component, or a gateway running
the dnsmasq utility.  The /etc/rc.local script generally sets the default gateway, and configures iptables.

\subsection {Container isolation}
Docker provides namespace isolation between different containers, and
between the containers and the host platform.  Note however, that all
containers and the host share the same operating system kernel.  Some
kernel configuration changes will affect all containers and the host.  For example,
use of sysctl to modify Address Space Layout Randomization (ASLR) will effect
all containers and the effects will persist in the host after the containers
are stopped.  However, some tuning parameters such as net.ipv4.ip\_forward are
isolated, i.e., local to the container. These do get reset in ways that are
hard to predict, so it is suggested that sysctl tuning be done in rc.local
scripts so that they happen on each boot.

\end{document}
