\documentclass[12pt]{article}
\usepackage{geometry}
\geometry{a4paper, total={170mm,257mm},left=20mm, top=20mm,}
\usepackage[colorlinks=true,linkcolor=blue,urlcolor=black]{hyperref}
\usepackage{bookmark}
\begin{document}
\begin{titlepage}
\title {Labtainer Lab Designer User Guide}
\maketitle

\vspace{2.0in}
This document was created by United States Government employees at 
The Center for the Information Systems Studies and Research (CISR) at the Naval Postgraduate School NPS. 
Please note that within the United States, copyright protection is not available for any works created  
by United States Government employees, pursuant to Title 17 United States Code Section 105.   
This document is in the public domain and is not subject to copyright. 
\end{titlepage}
\tableofcontents
\newpage
\section {Introduction}
This manual is intended for use by lab designers wanting
to create or adapt cyber security labs to use the Docker-based lab framework known
as "Labtainers".

\subsection{Intended use}
The Labtainer framework is designed for use with computer and network security
laboratory exercises targeting Linux environments, and it is built around 
standard Linux Docker containers.  A Labtainer lab may include multiple 
networked components, all running locally on a student's computer.
\subsection {Benefits of Labtainers}

Deploying cyber security labs using this framework
provides three primary benefits:

\begin{enumerate}
\item The lab execution environment is controlled and consistent
across all student computers regardless of the Linux distribution
and configuration present on individual student computers.  
This allows each lab designer to control
which software packages are present, the versions of libraries and
specific configuration settings, e.g., /etc file values. These configurations
may vary between labs, and they may vary between multiple containers in
a single lab.

\item Assessment of student lab activity can be automated through a
set of configuration files that identify expected results, thus
relieving lab instructors from having to individually review detailed lab
results.

\item Labs may be automatically "parameterized" for each student such that
students cannot easily copy results from another student or from internet
repositories.  
\end{enumerate}

Labtainers provide the advantages of a consistent
execution environment without requiring
an individual Virtual Machine (VM) per lab, and without requiring all labs to be adapted for
a common Linux execution environment.   These benefits can be realized 
whether or not labs are configured for automatic assessment, 
or are parameterized for each student.

Exercises that include multiple networked computers illustrate another advantage 
of using containers over VMs, namely, containers require significantly less resources
than do VMs.  A student laptop that struggles to run two or more VMs can readily 
run multiple containers simultaneously. 

\section {Overview of the Student Environment and Workflow}
Labtainers support laboratory exercises designed for Linux environments,
ranging from interaction with individual programs to labs that include
what appear to be multiple components and networks.  Students see and interact with Linux
environments, primarily via bash shell commands. In general, the Labtainer
framework implementation is not visible to the student, and the Linux
environment as seen by the student is not augmented to support the framework.
The Linux execution environments presented by Docker containers do not readily
include GUIs.  As a result, Labtainer should not be used for exercises requiring
GUIs.

Labtainers are intended for use on individual student computers.
The computer utilized by a student must include the Linux operating system, e.g.,
as a single VM.  This Linux operating system, referred to herein
as the "host", can be any distribution and version
which supports Dockers, and it must have Dockers installed.  
In addition to installing Dockers on their Linux host, the student must
obtain and expand a tarball, which contains the Labtainer workspace utilities.
(This tarball may someday be replaced by standard Linux distribution packages,
e.g., Debian and/or RPM packages.)  Students initiate any and all labs from a
single workspace directory on the Linux host.

To perform a specific Labtainer exercise, the student runs a \textit{start.py} command from
the Labtainer workspace, naming the lab exercise.  This results in one or more
containers starting up along with corresponding virtual terminals via which the 
student will interact with the containers.  These virtual terminals typically
present a bash shell.  Each container appears to the student as a separate
computer, and these computers may appear to be connected via one or more networks.  

When a student starts a given exercise for the first time, the framework fetches
Docker images from the Docker registry.  In its current implementation, a single
base image (about 500MB) is retrieved from the Docker registry, and the remainder of the image
is built on the student's computer, i.e., by fetching packages. This is transparent to
the student, other than waiting for downloads and package installation. 

After the student performs the lab exercise, artifacts from the container
environments are automatically collected into a zip file that appears on
the student's Linux host.  The student forwards this zip file to the instructor,
e.g., via email or a Learning Management System (LMS).  The instructor collects student zip files into a common
directory on his or her own Linux host, and then issues a command that starts
the instructor container(s) for that lab.  This results in automated assessment of student lab
activity, (if the lab is designed for that), and creation of an environment
in which the instructor can review the work of each student.

Many cyber security lab exercises are assessed through use of reports in which students
describe their activities and answer specific questions posed by the instructor.  Labtainers
are intended to augment, rather than supplant this type of reporting.  The framework does not
prescribe mechanisms for managing such reporting.  If a given lab exercise was not designed
to support automated assessment, then zip files from the students need not be collected.  On
the other hand, instructors may still wish to collect the zip files to manually review selected
student activity, e.g., student-developed programs, output files, or even bash history records.  

\section {Obtaining the Labtainer Development Kit}
Installation of Labtainers is described in in the Labtainer Student Guide,  
which also includes instructions for installing an Ubuntu VM (if you do not already have a Linux system),
and the Docker package.  Note Labtainers will work with any Linux
distribution that supports Dockers.  If you already have Dockers installed on a Linux system, 
reference the Student Guide for other dependencies. 
Future releases may include a separate development kit, but for now there is a single Labtainers distribution
that contains the entire framework.


\section {Defining New Labs}
\label{sec:new_labs}
The most challenging and critical part of designing a new cyber security lab
is the design of the lab itself, i.e., identifying learning objectives and
organizing exercises to achieve those objectives.  The Labtainer framework
does not specifically address any of that.  Rather, the framework is intended
to allow you to focus more time on the design of the lab and less time on mitigating and
explaining system administration burdens you are placing on students and instructors.
The framework does not require lab designers to program or create scripts.  The
lab designer primarily interacts with the framework by editing configuration files
which affect the student's execution environment and the optional automated
assessment of student activity.

The examples in this guide assume you have defined the LABTAINER\_DIR environment variable.
Set the LABTAINER\_DIR environment variable to the top of the svn repo, e.g.,
\begin{verbatim}
    export LABTAINER_DIR=/home/mike/labtainer/trunk
\end{verbatim}
It is suggested that you add the LABTAINER\_DIR definition in your .bashrc.

Labtainer exercises each have their own
directory under the "labs" directory in the project repository.
The first step in creating a new lab within the framework is to create
a directory for the lab and then cd to it.  The directory name will be the named
used by students when starting the lab.  It must be all lower case and not contain spaces.
\begin{verbatim}
    cd $LABTAINER_DIR/labs
    mkdir <new lab name>
    cd <new lab name>
\end{verbatim}

After the new lab directory is created, run the "new\_lab\_setup.sh" script.
\begin{verbatim}
    $LABTAINER_DIR/scripts/designer/bin/new_lab_setup.py
\end{verbatim}
This will create a set of template files that you can then customize
for the new lab.  These template files are referenced in the discussion
The result is a new labtainer lab that can be run.  While this new
lab will initially only present you with a bash shell to an
empty directory, it is worth testing the lab to understand the workflow.

\subsection{Testing the new lab}
Once a new lab directory is created, and the new\_lab\_setup.py has been run, then 
you can test the new, (currently empty) lab.  All student labs are launched from the
labtainer-student directory.  Lab development workflow is easiest if at least two
terminals or tabs are used, one in the new lab directory, and one in the labtainer-student
directory.  So, open a new tab or window and set the LABTAINER\_DIR environment variable
if you had not added it to your .bashrc.  Then:

\begin{verbatim}
    cd $LABTAINER_DIR/scripts/labtainer-student
\end{verbatim}
Then start the container using the:

\begin{verbatim}
    ./rebuild.py [labname] 
\end{verbatim}
command, where labname is the name of the lab you just created.  
The very first time you run this, it may take a bit of time because it fetches the 
base Labtainer Docker image from the Docker registry.  Subsequent builds
should be faster because your local Docker system will cache portions of the build.  

The rebuild.py command will remove and recreate the container
each time the script is run.  And it will rebuild the container image if any of its configuration 
information has changed.\footnote{The build process may generate warnings in red text, some of which are expected.  
These include an unreferenced "user" variable and the lack of apt-utils if apt-get is used to install packages in 
Dockerfiles.}  This is often necessary when building and testing new labs, to ensure the
new environment does not contain artifacts from previous runs.

Note the "rebuild.py" command is not intended for use by students, they would use the "start.py" command.  

Stop the containers with 
\begin{verbatim}
    ./stop.py [labname]
\end{verbatim}
Note that when you stop the container, a path to saved results is displayed.
This is the zip file that the student will forward to the instructor.

To test adding a "hello world" program to the new labtainer, perform the following steps:
\begin{itemize}
\item From the new lab directory window, cd \verb!$LABTAINER_DIR/labs/[labname]/[labname]!
\item Create a "hello world" program, e.g., in python or compiled C.
\item From the labtainer-student window, run rebuild.py [labname]
\end{itemize}
    
You should see the new program in the container's
home directory.  If you run the program from the container, and then stop the container
with stop.py, you will see the stdin and stdout results of the program within the
saved zip file.

Note how the "hello world" program was placed in \verb!$LABTAINER_DIR/labs/[labname]/[labname]!.
The seemingly redundant "labname" directories are a naming convention in which the
second directory names one of potentially many containers.  In this simple example,
the lab has but one container, whose name defaults to the lab name.

The following sections describe how to futher alter the lab execution environment seen by 
the student.

\section {Defining the lab execution environment}
A given lab typically requires some set of software packages, and some
system configuration, e.g., network settings.  
The framework captures most configuration details within a standard
Dockerfile.  A default Labtainer-specific Dockerfiles is placed in the new lab's "dockerfiles" 
directory when the new lab is created.  We use standard Docker file syntax, which is described at 
\url{https://docs.docker.com/engine/reference/builder/}

Lab designers should reference that Docker documentation for the 
syntax and semantics of these files.
Simple labs should be able to use the default Dockerfile copied by the 
new\_lab\_setup.py script.  That Dockerfile refers to a base Labtainer
image that contains the minimum set of Linux packages necessary to 
host a lab within the framework.  The default
execution environment builds off of a recent Ubuntu image.

A given lab can include multiple containers, each appearing as distinct
computers connected via networks.  The execution environment seen by a
student when interacting with one of these "computers" is therefore defined
by the configuration of the associated container.  

\subsection {Container Isolation}
Docker provides namespace isolation between different containers, and
between the containers and the host platform.  Note however, that all
containers and the host share the same operating system kernel.  Kernel
configuration changes will affect all containers and the host.  For example,
use of sysctl to modify Address Space Layout Randomization (ASLR) will effect
all containers and the effects will persist in the host after the containers
are stopped.

\subsection {Naming Containers}
If a lab is to include only one container, you can
skip ahead to the subsection titled Lab-specific files in the student's home directory.

You must assign a name to each container within the new lab.  Each new lab
starts with a single container, whose name matches the lab name.  You are free
to change that name. The names of containers should reflect their role in the lab,
e.g., "client" and "server".  Once you have picked container names, you must create
Dockerfiles and update the 
\begin{verbatim}
   $LABTAINER_DIR/labs/[labname]/config/start.config file.

\end{verbatim}
\noindent Each container must have its own Dockerfile within the 
\begin{verbatim}
   $LABTAINER_DIR/labs/[labname]/dockerfiles
\end{verbatim}
\noindent directory.  The naming convention for dockerfiles is
\begin{verbatim}
    Dockerfile.[labname].[container_name].[role]
\end{verbatim}
where role is either "student" or "instructor".  The framework will use the student Dockerfile to
create the instructor container unless a distinct instructor file is present (e.g., if the designer
wishes the instructor to have unique software packages.) The system automatically creates a student Dockerfile
when a new lab is created.  You are responsible for creating (copying to) additional Dockerfiles for other containers,
and changing the name of the initial Dockerfile if its container name changes.

The first line of each Dockerfile identifies the base image to be pulled from the Docker Hub.
The initial default image is a basic Ubuntu system with a minimal set of packages.  To use an
alternate image having additional networking packges (e.g., tcpdump, xinetd, sshd), change the first line to:
\begin{verbatim}
FROM mfthomps/labtainer.network
\end{verbatim}

Next, you must also describe your containers within the \textit{start.config} file as described below.

\subsection{Container definitions in start.config}
\label{start.config}
Most single container labs can use the automatically generated start.config file file
without modification.  Labs consisting of multiple containers must modify the start.config 
file.  The following describes the major sections of that configuration file.
\begin{itemize}
\item GLOBAL\_SETTINGS -- Beneath this keyword, the following values must be defined:

\begin{itemize}
\item GRADE\_CONTAINER [container name] -- All lab containers are available to the instructor while assessing student labs.
This setting identifies which of the lab containers will host automated grading functions.
\item HOST\_HOME\_XFER [dir name] --  Identifies the host directory via which to transfer student artifacts, relative to 
the home directory.  For students, this is where the zip files of their results end up.  For instructors, this is
where zip files should be gathered for assessment.
\item LAB\_MASTER\_SEED [seed] -- The master seed string for this lab.  It is combined with the student email
address to create an instance seed that controls parameterization of individual student labs.
\item REGISTRY [registry] -- The id of the Docker Hub registry that is to contain the lab images.
See \ref{publishing} for details on the use of this keyword.
\item COLLECT\_DOCS [yes/no] -- Optional directive to collect lab/docs content as part of student artifacts.
These are then available to the instructor in the labtainer\_xfer/[lab]/docs directory.  Also see \ref{instructions}.
\end{itemize}

\item NETWORK [network name] -- One of these sections is require for each network within the lab.  In addition to
providing a name for the network, the following values are defined:

\begin{itemize}
\item MASK [network address mask] -- The network mask, e.g., 172.25.0.0./24
\item GATEWAY [gateway address] -- The IP address of the network gateway
\end{itemize}

\item CONTAINER [container name] -- One of these sections is required for each container in the lab.
In addition to naming the container, the following values are defined: 

\begin{itemize}
\item TERMINALS [quantity] -- The number of virtual terminals to open and attach to this 
container when a lab starts.  If missing, it defaults to 1. Terminal titles are set to the 
bash shell prompt.
\item XTERM [title] [script] -- The named script is execute in a virtual terminal with the
give title.   The system will change to the user's home directory prior to executing the
script.  The script should be placed in container \_bin directory, i.e.,
\begin{verbatim}
    $LABTAINER_DIR/labs/[labname]/[container]/_bin
\end{verbatim}
\noindent If the title is "INSTRUCTIONS", no script is necessary and the instructions.txt file
in the container home directory will be displayed.
\item USER [user name] -- The user name whose account will be accessed via the virtual terminals. If
the user is not "ubuntu", then the Dockerfile must include an entry to add the user to the container.
(See the nmap-ssh lab mycomputer dockerfile as an example).
\item network name [ip address] -- Network address assignments for each network (defined via a NETWORK section), 
that is to be connected to this container.  A separate line should be entered for each network.
\end{itemize}
\end{itemize}
  
A simple example of a two-container lab can be found in 
\begin{verbatim}
    $LABTAINER_DIR/labs/telnetlab
\end{verbatim}


\subsection {Lab-specific files in the student's home directory}
Files that are to reside in the student's \$HOME directory are placed in the 
new lab container directory.  For example, if a lab includes a source code file, that
should be placed in the lab container directory, as a result, the file will appear in the student's
home directory within the container when the container starts.  The lab container
directory is at:  

\begin{verbatim}
    $LABTAINER_DIR/labs/[labname]/[container name]
\end{verbatim}
Note the name of the container name in labs with a single container matches the labname by default.

All files and directories in the lab container directory will be copied to the student's HOME
directory except for the \_bin and \_system directories.
Each initial Dockerfile from the templates include this line:
\begin{verbatim}
    ADD $labdir/$lab.tar.gz $HOME
\end{verbatim}
to accomplish the copying. Except as noted below, Dockerfile should not include any other ADD commands
to copy files to the HOME directory.
\subsubsection{Large or numerous files in the home directory} \label{large files}
If there are large sized, or a high quantity of files that are to be placed relative to a 
container home directory, those should be placed into a "home\_tar" directory at:
\begin{verbatim}
    $LABTAINER_DIR/labs/[labname]/[container_name]/home_tar/
\end{verbatim}
\noindent Use of this technique prevents these files from being collected as student artifacts, which
otherwise include copies of everything relative to the home directory.  This
can save considerable time and space, e.g., on the instructor's computer that must collect
all student artifacts.
The individual files should exist in the home\_tar directory, and the framework automatically
creates the tar file for transfer to the Docker image, (and will do so if an existing tar file
is older than any file in the directory).


\subsection{Lab-specific system files}
All files in the
\begin{verbatim}
    $LABTAINER_DIR/labs/[labname]/[container name]/_system
\end{verbatim}
directory will be copied to their corresponding paths relative to the root directory.
For example, configuration files for /etc should appear in \_system/etc/.

The initial Dockerfile from the templates include this line:
\begin{verbatim}
    ADD \$labdir/sys_\$lab.tar.gz /
\end{verbatim}
\noindent to accomplish the copying. 
If a lab contains a large quantity of system files, or large files, those
can be placed into the directory named:
\begin{verbatim}
    $LABTAINER_DIR/labs/[labname]/[container name]/sys_tar
\end{verbatim}
either as individual files, or in a "sys.tar" archive.  In the former case,
the framework will automatically create the sys.tar file.  This technique 
can save time in building lab images becauase the files do not need to be 
archived for each build.  

In general, files modified and maintained by the designer should go into the
\_system directory while static system files should go into the sys\_tar directory.

\subsection {System services}
The Dockerfile "ENTRYPOINT" command can be used to start a system service.  The general Docker 
model is that a single Docker container runs a single service, with logging being forwarded to 
the host.  Labtainers disregards this model because our goal is to make a container look more like a Linux
system rather than a conformant Docker container.  The default Labtainer Dockerfiles include an
ENTRYPOINT command that launches a \textit{faux\_init} script that starts rsyslog, (so that system logs
appear in /var/log), and runs rc.local.  The network configuration of the baseline Dockerfile also starts xinetd,
which will then fork services, e.g., the sshd, per the /etc/xinet.d/ configuration files.  The faux\_init script
can be extended (or replaced) as needed by the lab.  The example telnetlab illustrates the use of this script
to provide network services.

\subsection {Instructions for Students} \label{instructions}
Lab instructions for students can be displayed in a virtual terminal by placing an
"instructions.txt" file within the home directory of one of the containers.  Refer to existing
labs for conventions.  Additionally, textual message can be displayed to the student before any 
of the lab virtual terminals are created.  Any text within the 
\begin{verbatim}
   $LABTAINER_DIR/labs/[labname]/docs/read_first.txt
\end{verbatim}
\noindent file will be displayed on the Linux host in the terminal in which the student
starts the lab.  Any "LAB\_MANUAL" string in that file will be replaced with the full path
to a [labname].pdf file within that same docs directory.  And "LAB\_DOCS" is replaced by the
path to the lab docs directory.  One intended use is to prompt the
student to open a PDF lab manual and perhaps read parts of it prior to continuing with the lab.
Another intended use is to display the path to a reporting template that a student is to use
for answering lab-specific questions and note taking.  If the name of the symbols are prefaced
by "file://", then the paths will display as links that can be opened via a right click.
If the start.config file includes "COLLECT\_DOCS YES", the content of the lab/docs directory will be
included with the student artifacts and available extracted into the intstrutor's 
labtainer\_xfer/[lab]/docs directory.

\subsection {Running programs in Virtual Terminals}
The "XTERM" section in the 
\begin{verbatim}
    $LABTAINER_DIR/labs/[labname]/[container name]/_bin/
\end{verbatim}

\subsection{Final lab environment fixup}
The initial environment encountered by the student is further refined using
the optional \_bin/fixlocal.sh script.  The framework executes
this script the first time a student starts the lab container.  For example,
this could be used to compile lab-specific programs afer they have been parameterized,
(as described below).  Or this script could perform final configuration adjustments
that cannot be easily performed by the Dockerfile.  These scripts are per-container
and reside at:
\begin{verbatim}
    $LABTAINER_DIR/labs/[labname]/[container name]/_bin
\end{verbatim}

\subsection{Automatic copying files from containers to the host}
Some labs require the use of tools that are not available within containers. For example,
Wireshark cannot run in a container because it requires a GUI.  In such cases, the
student can be directed to run the tool from the host, however Labtainers does not support
automated assessment of the results.  However, the files accessed by such tools can be
parameterized by Labtainers, and then copied to the host.  Such files are identified within
\begin{verbatim}
    $LABTAINER_DIR/labs/[labname]/config/files_to_host.config
\end{verbatim}
\noindent with a format of "container:filename".  Any named files within the home directory of
the named container will be copied to the host computer into a directory named by the lab, relative
to the Labtainer working directory.

\section{Parameterizing a lab}
This section describes how to individualize the lab for each student to discourage
sharing of lab solutions.  This is achieved by defining symbols within source 
code or/and data.  The framework will replace these symbols with randomized values
specific to each student.  The config/parameter.config file identifies the files, and
the symbols within those files that are to be modified.  A simple example can be found in 
\begin{verbatim}
    $LABTAINER_DIR/labs/formatstring/formatstring/config/parameter.config
\end{verbatim}

That configuration file causes the string "SECRET2\_VALUE" within the file:
\begin{verbatim}
     /home/ubuntu/vul_prog.c
\end{verbatim}
to be replaced with a hexidecimal representation of a random value
between 0x41 and 0x5a, inclusive.

This symbolic replacement occurs when the student first starts the lab container,
but before the execution of the \_bin/fixlocal.sh script.  Thus, in the formatstring
lab, the executable program resulting from the fixlocal.sh script will be specific
to each student (though not necessarily unique).

Symbolic parameter replacement operations are defined within the config/parameter.config file.
Each line of that file must start with a \verb!"<parameter_id> : "!, which is any unique string, and
is followed by one of the following operations:

\begin{verbatim}
 RAND_REPLACE : <filename> : <symbol> : <LowerBound> : <UpperBound>
   Replace a symbol within the named file with a random value within a given
   range.  The random value generator is initialized with the lab instance
   seed.

     where: <filename> - the file name (file must exist) where <symbol> is 
                         to be replaced.  The file name is prefixed 
                         with a container name and a ":", (the container 
                         name is optional for single-container labs).  
                         This may be a list of files, delimited by semicolons. 
           <symbol> - the string to be replaced
           <LowerBound> and <UpperBound> specifies the lower and upper bound
                                       to be used by random generator
   example:

     some_parameter_id : RAND_REPLACE: client:/home/ubuntu/stack.c 
         : BUFFER_SIZE : 200 : 2000
     (all one line) will randomly replace the token string "BUFFER_SIZE" found in
     file stack.c on the mylab.client.student container with a number ranging from 
     200 to 2000
 
 HASH_CREATE : <filename> : <string>
   Create or overwrite a file with a hash of a given string and the lab 
   instance seed.
     where: <filename> - the file name that is to contain the resulting hash.
                         The file name is prefixed with a container name 
                         and a ":", (the container name is optional for 
                         single-container labs).  
                         This may be a list of files, delimited by semicolons 
                         The file name is is optionall, (in cases of a single
                         container).  This may be a 
                         list of files, delimited by semicolons.
            <string> -   the input to a MD5 hash operation (after concatenation 
                         with the lab instance seed)
                       
                   
   example:
     some_parameter_id : HASH_CREATE : client:/home/ubuntu/myseed 
        : bufferoverflowinstance
   A file named /home/ubuntu/myseed will be created (if it does not exist), 
   containing an MD5 hash of the lab instance seed concatentated with the 
   string 'bufferoverflowinstance'.
 
 HASH_REPLACE : <filename> : <symbol> : <string>
   Replace a symbol in a named file with a MD5 hash of a given string 
   concatenated with the lab instance seed.
     where: <filename> - the file name (file must exist) where <symbol> is 
                         to be replaced.  The file name is prefixed 
                         with a container name and a ":", (the container 
                         name is optional for single-container labs).  
                         This may be a list of files, delimited by semicolons. 
            <symbol> - a string that will be replaced by the hash
            <string> - a string contatenated with the lab instance seed and hashed

     example:
       some_parameter_id HASH_REPLACE : client:/root/.secret : 
           ROOT_SECRET : myrootfile
     The string "ROOT_SECRET" in file /root/.secret will be replaced with an MD5 hash
     of the concatenation of the lab instance seed and "myrootfile".
\end{verbatim}

The parameter\_id fields may be referenced during the automated grading function, described below. 


\section{Automated assessment of student labs}
This section describes how to configure a lab for automated assessment of student work.
Note the framework does not require that labs include automated assessment, e.g., the
"results" of a lab may consist entirely of a written report submitted by the student.
Support for automated collection of written reports is described in \ref{instructions}
and the use of COLLECT\_DOCS in the start.config file.

The goal of automated assessment is to provide instructors with some confidence that 
students performed the lab, and to give instructors insight into which parts
of a lab students may be having difficulty with.  The automated assessment functions are
not intended to standardize each student's approach to a lab, rather the goal is to permit
ad-hock exploration by students.  Therefore, lab designer should consider ways to identify
evidence that steps of a lab were performed rather than trying to identify everything a student
may have done in the course of the lab.

The framework's automated assessment functions generally assume the student will interact with one or
more programs or system utilities.  Each time the student invokes a selected program or utility, the 
framework captures copies of standard input and standard output, (stdin and stdout) into timestamped file sets.
This is transparent to the student.  (Also see the discussion of the "treataslocal" file for capturing
program output other than stdout.)  These timestamped file sets, and everything else relative to
the student's home directory, are automatically packaged when the student completes the lab.
These packages of artifacts are then transferred to the instructor, (e.g., via email or a CLE), and 
ingested into the instructor's system where lab assessment occurs.

The stdin and stdout for all non-system programs is captured, e.g., the results of an "ls" command
are not captured.  The stdin and stdout of system programs will be captured if the program
names appear at the beginning of a line in the \textit{treataslocal} file at
\begin{verbatim}
    $LABTAINER_DIR/labs/[labname]/[container name]/_bin/treataslocal
\end{verbatim}
\noindent System utilties should include their full path, e.g., "/usr/bin/nmap".

The treataslocal file program identifiers can include optional output file identifiers that
cause timestamped copies of specified files to be made whenever the program is invoked.  
If program file output from local programs is to be captured in timestamp files (in addition
to the stdout and stdin), simply include those program names in the treataslocal file.
These identifiers are of the form:
\begin{verbatim}
    delim_type:delim_value
        where delim_type is one of:
            starts -- the output file name is derived from the 
            substring following the given delim_value within the 
            command line typed by the student.  For example, 
            "dd starts:of=" for a "dd in=myfs.img of=newfile" 
            would yield an output file name of "newfile".
     
            follows -- the output file name is the command line 
            token following the given delim_value.  For example, 
            "myprogram follows:myprogram" for a command line of
            "myprogram outfile" would yield "outfile" as the output 
             file name.

            file -- the delim_value is the output file name
\end{verbatim}
\noindent The resulting timestamped files are located with the stdin and stdout files in .local/result

Non-system programs can be excluded from stdin/stdout capturing by including their names in
a "ignorelocal" file in that same directory.

\{section{Identify Lab-specifc Artifacts}
The automated assessment functions encourage labs to be organized into a set of distinct "goals".
For each goal, the lab designer should identify specific fields within stdin and/or stdout that
could be compared to "expected" values.  These lab-specific artifacts are identified within the
"instr\_config/results.config file".  Artifacts are identified in terms of:
\begin{enumerate}
\item the program that was invoked;
\item whether the artifact is in stdin or stdout
\item the line containing the artifact, and a token within that line.
\item ad-hoc properties, such as the quantity of lines in the stdin file.
\end{enumerate}

Each identified artifact is given a symbolic name. A named artifact is referred to herein as a \textit{result}, which 
is then referenced in the goals.config file to assess whether it is an expected value.

Consider the labs/formatstring/instr\_config/results.config file.  The first non-comment line
defines a result having the symbolic name "crashStringCanary".  This result is
found by looking at stdout from the "vul\_prog" program, finding the first line that starts with:
"*** stack smashing detected".  The result is assigned the value of the third space-delimited 
token in that line.

Entries within the results.config file each have the following format:

\begin{verbatim}
  <nametag> = <file_id> : <field_type> : <field_id> [: <line_type> : <line_id>]
      where:
            nametag -- The symbolic name of the result, which will be referenced in 
                       the goals configuration file.  It must be alphanumeric, 
                       underscores permitted.
            file_id -- Identifies a single file, or the set of files to be parsed.  
                       The format of this id is:
                [container_name:]<prog>.[stdin | stdout] | file_path
                   where <prog> is a program or utility name whose stdin and stdout 
                   artifacts will include timestamps.  The optional container_name 
                   identifies the container hosting the file.  Labs with a single 
                   container can omit this qualifier.  Alternately, an explicit 
                   file_path is intended for log files of services that persist 
                   across multiple student operations.  If the given path is not 
                   absolute, it is relative to the container user's home directory. 
                   The wildcard character '*' can be used in place of <prog>,
                   i.e., *.stdin is for all stdin artifacts and *.stdout is for all 
                   stdout artifacts.
            field_type - Optional, defaults to "TOKEN", possible values include:
               TOKEN      -- Treat the line as space-delimited tokens
               PARENS     -- The desired value is contained in parenthesis
               QUOTES     -- The desired value is contained in parenthesis
               SLASH      -- The desired value is contained within slashes, 
                             e.g., /foo/
               LINE_COUNT -- The quantity of lines in the file. Remaining fields 
                             are ignored.
               CONTAINS   -- The value of nametag will be set to true if the file 
                             contains the string represented in field_id.
               STRING_COUNT--The value of nametag will be set to the quantity of
                             occurances of the string represented in field_id.
               PARAM      -- The value of the nametag will be set to nth parameter
                             (0 is the program name), provided in the 
                             program invocation.  
                             
               SEARCH     -- The value nametag will be set to the search resulting
                             from the given field_id, which is treated as an
                             expression having the syntax of pythons parse.search 
                             function.  E.g., "frame.number=={:d}" would 
                             yield the frame number.
                              
            field_id -- An integer identifying the nth occurance of the field type.
                        Alternately may be "LAST" for the last occurance of the 
                        field type, or "ALL" for the entire line (which causes the 
                        field type to be ignored).  Or if field_type is SEARCH, the
                        field_id is treated as the search expression. 
                        If field_type is "CONTAINS", the 
                        remainder of the line is treated as a string to be 
                        searched for.  If field_type is "PARAM", the field_id is
                        the 1-based index of the parameter whose value is to be 
                        assigned, and no other fields should be present.
            line_type - Identifies how the line is to be identified, values include:
                LINE           --  The line_id will be an integer line number 
                                   (starting at one).  Use of this to identify 
                                   lines is discouraged since minor lab changes 
                                   might alter the count.
                STARTSWITH     --  the line_id will be a string.  This names the 
                                   first occurrence of a line that starts with 
                                   this string. 
                HAVESTRING      --  The line_id will be a string.  This names the 
                                   first occurrence of a line that contains the 
                                   string.
                NEXT_STARTSWITH -- the line_id will be a string.  This names the 
                                   line preceeding the first occurrence of a line 
                                   that starts with this string. 
            line_id - See line_type above.
\end{verbatim}

\subsection{Capturing information about the environment}
Some labs require the student to alter system configuration settings,
e.g., using the sysctl command to effect ASLR. A \textit{checklocal.sh} script in:
\begin{verbatim}
    $LABTAINER_DIR/labs/[labname]/[container name]/_bin
\end{verbatim}
is intended to contain whatever commands are necessary to record the 
state of the system at the time a program was invoked.  The stdout of
the checklocal.sh script is recorded at the beginning of the stdout artifact. 

\subsection{Assessing the student results}
Results of student lab activity are assigned symbolic names by the results.config file
as described above.  These results are then referenced in the goals.config to assess whether
the student obtained expected results.  Each lab goal defined in the goals.config file
will evaluate to TRUE or FALSE, with TRUE reflecting that the student met the defined goal.
Once evaluated, a goal may determine the state of subsequent goals within the goals.config file, 
i.e., through use of boolean expressions and temporal comparisons between goals.  The evaluated
state of each goal can then contribute to a student grade.

As noted earlier, student results may derive from multiple invocations of the same program or system utility.  
The framework does not discourage students from continuing to experiment and explore aspects of the 
exercise subsequent to obtaining the desired results.  In general, the assessment determines if the student
obtained expected results during any invocation of a program or system utility.  In those cases 
where the student is required to obtain the expected results during the last invocation of a program, 
the \textit{matchlast} goal type may be specified as described below.

To facilitate grading multiple attempts or explorations of a lab exercise, the framework associates
timestamps with the results of processing the results.config file.  A single timestamp will
include results from a stdin file and a stdout file.  In general, there will be a distinct, 
timestamped set of results for each occurrence of a student invoking a targeted program.

The following syntax defines each goal or subgoal within the goals.config file.  While the syntax
may appear complex, most goals can be expressed simply as can be seen in the labtainer examples distributed
with the framework.

\begin{verbatim}
<goal_id> = <type> : [<operator> : <resulttag> : <answertag> | <boolean_expression> 
            | goal1 : goal2 | <resulttag> | value : subgoal_list]
  Where: 
    <goal_id> - An identifer for the goal. It must be alphanumeric 
                (underscores permitted).
    <type> - must be one of the following:
         'matchany'    - Results from all timestamped sets are evaluated.
                         If the answertag names a result, then both that 
                         result and the resulttag must occur in the same 
                         timestamped set.  The 'matchany' goals are treated 
                         as a set of values, each timestamped based on the 
                         timestamp of the reference resulttag.
                      
         'matchlast'   - only results from the latest timestamped set are 
                         evaluated.
         'matchacross' - The resulttag and answertag name results.  The 
                         operator is applied against values in different 
                         timestamped sets.  For example, a "string_diff" 
                         operator would require the named results to have 
                         at least two distinct values in different 
                         timestamped sets.  Note: 'matchacross' cannot 
                         be used within the boolean expression defined below.
         'is_true'     - The goal is true if the value of the resulttag is 
                         true, e.g., from a "CONTAINS" field type in the 
                         results.config
         'is_false'    - The goal is true if the value of the resulttag is 
                         not true, e.g., from a "CONTAINS" field type in the 
                         results.config
         'boolean'     - The goal value is computed from a boolean expression 
                         consisting of goal_id's and boolean operators, ("and", 
                         "or", "and_not", "or_not", and "not"), and parenthisis 
                         for precedence.  The goal_id's must be from goals defined 
                         earlier in the goals.config file.  The goal evalutes to 
                         TRUE if the boolen expression evaluates to TRUE for any
                         of the timestamped sets of goal_ids, (see the 'matchany' 
                         discussion above).  The goal_id's cannot include any 
                         "matchacross" goals.
         'count_greater' The goal is true if the count of true subgoals in the 
                         list exceeds the given value.  The subgoals are 
                         summed across all timestamps.  The subgoal list is 
                         comma-separated within parenthesis.
         'time_before' - Both goal1 and goal2 must be goal_ids from previous 
                         *matchany* goal types.  Evaluates to TRUE if any TRUE 
                         goal1 has a timestamp that is before than any TRUE goal2
         'time_during' - Both goal1 and goal2 must be goal_ids from previous 
                         *matchany* goal types.  Timestamps include a start and 
                         end time, reflecting when the program starts and when it 
                         terminates.  Evaluates to TRUE if any TRUE goal1 has a 
                         start timestamp within the start and end times of any 
                         TRUE goal2.
         'execute'     - The <operator> is treated as a file name of a script to 
                         execute, with the resulttag and answertag passed to the 
                         script as arguments.  The resulttag is expected to be 
                         one of the symbolic names defined in the results.config 
                         file, while the answertag is expected to be a literal 
                         value or the symbolic name in the parameters.config file 
                         Note: the answertag cannot be a symbolic name from 
                         results.config 
         'count'       - If the remainder of the line only includes a resulttag,
                         then the goal value is assigned the quanity of occurances of 
                         the given resulttag.  Otherwise the goal value is assigned the
                         quantity of occurances that satisfy the given operator and
                         arguments.
         'value'       - The goal value is assigned the given resulttag value from
                         the most recent timestamped file that satisfies the resulttag.
 
<operator> - the following operators evaluate to TRUE as described below:
   'string_equal' -  The strings derived from <answertag> and <resulttag>
                     are equal.
   'string_diff' -   The strings derived from <answertag> and <resulttag>
                     are not equal.
   'string_start' -  The string derived from <answertag> is at the start of 
                     the string derived from <resulttag>.
           example:  answertag value = 'MySecret'
                     resulttag value = 'MySecretSauceIsSriracha'
   'string_end' -    The string derived from <answertag> is at the end of
                     the string derived from <resulttag>.
           example:  answertag value = 'Sriracha'
                     resulttag value = 'EatMoreFoodWithSriracha'
   'integer_equal' - Integers derived from <answertag> and <resulttag>
                     are equal.
   'integer_greater' - The integer derived from <answertag> is greater than
                       that derived from <resulttag>.
   'integer_lessthan'- The integer derived from <answertag> is less than
                       that derived from <resulttag>
   <executable_file> - If the type is 'execute' then <operator> is a filename of 
                       an executable.
             
<resulttag>  -- One of the symbolic names defined in the results.config file.
                The value is interpreted as either a string or an integer,
                depending on the operator as defined above.  Alternately, 
                for integer operators within matchany types,  this
                may be an arithmetic expression within parentheses.  For example,
                "(frame_number-44)".
                
<answertag>  -- Either a literal value (string, integer or hexidecimal), or a 
                symolic name defined in the results.confg file or the 
                parameters.config file:
 
                answer=<literal>    -- literal string, integer or hex value 
                                       (leading with 0x), interpretation depending 
                                       on the operator as described above.
                result.<symbol>     -- symbol from the results.config file
                parameter.<symbol>  -- symbol from the parameters.config file
                parameter_ascii.<symbol> -- same as above, but the value parsed as 
                                       an integer or hexidecimal and converted to an 
                                       ascii character.

\end{verbatim}
         Note that values derived from the parameters.config file are assigned the same values as
         were assigned when the lab was parameterized for the student.

\subsection{Simple Parameterization for Checking Own-work}
The simplest, though by no means robust, strategy for ensuring students
have turned in their own work, (vice getting a zip file from a friend and simply
changing the name of the file), is to individualize some file on one of the containers,
and then check that file and the archive file names during grading.  The framework does
this automatically and reports on any student archive that does not seem to have
originated from a Labtainer initiated with that student's email address"

\section{Networking}
Most networking is simply a matter of defining networks and assigning them to containers
as described in \ref{start.config}.  

In addition to networks properties defined in the
start.config file, each container \texttt{/etc/host} file includes a "my\_host entry" that names
the host Linux.  By Docker default, each container includes a default gateway that
leads to the Linux host.  This allows students to scp files to/from the container and host.
It also allows the student to reach external networks, e.g., to fetch additional packages in
support of student exploration.

In some instances, the lab designer will want to define a different default route for a 
container.  Each container derived from the Labtainer network base includes a \textit{set\_default\_gw.sh}
script that can be added to the \textit{/etc/rc.local} file to redefine the default gateway.
This script will automatically retain a route table entry so that the student can reach the "my\_host"
address.  Additionally, those baseline images include a \textit{togglegw.sh} script that the student
can use to toggle the default gateway between one that leads to the host, and one defined for the lab.
This allows students to add packages on components having lab-specific default gateways.

\section{Building, Maintaining and Publishing Labs} \label{publishing}
Typically, when a Labtainer is started, the container's associated Docker images are pulled from
the Docker Hub if they are not already local on the Linux host.  When building and editing labs,
the designer desires to run images reflecting recent changes that have been made.  The framework
includes logic to identify dependencies within containers whose image content has changed, 
and it will rebuild those images, (using the Docker build command).  The framework will only 
rebuld those images that have changed.  The designer can force the rebuild of all images within
a lab by appending the "-f" switch to the end of the "rebuild.py" command.

If you build a new Labtainer lab, the container images will not be on the Docker Hub unless you put
them there.  If they are not on the hub, each student's computer will rebuild your lab when they go to run it.
While this is fully functional, the build time may distract from performance of the lab.  If you
create your own public repository on the Docker Hub (https://hub.docker.com/), you can populate that
with your lab(s) by setting the "REGSITRY\_ID" value in the start.config file for the lab(s). You
would then use the distrib/publish.py script to build, tag and push your lab to your registry.

The distrib/mkdist.sh script is used by NPS to create the distribution tar file.  This script relies on
an internal NPS subversion repository as the source to the Labtainer scripts and labs.  The distrib/publish.py
script is used to rebuild and publish all of the Labtainer labs managed by NPS.

\subsection{Reuse of large file sets}
The use of "sys\_tar" desribed in \ref{large files} facilitates sharing of common
baselines of large or numerous files.  New labs should use symbolic links to
existing sys\_tar/sys.tar files.  The framework will replace these  symbolic links
with the actual tar files, (to satisfy Docker), whenever a lab is built.  However, when
maintained in svn, the designer should ensure the file is a symbolic link.  This will
avoid duplication of the files when a new distribution is created.

\section{Ubuntu Package Sources}
Labtainer images built on Ubuntu use apt-get to install packages.  Some enterprise environments have 
difficulty reaching selected Ubuntu software source mirrors.  A Docker image build will use the default 
Ubuntu package sources, "archive.ubuntu.com".  This hostname can be overridden via the 
trunk/config/labtainer.config file apt\_source entry, and having the following in your dockerfile:
\begin{verbatim}
RUN sed -i s/archive.ubuntu.com/$apt_source/ /etc/apt/sources.list
\end{verbatim}
\noindent prior to the installation of packages.  The Labatainer labs from NPS 
generally have sources.list entries that reflect sources that were working from NPS at the time of the build.

\section{Locale Settings}
The locale settings, (e.g., used when interpreting character encodings) are set to en\_US.utf-8 
as can be seen in 
\begin{verbatim}
trunk/scripts/labdesigner/base_dockerfiles/Dockerfile.labtainer.base
\end{verbatim}
Similar Dockerfile entries in new or existing labs can provide alternate locale settings.

\section{Limitations}
The labtainers framework limits labs to the Linux execution environment.
However, a lab designer could prescribe the inclusion of a separate
VM, e.g., a Windows system, and that VM could be networked with the Linux
VM that hosts the Docker containers.  Future work would be necessary to include
artifacts from the Windows system within the framework's automated assessment
and parameterization.

The process tree of the initial Linux process will
not look like a typical Linux system \textit{init} process.  Within containers that have no
services, the initial process, i.e., process ID 1, will be a bash shell.
Containers having services and logging will have an initial process that is
the script that launches the services, e.g., the \textit{faux\_init} script.  However,
other process tree's will appear as they
do in Linux system, and this includes inetd services.

Inquisitive students will see evidence of artifact collection.  Home directories
on containers includes a \texttt{.local} directory that includes Labtainer scripts that manage
capturing and collection of artifacts, and that directory contains the stdin and
stdout files generated by student actions. Additionally, when the student starts a process
that will have stdin and stdout captured, the student will see extra processes within
that process tree, e.g., the \texttt{tee} function that generates copies of those data streams.
All of the containers share the Linux kernel with the Linux host.  Changes to
kernel configuration settings, e.g., enabling ASLR, will be visible across all
of the containers.

\end{document}
