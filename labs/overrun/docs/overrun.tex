\documentclass[11pt]{article}

\usepackage{times}
\usepackage{epsf}
\usepackage{epsfig}
\usepackage{amsmath, alltt, amssymb, xspace}
\usepackage{wrapfig}
\usepackage{fancyhdr}
\usepackage{url}
\usepackage{verbatim}
\usepackage{fancyvrb}

\usepackage{subfigure}
\usepackage{cite}
%\usepackage{cases}
%\usepackage{ltexpprt}
%\usepackage{verbatim}

%\topmargin      -0.70in  % distance to headers
%\headheight     0.2in   % height of header box
%\headsep        0.4in   % distance to top line
%\footskip       0.3in   % distance from bottom line

% Horizontal alignment
\topmargin      -0.50in  % distance to headers
\oddsidemargin  0.0in
\evensidemargin 0.0in
\textwidth      6.5in
\textheight     8.9in 


%\centerfigcaptionstrue

%\def\baselinestretch{0.95}


\newcommand\discuss[1]{\{\textbf{Discuss:} \textit{#1}\}}
%\newcommand\todo[1]{\vspace{0.1in}\{\textbf{Todo:} \textit{#1}\}\vspace{0.1in}}
\newtheorem{problem}{Problem}[section]
%\newtheorem{theorem}{Theorem}
%\newtheorem{fact}{Fact}
\newtheorem{define}{Definition}[section]
%\newtheorem{analysis}{Analysis}
\newcommand\vspacenoindent{\vspace{0.1in} \noindent}

%\newenvironment{proof}{\noindent {\bf Proof}.}{\hspace*{\fill}~\mbox{\rule[0pt]{1.3ex}{1.3ex}}}
%\newcommand\todo[1]{\vspace{0.1in}\{\textbf{Todo:} \textit{#1}\}\vspace{0.1in}}

%\newcommand\reducespace{\vspace{-0.1in}}
% reduce the space between lines
%\def\baselinestretch{0.95}

\newcommand{\fixmefn}[1]{ \footnote{\sf\ \ \fbox{FIXME} #1} }
\newcommand{\todo}[1]{
\vspace{0.1in}
\fbox{\parbox{6in}{TODO: #1}}
\vspace{0.1in}
}

\newcommand{\mybox}[1]{
\vspace{0.2in}
\noindent
\fbox{\parbox{6.5in}{#1}}
\vspace{0.1in}
}


\newcounter{question}
\setcounter{question}{1}

\newcommand{\myquestion} {{\vspace{0.1in} \noindent \bf Question \arabic{question}:} \addtocounter{question}{1} \,}

\newcommand{\myproblem} {{\noindent \bf Problem \arabic{question}:} \addtocounter{question}{1} \,}


\newcommand{\copyrightnoticeA}[1]{
\vspace{0.1in}
\fbox{\parbox{6in}{\small Copyright \copyright\ 2006 - 2014\ \ Wenliang Du, Syracuse University.\\ 
      The development of this document is partially funded by 
      the National Science Foundation's Course, Curriculum, and Laboratory 
      Improvement (CCLI) program under Award No. 0618680 and 0231122. 
      Permission is granted to copy, distribute and/or modify this document
      under the terms of the GNU Free Documentation License, Version 1.2
      or any later version published by the Free Software Foundation.
      A copy of the license can be found at http://www.gnu.org/licenses/fdl.html.}}
\vspace{0.1in}
}


\newcommand{\copyrightnotice}[1]{
\vspace{0.1in}
\fbox{\parbox{6in}{\small Copyright \copyright\ 2006 - 2014\ \ Wenliang Du, Syracuse University.\\
      The development of this document is/was funded by three grants from
      the US National Science Foundation: Awards No. 0231122 and 0618680 from
      TUES/CCLI and  Award No. 1017771 from Trustworthy Computing.
      This lab was imported into the Labtainer framework by the Naval Postgraduate 
      School, Center for Cybersecurity and Cyber Operations under National Science 
      Foundation Award No. 1438893.
      Permission is granted to copy, distribute and/or modify this document
      under the terms of the GNU Free Documentation License, Version 1.2
      or any later version published by the Free Software Foundation.
      A copy of the license can be found at http://www.gnu.org/licenses/fdl.html.}}
\vspace{0.1in}
}

\newcommand{\copyrightnoticeB}[1]{
\vspace{0.1in}
\fbox{\parbox{6in}{\small Copyright \copyright\ 2006 - 2014\ \ Wenliang Du, Syracuse University.\\
      The development of this document is/was funded by the following grants from
      the US National Science Foundation: No. 0231122, 0618680, and 1303306.
      Permission is granted to copy, distribute and/or modify this document
      under the terms of the GNU Free Documentation License, Version 1.2
      or any later version published by the Free Software Foundation.
      A copy of the license can be found at http://www.gnu.org/licenses/fdl.html.}}
\vspace{0.1in}
}


\newcommand{\nocopyrightnotice}[1]{
\vspace{0.1in}
\fbox{\parbox{6in}{\small  
      The development of this document is funded by 
      the National Science Foundation's Course, Curriculum, and Laboratory 
      Improvement (CCLI) program under Award No. 0618680 and 0231122. 
      Permission is granted to copy, distribute and/or modify this document.
      }}
\vspace{0.1in}
}

\newcommand{\idea}[1]{
\vspace{0.1in}
{\sf IDEA:\ \ \fbox{\parbox{5in}{#1}}}
\vspace{0.1in}
}

\newcommand{\questionblock}[1]{
\vspace{0.1in}
\fbox{\parbox{6in}{#1}}
\vspace{0.1in}
}


\newcommand{\minix}{{\tt Minix}\xspace}
\newcommand{\unix}{{\tt Unix}\xspace}
\newcommand{\linux}{{\tt Linux}\xspace}
\newcommand{\ubuntu}{{\tt Ubuntu}\xspace}
\newcommand{\selinux}{{\tt SELinux}\xspace}
\newcommand{\freebsd}{{\tt FreeBSD}\xspace}
\newcommand{\solaris}{{\tt Solaris}\xspace}
\newcommand{\windowsnt}{{\tt Windows NT}\xspace}
\newcommand{\setuid}{{\tt Set-UID}\xspace}
%\newcommand{\smx}{{\tt Smx}\xspace}
\newcommand{\smx}{{\tt Minix}\xspace}
\newcommand{\relay}{{\tt relay}\xspace}
\newcommand{\isys}{{\tt iSYS}\xspace}
\newcommand{\ilan}{{\tt iLAN}\xspace}
\newcommand{\iSYS}{{\tt iSYS}\xspace}
\newcommand{\iLAN}{{\tt iLAN}\xspace}
\newcommand{\iLANs}{{\tt iLAN}s\xspace}
\newcommand{\bochs}{{\tt Bochs}\xspace}

\newcommand\FF{{\mathcal{F}}}

\newcommand{\argmax}[1]{
\begin{minipage}[t]{1.25cm}\parskip-1ex\begin{center}
argmax
#1
\end{center}\end{minipage}
\;
}

\newcommand{\bm}{\boldmath}
\newcommand  {\bx}    {\mbox{\boldmath $x$}}
\newcommand  {\by}    {\mbox{\boldmath $y$}}
\newcommand  {\br}    {\mbox{\boldmath $r$}}


%\pagestyle{fancyplain}
%\lhead[\thepage]{\thesection}      % Note the different brackets!
%\rhead[\thesection]{SEED Laboratories}
%\lfoot[\fancyplain{}{}]{Syracuse University} 
%\cfoot[\fancyplain{}{}]{\thepage} 

\newcommand{\tstamp}{\today}   
%\lhead[\fancyplain{}{\thepage}]         {\fancyplain{}{\rightmark}}
%\chead[\fancyplain{}{}]                 {\fancyplain{}{}}
%\rhead[\fancyplain{}{\rightmark}]       {\fancyplain{}{\thepage}}
%\lfoot[\fancyplain{}{}]                 {\fancyplain{\tstamp}{\tstamp}}
%\cfoot[\fancyplain{\thepage}{}]         {\fancyplain{\thepage}{}}
%\rfoot[\fancyplain{\tstamp} {\tstamp}]  {\fancyplain{}{}}

\pagestyle{fancy}
%\lhead{\bfseries Computer Security Course Project}
\lhead{\bfseries SEED Labs}
\chead{}
\rhead{\small \thepage}
\lfoot{}
\cfoot{}
\rfoot{}

\usepackage{listings}
\usepackage{color}

\definecolor{dkgreen}{rgb}{0,0.6,0}
\definecolor{gray}{rgb}{0.5,0.5,0.5}
\definecolor{mauve}{rgb}{0.58,0,0.82}

\lstset{frame=tb,
  language=C,
  aboveskip=3mm,
  belowskip=3mm,
  showstringspaces=false,
  columns=flexible,
  basicstyle={\small\ttfamily},
  numbers=none,
  numberstyle=\tiny\color{gray},
  keywordstyle=\color{blue},
  commentstyle=\color{dkgreen},
  stringstyle=\color{mauve},
  breaklines=true,
  breakatwhitespace=true,
  tabsize=3
}



\begin{document}

\begin{center}
{\LARGE Overrun of intended bounds in a C program}
\vspace{0.1in}\\
\end{center}


\section{Overview}
This exercise illustrates overrunning the intended bounds of
data structures in a C program.   It introduces memory 
addressing concepts that are a key part of understanding software
vulnerabilities that may lead to memory corruption or data leakage.
This lab provides an introduction to techniques that are used
in the buffer overflow (bufoverflow and buf64) and the printf labs
(formatstring and format64).

\subsection{Background}
This exercise assumes the student has some basic C language programming
experience and is familiar with simple data structures.
No coding is required in this lab, but it will help if the student
can understand a simple C program.

The GDB program is used to explore the executing program, including viewing
a bit of its disassembly.  However no assembly language background is necessary to perform
the lab.

\section{Lab Environment}
This lab runs in the Labtainer framework,
available at http://nps.edu/web/c3o/labtainers.
That site includes links to a pre-built virtual machine
that has Labtainers installed, however Labtainers can
be run on any Linux host that supports Docker containers
or on Docker Desktop on PCs and Macs.

From your labtainer-student directory start the lab using:
\begin{verbatim}
    labtainer overrun
\end{verbatim}
A link to this lab manual will be displayed.


\section{Tasks}

\subsection{Review the mystuff.c program}
A terminal opens when you start the lab.  At that terminal, view the mystuff.c program.  Use either {\tt vi} or {\tt nano}, or just type {\tt less mystuff.c}.

\subsubsection{The myData structure}
Look at the {\tt myData} structure.  
In the program, we declare the variable {\tt my\_data} to be a {\tt myData} struct.
Note the {\tt public\_info} character array has 20 elements.  As with any array, we can refer 
to elements of the array using an index. For example, {\tt my\_data.public\_info[4]} refers to the
fifth character in the array, and {\tt my\_data.public\_info[19]} refers to the very last character
in the array.

If 19 is the very last character in the array, what would {\tt my\_data.public\_info[20]} refer to?

\subsubsection{Addresses of fields}
After the program initializes the {\tt my\_data} structure, it displays the addresses of the
start of the {\tt public\_data} field, and the {\tt pin} field.  And it displays the memory values of those
fields.

\subsubsection{Memory content}
The program then enters a loop in which it allows the user to display hex values of individual characters
within the {\tt public\_info} field.  It is this loop that will let us explore the question asked earlier,
nameley: what would {\tt my\_data.public\_info[20]} refer to?

\subsection{Compile and run the program}
Use this command to compile the program:
\begin{verbatim}
    gcc -m32 -g -o mystuff mystuff.c
\end{verbatim}
\noindent Note the {\tt -m32} switch creates a 32-bit binary and the {\tt -g} switch includes symbols in the 
binary that will let us explore the program's execution using {\tt gdb}.

Run the program:
\begin{verbatim}
    ./mystuff
\end{verbatim}
\noindent and explore the values displayed at different offsets within (and beyond) the {\tt public\_info} field.
Note the displayed address of the {\tt public\_info} field and the address of the {\tt pin} field.  How many 
bytes separate the two fields?  Use the program to display the value of the pin field.  Note that if
your {\tt fav\_color} buffer size is odd, the compiler will \textit{pad} the buffer so that the next
variable starts on 4-byte word boundary.

\subsection{Explore with gdb}
Run the program under the GDB debugger:
\begin{verbatim}
    gdb mystuff
\end{verbatim}
\noindent Use the {\tt list} command to view the source code.  Set a breakpoint in the {\tt showMemory} 
function on the line where it will print the value at the given offset. (Use {\tt list showMemory} to view
source for that function.)  And then run the program from within
gdb:
\begin{verbatim}
   break <line number>
   run
\end{verbatim}
\noindent When the program hits the breakpoint, display 10 words (40 bytes) of system memory as hex
values starting at the {\tt data} structure:
\begin{verbatim}
   x/10x &data
\end{verbatim}
Does the memory content correspond to what you observed while running the program?

\subsubsection{Extra exploration}
Set a breakpoint at the end of the {\tt handleMyStuff} funtion, i.e., on the line of the final right brace
({\tt\}}) in that function.  Then continue with the {\tt c} command.  At the prompt for the next offset,
enter a q.  Then, when the program hits the breakpoint, display the disassembled program using:
\begin{verbatim}
   display/i $pc
   stepi
\end{verbatim}
\noindent And single step through the remainder of the {\tt handleMyStuff} function disassembly by
repeatedly pressing the {\tt Return} key until the program gets to the {\tt ret} instruction.

This is the point in the program at which the {\tt handleMyStuff} function will return to the {\tt main}
function.  The {\tt ret} instruction directs the processor to jump to the instruction at the address
contained at the current stack pointer.  Display the memory content pointed to by the stack register using:
\begin{verbatim}
  x $esp
\end{verbatim}
\noindent  The displayed value will become the next instruction address, which you can confirm using one
more {\tt nexti}.   Make note of that current instruction pointer.  
Look again at the stack address that held this return value.  Note that it is higher
than the address of the {\tt data} structure observed in the {\tt showMemory} function.  Compute 
and record the difference between the two addresses.

Rerun the program outside of the debugger and use it to display the return address value, one byte at a time.
Confirm that address is what you observed in gdb\footnote{The address may appear backwards to you.  Don't let 
that hang you up, it is an artifact of machine architecture that you can learn about 
by googling \textit{Endianess}}.  Imagine that the program let us modify the individual 
items in the {\tt public\_info} array.  When the program hits the {\tt ret} instruction that you viewed
in gdb, it would then \textit{return} to an address you wrote.

\section{Submission}
After finishing the lab, go to the terminal on your Linux system that was used to start the lab and type:
\begin{verbatim}
    stoplab 
\end{verbatim}
When you stop the lab, the system will display a path to the zipped lab results on your Linux system.  Provide that file to 
your instructor, e.g., via the Sakai site.

\copyrightnotice

\end{document}
