%\newpage
%\setcounter{page}{1}
%\setcounter{section}{0}

\documentclass[11pt]{article}

\usepackage{times}
\usepackage{epsf}
\usepackage{epsfig}
\usepackage{amsmath, alltt, amssymb, xspace}
\usepackage{wrapfig}
\usepackage{fancyhdr}
\usepackage{url}
\usepackage{verbatim}
\usepackage{fancyvrb}

\usepackage{subfigure}
\usepackage{cite}
%\usepackage{cases}
%\usepackage{ltexpprt}
%\usepackage{verbatim}

%\topmargin      -0.70in  % distance to headers
%\headheight     0.2in   % height of header box
%\headsep        0.4in   % distance to top line
%\footskip       0.3in   % distance from bottom line

% Horizontal alignment
\topmargin      -0.50in  % distance to headers
\oddsidemargin  0.0in
\evensidemargin 0.0in
\textwidth      6.5in
\textheight     8.9in 


%\centerfigcaptionstrue

%\def\baselinestretch{0.95}


\newcommand\discuss[1]{\{\textbf{Discuss:} \textit{#1}\}}
%\newcommand\todo[1]{\vspace{0.1in}\{\textbf{Todo:} \textit{#1}\}\vspace{0.1in}}
\newtheorem{problem}{Problem}[section]
%\newtheorem{theorem}{Theorem}
%\newtheorem{fact}{Fact}
\newtheorem{define}{Definition}[section]
%\newtheorem{analysis}{Analysis}
\newcommand\vspacenoindent{\vspace{0.1in} \noindent}

%\newenvironment{proof}{\noindent {\bf Proof}.}{\hspace*{\fill}~\mbox{\rule[0pt]{1.3ex}{1.3ex}}}
%\newcommand\todo[1]{\vspace{0.1in}\{\textbf{Todo:} \textit{#1}\}\vspace{0.1in}}

%\newcommand\reducespace{\vspace{-0.1in}}
% reduce the space between lines
%\def\baselinestretch{0.95}

\newcommand{\fixmefn}[1]{ \footnote{\sf\ \ \fbox{FIXME} #1} }
\newcommand{\todo}[1]{
\vspace{0.1in}
\fbox{\parbox{6in}{TODO: #1}}
\vspace{0.1in}
}

\newcommand{\mybox}[1]{
\vspace{0.2in}
\noindent
\fbox{\parbox{6.5in}{#1}}
\vspace{0.1in}
}


\newcounter{question}
\setcounter{question}{1}

\newcommand{\myquestion} {{\vspace{0.1in} \noindent \bf Question \arabic{question}:} \addtocounter{question}{1} \,}

\newcommand{\myproblem} {{\noindent \bf Problem \arabic{question}:} \addtocounter{question}{1} \,}


\newcommand{\copyrightnoticeA}[1]{
\vspace{0.1in}
\fbox{\parbox{6in}{\small Copyright \copyright\ 2006 - 2014\ \ Wenliang Du, Syracuse University.\\ 
      The development of this document is partially funded by 
      the National Science Foundation's Course, Curriculum, and Laboratory 
      Improvement (CCLI) program under Award No. 0618680 and 0231122. 
      Permission is granted to copy, distribute and/or modify this document
      under the terms of the GNU Free Documentation License, Version 1.2
      or any later version published by the Free Software Foundation.
      A copy of the license can be found at http://www.gnu.org/licenses/fdl.html.}}
\vspace{0.1in}
}


\newcommand{\copyrightnotice}[1]{
\vspace{0.1in}
\fbox{\parbox{6in}{\small Copyright \copyright\ 2006 - 2014\ \ Wenliang Du, Syracuse University.\\
      The development of this document is/was funded by three grants from
      the US National Science Foundation: Awards No. 0231122 and 0618680 from
      TUES/CCLI and  Award No. 1017771 from Trustworthy Computing.
      This lab was imported into the Labtainer framework by the Naval Postgraduate 
      School, Center for Cybersecurity and Cyber Operations under National Science 
      Foundation Award No. 1438893.
      Permission is granted to copy, distribute and/or modify this document
      under the terms of the GNU Free Documentation License, Version 1.2
      or any later version published by the Free Software Foundation.
      A copy of the license can be found at http://www.gnu.org/licenses/fdl.html.}}
\vspace{0.1in}
}

\newcommand{\copyrightnoticeB}[1]{
\vspace{0.1in}
\fbox{\parbox{6in}{\small Copyright \copyright\ 2006 - 2014\ \ Wenliang Du, Syracuse University.\\
      The development of this document is/was funded by the following grants from
      the US National Science Foundation: No. 0231122, 0618680, and 1303306.
      Permission is granted to copy, distribute and/or modify this document
      under the terms of the GNU Free Documentation License, Version 1.2
      or any later version published by the Free Software Foundation.
      A copy of the license can be found at http://www.gnu.org/licenses/fdl.html.}}
\vspace{0.1in}
}


\newcommand{\nocopyrightnotice}[1]{
\vspace{0.1in}
\fbox{\parbox{6in}{\small  
      The development of this document is funded by 
      the National Science Foundation's Course, Curriculum, and Laboratory 
      Improvement (CCLI) program under Award No. 0618680 and 0231122. 
      Permission is granted to copy, distribute and/or modify this document.
      }}
\vspace{0.1in}
}

\newcommand{\idea}[1]{
\vspace{0.1in}
{\sf IDEA:\ \ \fbox{\parbox{5in}{#1}}}
\vspace{0.1in}
}

\newcommand{\questionblock}[1]{
\vspace{0.1in}
\fbox{\parbox{6in}{#1}}
\vspace{0.1in}
}


\newcommand{\minix}{{\tt Minix}\xspace}
\newcommand{\unix}{{\tt Unix}\xspace}
\newcommand{\linux}{{\tt Linux}\xspace}
\newcommand{\ubuntu}{{\tt Ubuntu}\xspace}
\newcommand{\selinux}{{\tt SELinux}\xspace}
\newcommand{\freebsd}{{\tt FreeBSD}\xspace}
\newcommand{\solaris}{{\tt Solaris}\xspace}
\newcommand{\windowsnt}{{\tt Windows NT}\xspace}
\newcommand{\setuid}{{\tt Set-UID}\xspace}
%\newcommand{\smx}{{\tt Smx}\xspace}
\newcommand{\smx}{{\tt Minix}\xspace}
\newcommand{\relay}{{\tt relay}\xspace}
\newcommand{\isys}{{\tt iSYS}\xspace}
\newcommand{\ilan}{{\tt iLAN}\xspace}
\newcommand{\iSYS}{{\tt iSYS}\xspace}
\newcommand{\iLAN}{{\tt iLAN}\xspace}
\newcommand{\iLANs}{{\tt iLAN}s\xspace}
\newcommand{\bochs}{{\tt Bochs}\xspace}

\newcommand\FF{{\mathcal{F}}}

\newcommand{\argmax}[1]{
\begin{minipage}[t]{1.25cm}\parskip-1ex\begin{center}
argmax
#1
\end{center}\end{minipage}
\;
}

\newcommand{\bm}{\boldmath}
\newcommand  {\bx}    {\mbox{\boldmath $x$}}
\newcommand  {\by}    {\mbox{\boldmath $y$}}
\newcommand  {\br}    {\mbox{\boldmath $r$}}


%\pagestyle{fancyplain}
%\lhead[\thepage]{\thesection}      % Note the different brackets!
%\rhead[\thesection]{SEED Laboratories}
%\lfoot[\fancyplain{}{}]{Syracuse University} 
%\cfoot[\fancyplain{}{}]{\thepage} 

\newcommand{\tstamp}{\today}   
%\lhead[\fancyplain{}{\thepage}]         {\fancyplain{}{\rightmark}}
%\chead[\fancyplain{}{}]                 {\fancyplain{}{}}
%\rhead[\fancyplain{}{\rightmark}]       {\fancyplain{}{\thepage}}
%\lfoot[\fancyplain{}{}]                 {\fancyplain{\tstamp}{\tstamp}}
%\cfoot[\fancyplain{\thepage}{}]         {\fancyplain{\thepage}{}}
%\rfoot[\fancyplain{\tstamp} {\tstamp}]  {\fancyplain{}{}}

\pagestyle{fancy}
%\lhead{\bfseries Computer Security Course Project}
\lhead{\bfseries SEED Labs}
\chead{}
\rhead{\small \thepage}
\lfoot{}
\cfoot{}
\rfoot{}

\usepackage{listings}
\usepackage{color}

\definecolor{dkgreen}{rgb}{0,0.6,0}
\definecolor{gray}{rgb}{0.5,0.5,0.5}
\definecolor{mauve}{rgb}{0.58,0,0.82}

\lstset{frame=tb,
  language=C,
  aboveskip=3mm,
  belowskip=3mm,
  showstringspaces=false,
  columns=flexible,
  basicstyle={\small\ttfamily},
  numbers=none,
  numberstyle=\tiny\color{gray},
  keywordstyle=\color{blue},
  commentstyle=\color{dkgreen},
  stringstyle=\color{mauve},
  breaklines=true,
  breakatwhitespace=true,
  tabsize=3
}





\begin{document}



\begin{center}
{\LARGE Crypto Lab -- One-Way Hash Function and MAC}
\end{center}

\copyrightnotice

\newcounter{task}
\setcounter{task}{1}
\newcommand{\tasks} {\bf {\noindent (\arabic{task})} \addtocounter{task}{1} \,}

\section{Overview}

The learning objective of this lab is for students to get familiar with
one-way hash functions and Message Authentication Code (MAC). After 
finishing the lab, in addition to gaining a deeper undertanding
of the concepts, students should be able to 
use tools and 
write programs to generate one-way hash value and MAC for 
a given message.


\section{Lab Environment}
The lab is started from the Labtainer working
directory on your Docker-enabled host, e.g., a Linux VM.
From there, issue the command:
\begin{verbatim}
    start.py onewayhash
\end{verbatim}
The resulting virtual terminals will include a display of
a bash shell.  The openssl package
and other software described below are pre-installed
on the system.


\section{Lab Tasks}

\subsection{Task 1: Generating Message Digest and MAC}

In this task, we will play with various one-way hash algorithms. 
You can use the following {\tt openssl dgst} 
command to generate the hash value for a file. To see the manpages, you can 
type {\tt man openssl} and {\tt man dgst}.

\begin{Verbatim}[frame=single] 
  % openssl dgst dgsttype filename
\end{Verbatim} 

Please replace the {\tt dgsttype} with a specific one-way hash algorithm,
such as {\tt -md5}, {\tt -sha1}, {\tt -sha256}, etc.
And replace {\tt filename} with {\tt filetodigest.txt}, which is in your home directory.
In this task, you should try at least 3 different algorithms,
and describe your observations. 
You can find the supported one-way hash algorithms 
by typing {\tt "openssl dgst -h"}  NOTE: the list of algorithms included in the manpages is not correct.


\subsection{Task 2: Keyed Hash and HMAC}

In this task, we would like to generate a keyed hash (i.e. MAC)
for a file. We can use the {\tt -hmac} option (this option 
is currently undocumented,
but it is supported by {\tt openssl}). The following example
generates a keyed hash for a file using the HMAC-MD5 
algorithm. The string following the {\tt -hmac} option
is the key. 
\begin{Verbatim}[frame=single]
 % openssl dgst -md5 -hmac "abcdefg" filename 
\end{Verbatim} 

Please generate a keyed hash using HMAC-MD5, HMAC-SHA256, and
HMAC-SHA1 for any file that you choose. Please try several
keys with different length. Do we have to use a key 
with a fixed size in HMAC? If so, what is the key size? If 
not, why?


\subsection{Task 3: The Randomness of One-way Hash}


To understand the properties of one-way hash functions, we would like to
do the following exercise for MD5 and SHA256:

\begin{enumerate}
\item Create a text file named "edit-this-file.txt" of any length.
\item Generate the hash value  $H_1$ for this file using a specific hash algorithm.
\item Flip one bit of the input file.
      You can achieve this modification using {\tt hexedit}.
\item Generate the hash value $H_2$ for the modified file.
\item Please observe whether $H_1$ and $H_2$ are similar or not. Please describe
      your observations in the lab report. You can write a short program to count
      how many bits are the same between $H_1$ and $H_2$.

\end{enumerate}


\subsection{Task 4: One-Way Property versus Collision-Free Property}


In this task, we will investigate the difference between 
hash function's two perperties: one-way property versus collision-free property. 
We will use the brute-force method to see how long it takes to break
each of these properties. Instead of using {\tt openssl}'s command-line tools, 
you are required to write our own C programs to invoke the 
message digest functions in {\tt openssl}'s crypto library. 
A sample code can be found from 
\url{http://www.openssl.org/docs/crypto/EVP_DigestInit.html}. Please 
get familiar with this sample code.

Since most of the hash functions are quite strong against the brute-force
attack on those two properties, it will take us years to break them
using the brute-force method. To make the task feasible, we reduce 
the length of the hash value to 24 bits. We can use any one-way
hash function, but we only use the first 24 bits of the hash value
in this task. Namely, we are using a modified one-way hash function.
Please design an experiment to find out the following:

\begin{enumerate}
\item How many trials it will take you to break the one-way
property using the brute-force method? You should repeat your 
experiment for multiple times, and report your average number 
of trials.
\item How many trials it will take you to break the collision-free
property using the brute-force method? Similarly, you should 
report the average.
\item Based on your observation, which property is easier to 
break using the brute-force method?
\item (10 Bonus Points) Can you explain the difference in your 
observation mathematically?
\end{enumerate}

 

\section{Submission}
When the lab is completed, or you'd like to stop working for a while, run
\begin{verbatim}
    stop.py onewayhash
\end{verbatim}

from the host Labtainer working directory.  You can always restart the
Labtainer to continue your work.  When the Labtainer is stopped, a
zip file is created and copied to a location displayed by the stop.py
command.  When the lab is completed, send that zip file to the instructor.

You need to submit a detailed lab report to describe what you have
done and what you have observed; you also need to provide explanation
to the observations that are interesting or surprising. In your report,
you need to answer all the questions listed in this lab.




\end{document}


