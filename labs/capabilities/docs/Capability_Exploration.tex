%%%%%%%%%%%%%%%%%%%%%%%%%%%%%%%%%%%%%%%%%%%%%%%%%%%%%%%%%%%%%%%%%%%
% Copyright (c) 2006 Wenliang Du, Syracuse University.
% The development of this document is funded by Grant DUE-0231122
% from the National Science Foundation. Permission is granted
% to copy, distribute and/or modify this document under the terms
% of the GNU Free Documentation License, Version 1.2 or any later
% version published by the Free Software Foundation. A copy of the
% license can be found at http://www.gnu.org/licenses/fdl.html.
%%%%%%%%%%%%%%%%%%%%%%%%%%%%%%%%%%%%%%%%%%%%%%%%%%%%%%%%%%%%%%%%%%%

\documentclass[11pt]{article}

\usepackage{times}
\usepackage{epsf}
\usepackage{epsfig}
\usepackage{amsmath, alltt, amssymb, xspace}
\usepackage{wrapfig}
\usepackage{fancyhdr}
\usepackage{url}
\usepackage{verbatim}
\usepackage{fancyvrb}

\usepackage{subfigure}
\usepackage{cite}
%\usepackage{cases}
%\usepackage{ltexpprt}
%\usepackage{verbatim}

%\topmargin      -0.70in  % distance to headers
%\headheight     0.2in   % height of header box
%\headsep        0.4in   % distance to top line
%\footskip       0.3in   % distance from bottom line

% Horizontal alignment
\topmargin      -0.50in  % distance to headers
\oddsidemargin  0.0in
\evensidemargin 0.0in
\textwidth      6.5in
\textheight     8.9in 


%\centerfigcaptionstrue

%\def\baselinestretch{0.95}


\newcommand\discuss[1]{\{\textbf{Discuss:} \textit{#1}\}}
%\newcommand\todo[1]{\vspace{0.1in}\{\textbf{Todo:} \textit{#1}\}\vspace{0.1in}}
\newtheorem{problem}{Problem}[section]
%\newtheorem{theorem}{Theorem}
%\newtheorem{fact}{Fact}
\newtheorem{define}{Definition}[section]
%\newtheorem{analysis}{Analysis}
\newcommand\vspacenoindent{\vspace{0.1in} \noindent}

%\newenvironment{proof}{\noindent {\bf Proof}.}{\hspace*{\fill}~\mbox{\rule[0pt]{1.3ex}{1.3ex}}}
%\newcommand\todo[1]{\vspace{0.1in}\{\textbf{Todo:} \textit{#1}\}\vspace{0.1in}}

%\newcommand\reducespace{\vspace{-0.1in}}
% reduce the space between lines
%\def\baselinestretch{0.95}

\newcommand{\fixmefn}[1]{ \footnote{\sf\ \ \fbox{FIXME} #1} }
\newcommand{\todo}[1]{
\vspace{0.1in}
\fbox{\parbox{6in}{TODO: #1}}
\vspace{0.1in}
}

\newcommand{\mybox}[1]{
\vspace{0.2in}
\noindent
\fbox{\parbox{6.5in}{#1}}
\vspace{0.1in}
}


\newcounter{question}
\setcounter{question}{1}

\newcommand{\myquestion} {{\vspace{0.1in} \noindent \bf Question \arabic{question}:} \addtocounter{question}{1} \,}

\newcommand{\myproblem} {{\noindent \bf Problem \arabic{question}:} \addtocounter{question}{1} \,}


\newcommand{\copyrightnoticeA}[1]{
\vspace{0.1in}
\fbox{\parbox{6in}{\small Copyright \copyright\ 2006 - 2014\ \ Wenliang Du, Syracuse University.\\ 
      The development of this document is partially funded by 
      the National Science Foundation's Course, Curriculum, and Laboratory 
      Improvement (CCLI) program under Award No. 0618680 and 0231122. 
      Permission is granted to copy, distribute and/or modify this document
      under the terms of the GNU Free Documentation License, Version 1.2
      or any later version published by the Free Software Foundation.
      A copy of the license can be found at http://www.gnu.org/licenses/fdl.html.}}
\vspace{0.1in}
}


\newcommand{\copyrightnotice}[1]{
\vspace{0.1in}
\fbox{\parbox{6in}{\small Copyright \copyright\ 2006 - 2014\ \ Wenliang Du, Syracuse University.\\
      The development of this document is/was funded by three grants from
      the US National Science Foundation: Awards No. 0231122 and 0618680 from
      TUES/CCLI and  Award No. 1017771 from Trustworthy Computing.
      This lab was imported into the Labtainer framework by the Naval Postgraduate 
      School, Center for Cybersecurity and Cyber Operations under National Science 
      Foundation Award No. 1438893.
      Permission is granted to copy, distribute and/or modify this document
      under the terms of the GNU Free Documentation License, Version 1.2
      or any later version published by the Free Software Foundation.
      A copy of the license can be found at http://www.gnu.org/licenses/fdl.html.}}
\vspace{0.1in}
}

\newcommand{\copyrightnoticeB}[1]{
\vspace{0.1in}
\fbox{\parbox{6in}{\small Copyright \copyright\ 2006 - 2014\ \ Wenliang Du, Syracuse University.\\
      The development of this document is/was funded by the following grants from
      the US National Science Foundation: No. 0231122, 0618680, and 1303306.
      Permission is granted to copy, distribute and/or modify this document
      under the terms of the GNU Free Documentation License, Version 1.2
      or any later version published by the Free Software Foundation.
      A copy of the license can be found at http://www.gnu.org/licenses/fdl.html.}}
\vspace{0.1in}
}


\newcommand{\nocopyrightnotice}[1]{
\vspace{0.1in}
\fbox{\parbox{6in}{\small  
      The development of this document is funded by 
      the National Science Foundation's Course, Curriculum, and Laboratory 
      Improvement (CCLI) program under Award No. 0618680 and 0231122. 
      Permission is granted to copy, distribute and/or modify this document.
      }}
\vspace{0.1in}
}

\newcommand{\idea}[1]{
\vspace{0.1in}
{\sf IDEA:\ \ \fbox{\parbox{5in}{#1}}}
\vspace{0.1in}
}

\newcommand{\questionblock}[1]{
\vspace{0.1in}
\fbox{\parbox{6in}{#1}}
\vspace{0.1in}
}


\newcommand{\minix}{{\tt Minix}\xspace}
\newcommand{\unix}{{\tt Unix}\xspace}
\newcommand{\linux}{{\tt Linux}\xspace}
\newcommand{\ubuntu}{{\tt Ubuntu}\xspace}
\newcommand{\selinux}{{\tt SELinux}\xspace}
\newcommand{\freebsd}{{\tt FreeBSD}\xspace}
\newcommand{\solaris}{{\tt Solaris}\xspace}
\newcommand{\windowsnt}{{\tt Windows NT}\xspace}
\newcommand{\setuid}{{\tt Set-UID}\xspace}
%\newcommand{\smx}{{\tt Smx}\xspace}
\newcommand{\smx}{{\tt Minix}\xspace}
\newcommand{\relay}{{\tt relay}\xspace}
\newcommand{\isys}{{\tt iSYS}\xspace}
\newcommand{\ilan}{{\tt iLAN}\xspace}
\newcommand{\iSYS}{{\tt iSYS}\xspace}
\newcommand{\iLAN}{{\tt iLAN}\xspace}
\newcommand{\iLANs}{{\tt iLAN}s\xspace}
\newcommand{\bochs}{{\tt Bochs}\xspace}

\newcommand\FF{{\mathcal{F}}}

\newcommand{\argmax}[1]{
\begin{minipage}[t]{1.25cm}\parskip-1ex\begin{center}
argmax
#1
\end{center}\end{minipage}
\;
}

\newcommand{\bm}{\boldmath}
\newcommand  {\bx}    {\mbox{\boldmath $x$}}
\newcommand  {\by}    {\mbox{\boldmath $y$}}
\newcommand  {\br}    {\mbox{\boldmath $r$}}


%\pagestyle{fancyplain}
%\lhead[\thepage]{\thesection}      % Note the different brackets!
%\rhead[\thesection]{SEED Laboratories}
%\lfoot[\fancyplain{}{}]{Syracuse University} 
%\cfoot[\fancyplain{}{}]{\thepage} 

\newcommand{\tstamp}{\today}   
%\lhead[\fancyplain{}{\thepage}]         {\fancyplain{}{\rightmark}}
%\chead[\fancyplain{}{}]                 {\fancyplain{}{}}
%\rhead[\fancyplain{}{\rightmark}]       {\fancyplain{}{\thepage}}
%\lfoot[\fancyplain{}{}]                 {\fancyplain{\tstamp}{\tstamp}}
%\cfoot[\fancyplain{\thepage}{}]         {\fancyplain{\thepage}{}}
%\rfoot[\fancyplain{\tstamp} {\tstamp}]  {\fancyplain{}{}}

\pagestyle{fancy}
%\lhead{\bfseries Computer Security Course Project}
\lhead{\bfseries SEED Labs}
\chead{}
\rhead{\small \thepage}
\lfoot{}
\cfoot{}
\rfoot{}

\usepackage{listings}
\usepackage{color}

\definecolor{dkgreen}{rgb}{0,0.6,0}
\definecolor{gray}{rgb}{0.5,0.5,0.5}
\definecolor{mauve}{rgb}{0.58,0,0.82}

\lstset{frame=tb,
  language=C,
  aboveskip=3mm,
  belowskip=3mm,
  showstringspaces=false,
  columns=flexible,
  basicstyle={\small\ttfamily},
  numbers=none,
  numberstyle=\tiny\color{gray},
  keywordstyle=\color{blue},
  commentstyle=\color{dkgreen},
  stringstyle=\color{mauve},
  breaklines=true,
  breakatwhitespace=true,
  tabsize=3
}




\begin{document}

\begin{center}
{\LARGE Linux Capability Exploration Lab}
\end{center}

\copyrightnotice

\section{Lab Description}

The learning objective of this lab is for students
to gain first-hand experiences on the
use of capabilities to achieve the principle of least privilege.
This lab is based on POSIX 1.e capabilities, which is implemented in
recent versions of \linux kernel.

Capability based systems are sometimes promoted as an access control
strategy in contrast to the use of Access Control Lists (ACLs) or
Unix file permissions.  In practice, Linux systems typically use 
capabilities to limit program privilege rather than to control
access to named objects.  This lab focuses on the use of capabilities
to limit privilege.

\section{Lab Environment}
This lab runs in the Labtainer framework,
available at http://my.nps.edu/web/c3o/labtainers.
That site includes links to a pre-built virtual machine
that has Labtainers installed, however Labtainers can
be run on any Linux host that supports Docker containers.

From your labtainer-student directory start the lab using:
\begin{verbatim}
    start.py capabilities
\end{verbatim}
Links to this lab manual and to an empty lab report will be displayed.  If you create your lab report on a separate system,
be sure to copy it back to the specified location on your Linux system.

\vspace{0.1in}
\noindent
For this lab, you need to get familiar with the
following commands that come with {\tt libcap}:
\begin{itemize}
\item {\tt setcap}:   assign capabilities to a file.
\item {\tt getcap}:   display the capabilities that carried by a file.
\item {\tt getpcaps}: display the capabilities carried by a process.
\end{itemize}

\section{Lab Tasks}

In a capability system, when a program is executed, its corresponding
process is initialized with a list of capabilities (tokens).
When the process tries to access an object,
the operating system checks the process' capabilities,
and decides whether to grant the access or not.


\subsection{Task 1: Experiencing Capabilities}

In operating systems such a Linux, there are many privileged operations that can
only be conducted by privileged users. Examples of privileged
operations include configuring network interface cards, backing
up all the user files, shutting down the computers, etc.
Without capabilities, these operations can only be carried out
by superusers, who often have more privileges than are
needed for the intended tasks. Therefore, letting superusers
perform these privileged operations is a violation of
the {\em Least-Privilege Principle}.

Privileged operations are necessary in Linux and other Unix based 
operating systems.
All \setuid programs involve privileged operations that cannot
be performed by normal users. To allow normal users to run
these programs, \setuid programs turn normal users into
powerful users (e.g. root) temporarily, even though 
the involved privileged operations do not need all privileges 
provided to superuser.
This is dangerous: if the program is compromised, adversaries
might get the root privilege.

Capabilities divide the root privilege into a set of
distinct privileges. Each of these privileges is called
a capability. With capabilities, we do not need to be a
superuser to conduct privileged operations.
All we need is to have the capabilities that are needed for
the privileged operations. Therefore, even if a privileged program
is compromised, adversaries can only get limited power. This
way, the risks of privileged program can be reduced.

Capabilities has been implemented in \linux for quite some time,
but they could only be assigned to processes.
Since kernel version {\tt 2.6.24}, capabilities can be assigned to
files (i.e., programs) and turn those programs into privileged programs.
When a privileged program is executed, the running process will carry
those capabilities that are assigned to the program. In some sense,
this is similar to the \setuid files, but the major difference is the
amount of privileged carried by the running processes.

We will use an example to show how capabilities can be used to
remove unnecessary privileges assigned to certain privileged programs.
First, as the unprivileged ubuntu user, run the following command:
\begin{verbatim}
% ping www.google.com
\end{verbatim}

The program should run successfully. If you look at the file attributes
of the program {\tt /bin/ping}, you will see that {\tt ping}
it is a \setuid program with the owner being root, i.e., when you
execute {\tt ping}, your effective user id becomes root, and the running
process therefore runs with root privileges. If there are vulnerabilities in {\tt ping},
the entire system can be compromised. Using capabilities, we can
remove unnecessary privileges from {\tt ping}.

First, let us turn {\tt /bin/ping} into a non-\setuid
program. This can be done via the following command:
\begin{verbatim}
   sudo chmod u-s /bin/ping
\end{verbatim}

Now, run {\tt 'ping www.google.com'}, and see what happens.
The command should fail. This is because {\tt ping} needs
to open a RAW socket, which is a privileged operation that can only be
conducted by root (before capabilities are implemented). That is why
{\tt ping} has to be a \setuid program. Let us only assign the {\tt cap\_net\_raw}
capability to {\tt ping}, and see what happens:

\begin{Verbatim}[frame=single] 
   sudo setcap cap_net_raw=ep /bin/ping
   ping www.google.com
\end{Verbatim}

\subsubsection{Task 1.1 Allow unprivileged users to run tcpdump}
In addition to reducing privileges associated with setuid programs,
capabilities can be used to allow unprivileged users to run
selected programs without granting those users sudo privileges.
A common example is the \texttt{/usr/bin/tcpdump} program.  In
other Labtainer exercises, we use tcpdump to capture network
traffic, however we do that but running
\begin{Verbatim}[frame=single] 
   sudo tcpdump
\end{Verbatim}

Modify the tcpdump program so that an unprivileged user can run it.

\subsubsection{Task 1.2 Convert passwd to use capabilities}
Some setuid programs require several different capabilities
to operate.  The file: {\tt include/linux/capability.h}
describes the various capabilities available in Linux.

The \texttt{/usr/bin/passwd} program requires the following
capabilities to operate:

\begin{Verbatim}[frame=single] 
    cap_chown
    cap_dac_override
    cap_fowner
\end{Verbatim}

Modify the \texttt{passwd} program to use capabilities instead of setuid,
then demonstrate that it still works by changing the ubuntu user password
(which initially is \texttt{ubuntu}).

\textbf{Note:} When you run the passwd utility in this lab, it will display
your passwords as you type them.  Usually, passwords are not displayed.  This
is a limitation of the Labtainer environment.


\subsection{Task 2: Adjusting Privileges}

With capabilities, it is possible to dynamically adjust the amount of privileges  
a process has, which is consistent with the principle of least privilege.
For example, when a privilege is no longer needed in a process,
we should allow the process to permanently remove the capabilities
relevant to this privilege. Therefore,
even if the process is compromised, attackers will not be able to
gain these deleted capabilities.
Adjusting privileges can be achieved using the following
capability management operations.
\begin{enumerate}
  \item {\em Deleting:}  A process can permanently delete a capability.
  \item {\em Disabling:} A process can temporarily disable a capability.
        Unlike deleting, disabling is only temporary; the process
        can later enable it.
  \item {\em Enabling:} A process can enable a capability that is temporarily disabled.
        A deleted capability cannot be enabled.
\end{enumerate}

Without capabilities, a privileged \setuid program can also
delete/disable/enable its own privileged. This is done via
the {\tt setuid()} and {\tt seteuid()} system calls; namely, a process
can change its effective user id during the run time. The
granularity is quite coarse using these system calls, because you can either
be the privileged users (e.g. root) or a non-privileged users.
With capabilities, the privileges can be adjusted in a much
finer fashion, because each capability can be independently adjusted.

To support dynamic capability adjustment, \linux uses a mechanism similar
to the \setuid mechanism, i.e., a process carries three capability sets:
permitted (P), inheritable (I), and  effective (E).
The permitted set consists of the capabilities that the process is permitted
to use; however, this set of capabilities might not be active.
The effective set consists of those capabilities that the process can currently
use (this is like the effective user uid in the \setuid mechanism).
The effective set must always be a subset of the permitted set.
The process can change the contents of the effective set at any time as long as the
effective set does not exceed the permitted set. The inheritable set is used
only for calculating the new capability sets after {\tt exec()}, i.e.,
which capabilities can be inherited by the children processes.

Review the \texttt{mypriv.c} program in the home directory.  Answer the
questions embedded in the source code of the program.  Then use the
\texttt{build.sh} script to compile and link the program. Then use \texttt{setcap}
to set the \texttt{CAP\_DAC\_READ\_SEARCH} capability, and run the program.
Compare the results to what you expected.

After you finish the above task, please answer the following questions:
\begin{itemize}
\item \myquestion After a program (running as normal user)  disables a capability A,
it is compromised
by a buffer-overflow attack. The attacker successfully injects his malicious
code into this program's stack space and starts to run it. Can this attacker use
the capability A? What if the process deleted the capability, can the attacker
uses the capability?

\item \myquestion The same as the previous question, except replacing the
buffer-overflow attack with the race condition attack. Namely, if
the attacker exploits the race condition in this program, can he use the
capability A if the capability is disabled? What if the capability is
deleted?


\end{itemize}


\section{Submission}

You need to submit a detailed lab report to describe what you have
done and what you have observed; you also need to provide explanation
to the observations that are interesting or surprising. In your report,
you need to answer all the questions listed in this lab.

After finishing the lab, go to the terminal on your Linux system that was used to start the lab and type:
\begin{verbatim}
    stop.py capabilities
\end{verbatim}
When you stop the lab, the system will display a path to the zipped lab results on your Linux system.  Provide that file to
your instructor, e.g., via the Sakai site.



\end{document}
