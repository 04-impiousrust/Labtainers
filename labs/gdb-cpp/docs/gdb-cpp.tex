\documentclass[11pt]{article}

\usepackage{times}
\usepackage{epsf}
\usepackage{epsfig}
\usepackage{amsmath, alltt, amssymb, xspace}
\usepackage{wrapfig}
\usepackage{fancyhdr}
\usepackage{url}
\usepackage{verbatim}
\usepackage{fancyvrb}

\usepackage{subfigure}
\usepackage{cite}
%\usepackage{cases}
%\usepackage{ltexpprt}
%\usepackage{verbatim}

%\topmargin      -0.70in  % distance to headers
%\headheight     0.2in   % height of header box
%\headsep        0.4in   % distance to top line
%\footskip       0.3in   % distance from bottom line

% Horizontal alignment
\topmargin      -0.50in  % distance to headers
\oddsidemargin  0.0in
\evensidemargin 0.0in
\textwidth      6.5in
\textheight     8.9in 


%\centerfigcaptionstrue

%\def\baselinestretch{0.95}


\newcommand\discuss[1]{\{\textbf{Discuss:} \textit{#1}\}}
%\newcommand\todo[1]{\vspace{0.1in}\{\textbf{Todo:} \textit{#1}\}\vspace{0.1in}}
\newtheorem{problem}{Problem}[section]
%\newtheorem{theorem}{Theorem}
%\newtheorem{fact}{Fact}
\newtheorem{define}{Definition}[section]
%\newtheorem{analysis}{Analysis}
\newcommand\vspacenoindent{\vspace{0.1in} \noindent}

%\newenvironment{proof}{\noindent {\bf Proof}.}{\hspace*{\fill}~\mbox{\rule[0pt]{1.3ex}{1.3ex}}}
%\newcommand\todo[1]{\vspace{0.1in}\{\textbf{Todo:} \textit{#1}\}\vspace{0.1in}}

%\newcommand\reducespace{\vspace{-0.1in}}
% reduce the space between lines
%\def\baselinestretch{0.95}

\newcommand{\fixmefn}[1]{ \footnote{\sf\ \ \fbox{FIXME} #1} }
\newcommand{\todo}[1]{
\vspace{0.1in}
\fbox{\parbox{6in}{TODO: #1}}
\vspace{0.1in}
}

\newcommand{\mybox}[1]{
\vspace{0.2in}
\noindent
\fbox{\parbox{6.5in}{#1}}
\vspace{0.1in}
}


\newcounter{question}
\setcounter{question}{1}

\newcommand{\myquestion} {{\vspace{0.1in} \noindent \bf Question \arabic{question}:} \addtocounter{question}{1} \,}

\newcommand{\myproblem} {{\noindent \bf Problem \arabic{question}:} \addtocounter{question}{1} \,}


\newcommand{\copyrightnoticeA}[1]{
\vspace{0.1in}
\fbox{\parbox{6in}{\small Copyright \copyright\ 2006 - 2014\ \ Wenliang Du, Syracuse University.\\ 
      The development of this document is partially funded by 
      the National Science Foundation's Course, Curriculum, and Laboratory 
      Improvement (CCLI) program under Award No. 0618680 and 0231122. 
      Permission is granted to copy, distribute and/or modify this document
      under the terms of the GNU Free Documentation License, Version 1.2
      or any later version published by the Free Software Foundation.
      A copy of the license can be found at http://www.gnu.org/licenses/fdl.html.}}
\vspace{0.1in}
}


\newcommand{\copyrightnotice}[1]{
\vspace{0.1in}
\fbox{\parbox{6in}{\small Copyright \copyright\ 2006 - 2014\ \ Wenliang Du, Syracuse University.\\
      The development of this document is/was funded by three grants from
      the US National Science Foundation: Awards No. 0231122 and 0618680 from
      TUES/CCLI and  Award No. 1017771 from Trustworthy Computing.
      This lab was imported into the Labtainer framework by the Naval Postgraduate 
      School, Center for Cybersecurity and Cyber Operations under National Science 
      Foundation Award No. 1438893.
      Permission is granted to copy, distribute and/or modify this document
      under the terms of the GNU Free Documentation License, Version 1.2
      or any later version published by the Free Software Foundation.
      A copy of the license can be found at http://www.gnu.org/licenses/fdl.html.}}
\vspace{0.1in}
}

\newcommand{\copyrightnoticeB}[1]{
\vspace{0.1in}
\fbox{\parbox{6in}{\small Copyright \copyright\ 2006 - 2014\ \ Wenliang Du, Syracuse University.\\
      The development of this document is/was funded by the following grants from
      the US National Science Foundation: No. 0231122, 0618680, and 1303306.
      Permission is granted to copy, distribute and/or modify this document
      under the terms of the GNU Free Documentation License, Version 1.2
      or any later version published by the Free Software Foundation.
      A copy of the license can be found at http://www.gnu.org/licenses/fdl.html.}}
\vspace{0.1in}
}


\newcommand{\nocopyrightnotice}[1]{
\vspace{0.1in}
\fbox{\parbox{6in}{\small  
      The development of this document is funded by 
      the National Science Foundation's Course, Curriculum, and Laboratory 
      Improvement (CCLI) program under Award No. 0618680 and 0231122. 
      Permission is granted to copy, distribute and/or modify this document.
      }}
\vspace{0.1in}
}

\newcommand{\idea}[1]{
\vspace{0.1in}
{\sf IDEA:\ \ \fbox{\parbox{5in}{#1}}}
\vspace{0.1in}
}

\newcommand{\questionblock}[1]{
\vspace{0.1in}
\fbox{\parbox{6in}{#1}}
\vspace{0.1in}
}


\newcommand{\minix}{{\tt Minix}\xspace}
\newcommand{\unix}{{\tt Unix}\xspace}
\newcommand{\linux}{{\tt Linux}\xspace}
\newcommand{\ubuntu}{{\tt Ubuntu}\xspace}
\newcommand{\selinux}{{\tt SELinux}\xspace}
\newcommand{\freebsd}{{\tt FreeBSD}\xspace}
\newcommand{\solaris}{{\tt Solaris}\xspace}
\newcommand{\windowsnt}{{\tt Windows NT}\xspace}
\newcommand{\setuid}{{\tt Set-UID}\xspace}
%\newcommand{\smx}{{\tt Smx}\xspace}
\newcommand{\smx}{{\tt Minix}\xspace}
\newcommand{\relay}{{\tt relay}\xspace}
\newcommand{\isys}{{\tt iSYS}\xspace}
\newcommand{\ilan}{{\tt iLAN}\xspace}
\newcommand{\iSYS}{{\tt iSYS}\xspace}
\newcommand{\iLAN}{{\tt iLAN}\xspace}
\newcommand{\iLANs}{{\tt iLAN}s\xspace}
\newcommand{\bochs}{{\tt Bochs}\xspace}

\newcommand\FF{{\mathcal{F}}}

\newcommand{\argmax}[1]{
\begin{minipage}[t]{1.25cm}\parskip-1ex\begin{center}
argmax
#1
\end{center}\end{minipage}
\;
}

\newcommand{\bm}{\boldmath}
\newcommand  {\bx}    {\mbox{\boldmath $x$}}
\newcommand  {\by}    {\mbox{\boldmath $y$}}
\newcommand  {\br}    {\mbox{\boldmath $r$}}


%\pagestyle{fancyplain}
%\lhead[\thepage]{\thesection}      % Note the different brackets!
%\rhead[\thesection]{SEED Laboratories}
%\lfoot[\fancyplain{}{}]{Syracuse University} 
%\cfoot[\fancyplain{}{}]{\thepage} 

\newcommand{\tstamp}{\today}   
%\lhead[\fancyplain{}{\thepage}]         {\fancyplain{}{\rightmark}}
%\chead[\fancyplain{}{}]                 {\fancyplain{}{}}
%\rhead[\fancyplain{}{\rightmark}]       {\fancyplain{}{\thepage}}
%\lfoot[\fancyplain{}{}]                 {\fancyplain{\tstamp}{\tstamp}}
%\cfoot[\fancyplain{\thepage}{}]         {\fancyplain{\thepage}{}}
%\rfoot[\fancyplain{\tstamp} {\tstamp}]  {\fancyplain{}{}}

\pagestyle{fancy}
%\lhead{\bfseries Computer Security Course Project}
\lhead{\bfseries SEED Labs}
\chead{}
\rhead{\small \thepage}
\lfoot{}
\cfoot{}
\rfoot{}

\usepackage{listings}
\usepackage{color}

\definecolor{dkgreen}{rgb}{0,0.6,0}
\definecolor{gray}{rgb}{0.5,0.5,0.5}
\definecolor{mauve}{rgb}{0.58,0,0.82}

\lstset{frame=tb,
  language=C,
  aboveskip=3mm,
  belowskip=3mm,
  showstringspaces=false,
  columns=flexible,
  basicstyle={\small\ttfamily},
  numbers=none,
  numberstyle=\tiny\color{gray},
  keywordstyle=\color{blue},
  commentstyle=\color{dkgreen},
  stringstyle=\color{mauve},
  breaklines=true,
  breakatwhitespace=true,
  tabsize=3
}



\begin{document}

\begin{center}
{\LARGE Debugging a C++ program under Unix: gdb Tutorial}
\vspace{0.1in}\\
\end{center}

\section{Overview}
This tutorial was originally written for CS 342 at Washington University by Andrew Gilpin.

This tutorial is intended to help a programmer who is new to the Unix/Linux environment to get started with using the gdb debugger. This tutorial assumes you already know how to program in C++ and you can compile and execute programs. It also assumes that you basically know what a debugger is and are motivated to use one.

\section{Lab Environment}
\label{environment}
This lab runs in the Labtainer framework,
available at http://nps.edu/web/c3o/labtainers.
That site includes links to a pre-built virtual machine
that has Labtainers installed, however Labtainers can
be run on any Linux host that supports Docker containers.

From your labtainer-student directory start the lab using:
\begin{verbatim}
    labtainer gdb-cpp
\end{verbatim}
\noindent A link to this lab manual will be displayed.  
\section{Source code}
To help illustrate some of the debugging principles we will use a running example of a buggy program. As you progress through this tutorial, you will use the debugger to locate and fix errors in the code. The code and a simple makefile is located on the gdb-cpp computer that starts when you run this lab.
The code is very simple and consists of two class definitions, a node and a linked list. There is also a simple driver to test the list. All of the code was placed into a single file to make illustrating the process of debugging a little easier.

The program and makefile are on the computer created when the lab starts, and can be seen in the home directory.

\section{Background}
Debugging is something that can't be avoided. Every programmer will at one point in their programming career have to debug a section of code. There are many ways to go about debugging, from printing out messages to the screen, using a debugger, or just thinking about what the program is doing and making an educated guess as to what the problem is.
Before a bug can be fixed, the source of the bug must be located. For example, with segmentation faults, it is useful to know on which line of code the seg fault is occuring. Once the line of code in question has been found, it is useful to know about the values in that method, who called the method, and why (specifically) the error is occuring. Using a debugger makes finding all of this information very simple.

\section{Tasks}
\subsection{Build and run}
Go ahead and make the program for this tutorial, and run the program. The program will print out some messages, and then it will print that it has received a segmentation fault signal, resulting in a program crash. Given the information on the screen at this point, it is near impossible to determine why the program crashed, much less how to fix the problem. We will now begin to debug this program.

\subsection{Loading a program in gdb}
So you now have an executable file (in this case main) and you want to debug it. First you must launch the debugger. The debugger is called gdb and you can tell it which file to debug at the shell prompt. So to debug main we want to type gdb main. 

gdb is now waiting for the user to type a command. We need to run the program so that the debugger can help us see what happens when the program crashes. Type run at the (gdb) prompt. Here is what you should see when you run the command:
\begin{verbatim}
(gdb) run
Starting program: /home/ubuntu/main 
Creating Node, 1 are in existence right now
Creating Node, 2 are in existence right now
Creating Node, 3 are in existence right now
Creating Node, 4 are in existence right now
The fully created list is:
4
3
2
1

Now removing elements:
Creating Node, 5 are in existence right now
Destroying Node, 4 are in existence right now
4
3
2
1


Program received signal SIGSEGV, Segmentation fault.
0x000055555555586c in Node<int>::next (this=0x0) at main.cc:28
28	  Node<T>* next () const { return next_; }

(gdb)
\end{verbatim}
The program crashed so lets see what kind of information we can gather.
Inspecting crashes
We can see the that the program was at line 28 of main.cc, that this points to 0, and we can see the line of code that was executed. But we also want to know who called this method and we would like to be able to examine values in the calling methods. So at the gdb prompt, we type backtrace which gives the following output:

\begin{verbatim}
(gdb) backtrace 
#0  0x000055555555586c in Node<int>::next (this=0x0) at main.cc:28
#1  0x0000555555555763 in LinkedList<int>::remove (this=0x55555556aeb0, 
    item_to_remove=@0x7fffffffe43c: 1) at main.cc:77
#2  0x00005555555553b1 in main (argc=1, argv=0x7fffffffe558) at main.cc:120
(gdb)

\end{verbatim}
So in addition to what we knew about the current method and the local variables, we can now also see what methods called us and what their parameters were. For example, we can see that we were called by $LinkedList<int>::remove ()$ where the parameter item\_to\_remove is at address 0x7fffffffe43c. It may help us to understand our bug if we know the value of item\_to\_remove, so we want to see the value at the address of item\_to\_remove. This can be done using the x command using the address as a parameter. ("x" can be thought of as being short for "examine".) Here is the result of running the command:

\begin{verbatim}
(gdb) x 0x7fffffffe43c
0x7fffffffe43c:	0x00000001
(gdb)
\end{verbatim}

So the program is crashing while trying to run $LinkedList<int>::remove$ with a parameter of 1. We have now narrowed the problem down to a specific function and a specific value for the parameter.

\subsection{Conditional breakpoints}
Now that we know where and when the segfault is occuring, we want to watch what the program is doing right before it crashes. One way to do this is to step through, one at a time, every statement of the program until we get to the point of execution where we want to see what is happening. This works, but sometimes you may want to just run to a particular section of code and stop execution at that point so you can examine data at that location.
If you have ever used a debugger you are probably familiar with the concept of breakpoints. Basically, a breakpoint is a line in the source code where the debugger should break execution. In our example, we want to look at the code in $LinkedList<int>::remove()$ so we would want to set a breakpoint at line 52 of main.cc.\footnote{While gdb lets you set a break at the start of a function, e.g., break $LinkedList<int>::remove$, gdb breaks 64-bit applications
prior to execution of the function, e.g., at line 51 in our example, making such breakpoints less useful.}

\begin{verbatim}
(gdb) break 52
Breakpoint 1 at 0x29fa0: file main.cc, line 52.
(gdb)
\end{verbatim}
So now Breakpoint 1 is set at main.cc, line 52 as desired. (The reason the breakpoint gets a number is so we can refer to the breakpoint later, for example if we want to delete it.) So when the program is run, it will return control to the debugger everytime it reaches line 52. This may not be desirable if the method is called many times but only has problems with certain values that are passed. Conditional breakpoints can help us here. For our example, we know that the program crashes when $LinkedList<int>::remove()$ is called with a value of 1. So we might want to tell the debugger to only break at line 52 if item\_to\_remove is equal to 1. This can be done by issuing the following command:
\begin{verbatim}
(gdb) condition 1 item_to_remove==1
(gdb)
\end{verbatim}
This basically says "Only break at Breakpoint 1 if the value of item\_to\_remove is 1." Now we can run the program and know that the debugger will only break here when the specified condition is true.\footnote{An alternate syntax would have been to use {\tt break 52 if item\_to\_remove==1}}

\subsection{Stepping}
Continuing with the example above, we have set a conditional breakpoint and now want to go through this method one line at a time and see if we can locate the source of the error. This is accomplished using the step command. gdb has the nice feature that when enter is pressed without typing a command, the last command is automatically used. That way we can step through by simply tapping the enter key after the first step has been entered. Here is what this looks like:
\begin{verbatim}
(gdb) run
Start it from the beginning? (y or n) y
Starting program: /home/ubuntu/main 
Creating Node, 1 are in existence right now
Creating Node, 2 are in existence right now
Creating Node, 3 are in existence right now
Creating Node, 4 are in existence right now
The fully created list is:
4
3
2
1

Now removing elements:
Creating Node, 5 are in existence right now
Destroying Node, 4 are in existence right now
4
3
2
1


Breakpoint 1, LinkedList<int>::remove (this=0x55555556aeb0, 
  item_to_remove=@0x7fffffffe43c: 1)
    at main.cc:52
52	    Node<T> *marker = head_;
(gdb) step
53	    Node<T> *temp = 0;  // temp points to one behind as we iterate
(gdb) 
55	    while (marker != 0) {
(gdb) 
56	      if (marker->value() == item_to_remove) {
(gdb) 
Node<int>::value (this=0x7ffff7f1444e <std::ostream::put(char)+94>) at main.cc:30
30	  const T& value () const { return value_; }
(gdb) 
LinkedList<int>::remove (this=0x55555556aeb0, item_to_remove=@0x7fffffffe43c: 
   1) at main.cc:75
75	      marker = 0;  // reset the marker
(gdb) 
76	      temp = marker;
(gdb) 
77	      marker = marker->next();
(gdb) 
Node<int>::next (this=0x55555556b360) at main.cc:28
28	  Node<T>* next () const { return next_; }
(gdb) 

Program received signal SIGSEGV, Segmentation fault.
0x000055555555586c in Node<int>::next (this=0x0) at main.cc:28
28	  Node<T>* next () const { return next_; }

(gdb)
\end{verbatim}
After typing run, gdb asks us if we want to restart the program, which we do. It then proceeds to run and breaks at the desired location in the program. Then we type step and proceed to hit enter to step through the program. Note that the debugger steps into functions that are called. If you don't want to do this, you can use next instead of step which otherwise has the same behavior.
The error in the program is obvious. At line 75 marker is set to 0, but at line 77 a member of marker is accessed. Since the program can't access memory location 0, the seg fault occurs. In this example, nothing has to be done to marker and the error can be avoided by simply removing line 75 from main.cc.

If you look at the output from running the program, you will see first of all that the program runs without crashing, but there is a memory leak somewhere in the program. (Hint: It is in the $LinkedList<T>::remove() function$. One of the cases for remove doesn't work properly.) It is left as an exercise to the reader to use the debugger in locating and fixing this bug.

gdb can be exited by typing quit.

Further information
This document only covers the bare minimum number of commands necessary to get started using gdb. For more information about gdb see the gdb man page or take a look at a very long description of gdb here. Online help can be accessed by typing help while running gdb. 

\section{Notes}
There is another bug in the source code for the linked list that is not mentioned in the above code. The bug does not show up for the sequence of inserts and removes that are in the provided driver code, but for other sequences the bug shows up. For example, inserting 1, 2, 3, and 4, and then trying to remove 2 will show the error. The bug fix is pretty simple and is left as an exercise.

\section{Submission}
After finishing the lab, go to the terminal on your Linux system that was used to start the lab and type:
\begin{verbatim}
    stoplab 
\end{verbatim}
When you stop the lab, the system will display a path to the zipped lab results on your Linux system.  Provide that file to 
your instructor, e.g., via the Sakai site.
\end{document}
