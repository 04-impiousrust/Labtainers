\documentclass[11pt]{article}

\usepackage{times}
\usepackage{epsf}
\usepackage{epsfig}
\usepackage{amsmath, alltt, amssymb, xspace}
\usepackage{wrapfig}
\usepackage{fancyhdr}
\usepackage{url}
\usepackage{verbatim}
\usepackage{fancyvrb}

\usepackage{subfigure}
\usepackage{cite}
%\usepackage{cases}
%\usepackage{ltexpprt}
%\usepackage{verbatim}

%\topmargin      -0.70in  % distance to headers
%\headheight     0.2in   % height of header box
%\headsep        0.4in   % distance to top line
%\footskip       0.3in   % distance from bottom line

% Horizontal alignment
\topmargin      -0.50in  % distance to headers
\oddsidemargin  0.0in
\evensidemargin 0.0in
\textwidth      6.5in
\textheight     8.9in 


%\centerfigcaptionstrue

%\def\baselinestretch{0.95}


\newcommand\discuss[1]{\{\textbf{Discuss:} \textit{#1}\}}
%\newcommand\todo[1]{\vspace{0.1in}\{\textbf{Todo:} \textit{#1}\}\vspace{0.1in}}
\newtheorem{problem}{Problem}[section]
%\newtheorem{theorem}{Theorem}
%\newtheorem{fact}{Fact}
\newtheorem{define}{Definition}[section]
%\newtheorem{analysis}{Analysis}
\newcommand\vspacenoindent{\vspace{0.1in} \noindent}

%\newenvironment{proof}{\noindent {\bf Proof}.}{\hspace*{\fill}~\mbox{\rule[0pt]{1.3ex}{1.3ex}}}
%\newcommand\todo[1]{\vspace{0.1in}\{\textbf{Todo:} \textit{#1}\}\vspace{0.1in}}

%\newcommand\reducespace{\vspace{-0.1in}}
% reduce the space between lines
%\def\baselinestretch{0.95}

\newcommand{\fixmefn}[1]{ \footnote{\sf\ \ \fbox{FIXME} #1} }
\newcommand{\todo}[1]{
\vspace{0.1in}
\fbox{\parbox{6in}{TODO: #1}}
\vspace{0.1in}
}

\newcommand{\mybox}[1]{
\vspace{0.2in}
\noindent
\fbox{\parbox{6.5in}{#1}}
\vspace{0.1in}
}


\newcounter{question}
\setcounter{question}{1}

\newcommand{\myquestion} {{\vspace{0.1in} \noindent \bf Question \arabic{question}:} \addtocounter{question}{1} \,}

\newcommand{\myproblem} {{\noindent \bf Problem \arabic{question}:} \addtocounter{question}{1} \,}


\newcommand{\copyrightnoticeA}[1]{
\vspace{0.1in}
\fbox{\parbox{6in}{\small Copyright \copyright\ 2006 - 2014\ \ Wenliang Du, Syracuse University.\\ 
      The development of this document is partially funded by 
      the National Science Foundation's Course, Curriculum, and Laboratory 
      Improvement (CCLI) program under Award No. 0618680 and 0231122. 
      Permission is granted to copy, distribute and/or modify this document
      under the terms of the GNU Free Documentation License, Version 1.2
      or any later version published by the Free Software Foundation.
      A copy of the license can be found at http://www.gnu.org/licenses/fdl.html.}}
\vspace{0.1in}
}


\newcommand{\copyrightnotice}[1]{
\vspace{0.1in}
\fbox{\parbox{6in}{\small Copyright \copyright\ 2006 - 2014\ \ Wenliang Du, Syracuse University.\\
      The development of this document is/was funded by three grants from
      the US National Science Foundation: Awards No. 0231122 and 0618680 from
      TUES/CCLI and  Award No. 1017771 from Trustworthy Computing.
      This lab was imported into the Labtainer framework by the Naval Postgraduate 
      School, Center for Cybersecurity and Cyber Operations under National Science 
      Foundation Award No. 1438893.
      Permission is granted to copy, distribute and/or modify this document
      under the terms of the GNU Free Documentation License, Version 1.2
      or any later version published by the Free Software Foundation.
      A copy of the license can be found at http://www.gnu.org/licenses/fdl.html.}}
\vspace{0.1in}
}

\newcommand{\copyrightnoticeB}[1]{
\vspace{0.1in}
\fbox{\parbox{6in}{\small Copyright \copyright\ 2006 - 2014\ \ Wenliang Du, Syracuse University.\\
      The development of this document is/was funded by the following grants from
      the US National Science Foundation: No. 0231122, 0618680, and 1303306.
      Permission is granted to copy, distribute and/or modify this document
      under the terms of the GNU Free Documentation License, Version 1.2
      or any later version published by the Free Software Foundation.
      A copy of the license can be found at http://www.gnu.org/licenses/fdl.html.}}
\vspace{0.1in}
}


\newcommand{\nocopyrightnotice}[1]{
\vspace{0.1in}
\fbox{\parbox{6in}{\small  
      The development of this document is funded by 
      the National Science Foundation's Course, Curriculum, and Laboratory 
      Improvement (CCLI) program under Award No. 0618680 and 0231122. 
      Permission is granted to copy, distribute and/or modify this document.
      }}
\vspace{0.1in}
}

\newcommand{\idea}[1]{
\vspace{0.1in}
{\sf IDEA:\ \ \fbox{\parbox{5in}{#1}}}
\vspace{0.1in}
}

\newcommand{\questionblock}[1]{
\vspace{0.1in}
\fbox{\parbox{6in}{#1}}
\vspace{0.1in}
}


\newcommand{\minix}{{\tt Minix}\xspace}
\newcommand{\unix}{{\tt Unix}\xspace}
\newcommand{\linux}{{\tt Linux}\xspace}
\newcommand{\ubuntu}{{\tt Ubuntu}\xspace}
\newcommand{\selinux}{{\tt SELinux}\xspace}
\newcommand{\freebsd}{{\tt FreeBSD}\xspace}
\newcommand{\solaris}{{\tt Solaris}\xspace}
\newcommand{\windowsnt}{{\tt Windows NT}\xspace}
\newcommand{\setuid}{{\tt Set-UID}\xspace}
%\newcommand{\smx}{{\tt Smx}\xspace}
\newcommand{\smx}{{\tt Minix}\xspace}
\newcommand{\relay}{{\tt relay}\xspace}
\newcommand{\isys}{{\tt iSYS}\xspace}
\newcommand{\ilan}{{\tt iLAN}\xspace}
\newcommand{\iSYS}{{\tt iSYS}\xspace}
\newcommand{\iLAN}{{\tt iLAN}\xspace}
\newcommand{\iLANs}{{\tt iLAN}s\xspace}
\newcommand{\bochs}{{\tt Bochs}\xspace}

\newcommand\FF{{\mathcal{F}}}

\newcommand{\argmax}[1]{
\begin{minipage}[t]{1.25cm}\parskip-1ex\begin{center}
argmax
#1
\end{center}\end{minipage}
\;
}

\newcommand{\bm}{\boldmath}
\newcommand  {\bx}    {\mbox{\boldmath $x$}}
\newcommand  {\by}    {\mbox{\boldmath $y$}}
\newcommand  {\br}    {\mbox{\boldmath $r$}}


%\pagestyle{fancyplain}
%\lhead[\thepage]{\thesection}      % Note the different brackets!
%\rhead[\thesection]{SEED Laboratories}
%\lfoot[\fancyplain{}{}]{Syracuse University} 
%\cfoot[\fancyplain{}{}]{\thepage} 

\newcommand{\tstamp}{\today}   
%\lhead[\fancyplain{}{\thepage}]         {\fancyplain{}{\rightmark}}
%\chead[\fancyplain{}{}]                 {\fancyplain{}{}}
%\rhead[\fancyplain{}{\rightmark}]       {\fancyplain{}{\thepage}}
%\lfoot[\fancyplain{}{}]                 {\fancyplain{\tstamp}{\tstamp}}
%\cfoot[\fancyplain{\thepage}{}]         {\fancyplain{\thepage}{}}
%\rfoot[\fancyplain{\tstamp} {\tstamp}]  {\fancyplain{}{}}

\pagestyle{fancy}
%\lhead{\bfseries Computer Security Course Project}
\lhead{\bfseries SEED Labs}
\chead{}
\rhead{\small \thepage}
\lfoot{}
\cfoot{}
\rfoot{}

\usepackage{listings}
\usepackage{color}

\definecolor{dkgreen}{rgb}{0,0.6,0}
\definecolor{gray}{rgb}{0.5,0.5,0.5}
\definecolor{mauve}{rgb}{0.58,0,0.82}

\lstset{frame=tb,
  language=C,
  aboveskip=3mm,
  belowskip=3mm,
  showstringspaces=false,
  columns=flexible,
  basicstyle={\small\ttfamily},
  numbers=none,
  numberstyle=\tiny\color{gray},
  keywordstyle=\color{blue},
  commentstyle=\color{dkgreen},
  stringstyle=\color{mauve},
  breaklines=true,
  breakatwhitespace=true,
  tabsize=3
}



\begin{document}

\begin{center}
{\LARGE Wireshark Introduction}
\vspace{0.1in}\\
\end{center}


\section{Overview}
This exercise introduces the the Wireshark network traffic analysis tool.
The student will use Wireshark to view network traffic captured in a ``PCAP''
file and locate a specific packet.  PCAP files contain
copies of network traffic stored in a format that can be processed
by various network analysis tools such as Wireshark and
tcpdump.  PCAP is short for ``packet capture''.

\subsection{Background}
This exercise assumes the student has received instruction TCP/IP networking.

This lab exercise only touches on some of the most basic features of Wireshark.
Details on using the tool can be found at \url{https://www.wireshark.org/docs/wsug\_html\_chunked/ChapterIntroduction.html}

\section{Lab Environment}
This lab runs in the Labtainer framework,
available at http://nps.edu/web/c3o/labtainers.
That site includes links to a pre-built virtual machine
that has Labtainers installed, however Labtainers can
be run on any Linux host that supports Docker containers.

From your labtainer-student directory start the lab using:
\begin{verbatim}
    labtainer wireshark
\end{verbatim}
A link to this lab manual will be displayed.

\section{Tasks}
\subsection{Explore}
Use the {\tt ls} command to view the content of the directory in the terminal that opened when you started the lab.
That {\tt telnet.pcapng} file contains the network traffic you will analyze.  Use
\begin{verbatim}
   file telnet.pcapng
\end{verbatim}
\noindent to view information about the file.  Note it is a \textit{pcap-ng} file, which is a variation of the PCAP file format
currently used as the default format for Wireshark and other analysis tools.

\subsection{Run wireshark to perform PCAP Analysis}

Start Wireshark using the {\tt wireshark} command.  Then use {\tt File->Open} to open the telnet.pcapng file.

Telnet is a communications protocol that allows a user to 
issue shell commands to a remote host. Telnet network 
traffic is not encrypted. Refer to the telnet lab for further background.

\subsection{Find a specific packet}

Locate the single packet which contains the password provided when the user attempted to use Telnet to login as the "john" user. 


\textbf{Hint}: If you type {\tt telnet.data} into the field that says ``Add a display filter'' (above the list of packets), the
tool will display only Telnet data packets.  Press {\tt return} to apply the filter.

Once you locate the single packet containing the invalid password, use {\tt File=>Export specified packets} to save the single
packet that you located.  Save the single packet as 'invalidpassword.pcapng'.  Be sure to select {\tt Selected packets only} in the Export dialog and be sure to get the file name exactly right.

After you save the packet, you might then use {\tt File=>Open} to open your new pcap file to confirm it contains the correct packet.

\section{Submission}
After finishing the lab, go to the terminal on your Linux system that was used to start the lab and type:
\begin{verbatim}
    stoplab 
\end{verbatim}
When you stop the lab, the system will display a path to the zipped lab results on your Linux system.  Provide that file to 
your instructor, e.g., via the Sakai site.

\copyrightnotice

\end{document}
