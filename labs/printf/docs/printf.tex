\documentclass[11pt]{article}

\usepackage{times}
\usepackage{epsf}
\usepackage{epsfig}
\usepackage{amsmath, alltt, amssymb, xspace}
\usepackage{wrapfig}
\usepackage{fancyhdr}
\usepackage{url}
\usepackage{verbatim}
\usepackage{fancyvrb}

\usepackage{subfigure}
\usepackage{cite}
%\usepackage{cases}
%\usepackage{ltexpprt}
%\usepackage{verbatim}

%\topmargin      -0.70in  % distance to headers
%\headheight     0.2in   % height of header box
%\headsep        0.4in   % distance to top line
%\footskip       0.3in   % distance from bottom line

% Horizontal alignment
\topmargin      -0.50in  % distance to headers
\oddsidemargin  0.0in
\evensidemargin 0.0in
\textwidth      6.5in
\textheight     8.9in 


%\centerfigcaptionstrue

%\def\baselinestretch{0.95}


\newcommand\discuss[1]{\{\textbf{Discuss:} \textit{#1}\}}
%\newcommand\todo[1]{\vspace{0.1in}\{\textbf{Todo:} \textit{#1}\}\vspace{0.1in}}
\newtheorem{problem}{Problem}[section]
%\newtheorem{theorem}{Theorem}
%\newtheorem{fact}{Fact}
\newtheorem{define}{Definition}[section]
%\newtheorem{analysis}{Analysis}
\newcommand\vspacenoindent{\vspace{0.1in} \noindent}

%\newenvironment{proof}{\noindent {\bf Proof}.}{\hspace*{\fill}~\mbox{\rule[0pt]{1.3ex}{1.3ex}}}
%\newcommand\todo[1]{\vspace{0.1in}\{\textbf{Todo:} \textit{#1}\}\vspace{0.1in}}

%\newcommand\reducespace{\vspace{-0.1in}}
% reduce the space between lines
%\def\baselinestretch{0.95}

\newcommand{\fixmefn}[1]{ \footnote{\sf\ \ \fbox{FIXME} #1} }
\newcommand{\todo}[1]{
\vspace{0.1in}
\fbox{\parbox{6in}{TODO: #1}}
\vspace{0.1in}
}

\newcommand{\mybox}[1]{
\vspace{0.2in}
\noindent
\fbox{\parbox{6.5in}{#1}}
\vspace{0.1in}
}


\newcounter{question}
\setcounter{question}{1}

\newcommand{\myquestion} {{\vspace{0.1in} \noindent \bf Question \arabic{question}:} \addtocounter{question}{1} \,}

\newcommand{\myproblem} {{\noindent \bf Problem \arabic{question}:} \addtocounter{question}{1} \,}


\newcommand{\copyrightnoticeA}[1]{
\vspace{0.1in}
\fbox{\parbox{6in}{\small Copyright \copyright\ 2006 - 2014\ \ Wenliang Du, Syracuse University.\\ 
      The development of this document is partially funded by 
      the National Science Foundation's Course, Curriculum, and Laboratory 
      Improvement (CCLI) program under Award No. 0618680 and 0231122. 
      Permission is granted to copy, distribute and/or modify this document
      under the terms of the GNU Free Documentation License, Version 1.2
      or any later version published by the Free Software Foundation.
      A copy of the license can be found at http://www.gnu.org/licenses/fdl.html.}}
\vspace{0.1in}
}


\newcommand{\copyrightnotice}[1]{
\vspace{0.1in}
\fbox{\parbox{6in}{\small Copyright \copyright\ 2006 - 2014\ \ Wenliang Du, Syracuse University.\\
      The development of this document is/was funded by three grants from
      the US National Science Foundation: Awards No. 0231122 and 0618680 from
      TUES/CCLI and  Award No. 1017771 from Trustworthy Computing.
      This lab was imported into the Labtainer framework by the Naval Postgraduate 
      School, Center for Cybersecurity and Cyber Operations under National Science 
      Foundation Award No. 1438893.
      Permission is granted to copy, distribute and/or modify this document
      under the terms of the GNU Free Documentation License, Version 1.2
      or any later version published by the Free Software Foundation.
      A copy of the license can be found at http://www.gnu.org/licenses/fdl.html.}}
\vspace{0.1in}
}

\newcommand{\copyrightnoticeB}[1]{
\vspace{0.1in}
\fbox{\parbox{6in}{\small Copyright \copyright\ 2006 - 2014\ \ Wenliang Du, Syracuse University.\\
      The development of this document is/was funded by the following grants from
      the US National Science Foundation: No. 0231122, 0618680, and 1303306.
      Permission is granted to copy, distribute and/or modify this document
      under the terms of the GNU Free Documentation License, Version 1.2
      or any later version published by the Free Software Foundation.
      A copy of the license can be found at http://www.gnu.org/licenses/fdl.html.}}
\vspace{0.1in}
}


\newcommand{\nocopyrightnotice}[1]{
\vspace{0.1in}
\fbox{\parbox{6in}{\small  
      The development of this document is funded by 
      the National Science Foundation's Course, Curriculum, and Laboratory 
      Improvement (CCLI) program under Award No. 0618680 and 0231122. 
      Permission is granted to copy, distribute and/or modify this document.
      }}
\vspace{0.1in}
}

\newcommand{\idea}[1]{
\vspace{0.1in}
{\sf IDEA:\ \ \fbox{\parbox{5in}{#1}}}
\vspace{0.1in}
}

\newcommand{\questionblock}[1]{
\vspace{0.1in}
\fbox{\parbox{6in}{#1}}
\vspace{0.1in}
}


\newcommand{\minix}{{\tt Minix}\xspace}
\newcommand{\unix}{{\tt Unix}\xspace}
\newcommand{\linux}{{\tt Linux}\xspace}
\newcommand{\ubuntu}{{\tt Ubuntu}\xspace}
\newcommand{\selinux}{{\tt SELinux}\xspace}
\newcommand{\freebsd}{{\tt FreeBSD}\xspace}
\newcommand{\solaris}{{\tt Solaris}\xspace}
\newcommand{\windowsnt}{{\tt Windows NT}\xspace}
\newcommand{\setuid}{{\tt Set-UID}\xspace}
%\newcommand{\smx}{{\tt Smx}\xspace}
\newcommand{\smx}{{\tt Minix}\xspace}
\newcommand{\relay}{{\tt relay}\xspace}
\newcommand{\isys}{{\tt iSYS}\xspace}
\newcommand{\ilan}{{\tt iLAN}\xspace}
\newcommand{\iSYS}{{\tt iSYS}\xspace}
\newcommand{\iLAN}{{\tt iLAN}\xspace}
\newcommand{\iLANs}{{\tt iLAN}s\xspace}
\newcommand{\bochs}{{\tt Bochs}\xspace}

\newcommand\FF{{\mathcal{F}}}

\newcommand{\argmax}[1]{
\begin{minipage}[t]{1.25cm}\parskip-1ex\begin{center}
argmax
#1
\end{center}\end{minipage}
\;
}

\newcommand{\bm}{\boldmath}
\newcommand  {\bx}    {\mbox{\boldmath $x$}}
\newcommand  {\by}    {\mbox{\boldmath $y$}}
\newcommand  {\br}    {\mbox{\boldmath $r$}}


%\pagestyle{fancyplain}
%\lhead[\thepage]{\thesection}      % Note the different brackets!
%\rhead[\thesection]{SEED Laboratories}
%\lfoot[\fancyplain{}{}]{Syracuse University} 
%\cfoot[\fancyplain{}{}]{\thepage} 

\newcommand{\tstamp}{\today}   
%\lhead[\fancyplain{}{\thepage}]         {\fancyplain{}{\rightmark}}
%\chead[\fancyplain{}{}]                 {\fancyplain{}{}}
%\rhead[\fancyplain{}{\rightmark}]       {\fancyplain{}{\thepage}}
%\lfoot[\fancyplain{}{}]                 {\fancyplain{\tstamp}{\tstamp}}
%\cfoot[\fancyplain{\thepage}{}]         {\fancyplain{\thepage}{}}
%\rfoot[\fancyplain{\tstamp} {\tstamp}]  {\fancyplain{}{}}

\pagestyle{fancy}
%\lhead{\bfseries Computer Security Course Project}
\lhead{\bfseries SEED Labs}
\chead{}
\rhead{\small \thepage}
\lfoot{}
\cfoot{}
\rfoot{}

\usepackage{listings}
\usepackage{color}

\definecolor{dkgreen}{rgb}{0,0.6,0}
\definecolor{gray}{rgb}{0.5,0.5,0.5}
\definecolor{mauve}{rgb}{0.58,0,0.82}

\lstset{frame=tb,
  language=C,
  aboveskip=3mm,
  belowskip=3mm,
  showstringspaces=false,
  columns=flexible,
  basicstyle={\small\ttfamily},
  numbers=none,
  numberstyle=\tiny\color{gray},
  keywordstyle=\color{blue},
  commentstyle=\color{dkgreen},
  stringstyle=\color{mauve},
  breaklines=true,
  breakatwhitespace=true,
  tabsize=3
}



\begin{document}

\begin{center}
{\LARGE Introduction to printf memory refererences}
\vspace{0.1in}\\
\end{center}


\section{Overview}
This exercise introduces the printf function and 
encourages the student to explore the manner in which the
function references memory addresses in response to its
given format specification.
This lab provides an introduction to techniques that are used
in the more advanced printf labs (formatstring and format64).

\subsection{Background}
This exercise assumes the student has some basic C language programming
experience and is somewhat familiar with the use of gdb\footnote{This lab
manual provides detailed gdb commands to accomplished the prescribed tasks,
and can serve as an introduction to gdb.}

No coding is required in this lab, but it will help if the student
can understand a simple C program.
The gdb program is used to explore the executing program, including viewing
a bit of its disassembly.  Some assembly language background 
would be helpful in performing the lab, but is not necessary.

\section{Lab Environment}
This lab runs in the Labtainer framework,
available at http://nps.edu/web/c3o/labtainers.
That site includes links to a pre-built virtual machine
that has Labtainers installed, however Labtainers can
be run on any Linux host that supports Docker containers
or on Docker Desktop on PCs and Macs.

From your labtainer-student directory start the lab using:
\begin{verbatim}
    labtainer printf
\end{verbatim}
A link to this lab manual will be displayed.


\section{Tasks}

\subsection{Review the printTest.c program}
A terminal opens when you start the lab.  At that terminal, view the printTest.c program.  Use either {\tt vi} or {\tt nano}, or just type {\tt less printTest.c}.

Observe the syntax of the first {\tt printf} statement.  The first parameter
is a format string that contains literal text to be displayed, and one or more
one or more \textit{conversion specifications} that determine how any remaining
parameters are displayed.  The conversion specification begins with the {\tt \%} symbol.  In the first printf statement, the conversion specification is
a {\tt \%d}, which directs printf to display the parameter as an integer.
Thus, the value of {\tt var1} would be displayed as an integer following the string "{\tt var1 is: }".  The {\tt \symbol{92}n} ``escape n'' sequence causes printf to
generate a newline.

The second printf statement illustrates how we can display the values of
multiple parameters.  In this case, the hexadecimal representation of an integer (the {\tt \%x}) followed by a string (using the 
{\tt \%s} conversion specification).

The printf function has an extremely rich set  of conversion specifications, but most those are not important for this lab.  What \textbf{is} important for this lab is the manner in which printf references memory to find the values to be displayed.

The third printf statement is vulnerable to mischief, as we will see 
in this lab.

\subsection{Run printTest}
The {\tt mkit.sh} script will compile the program as a 32-bit executable:
\begin{verbatim}
   ./mkit
\end{verbatim}
\noindent You may then run the program:
\begin{verbatim}
   ./printTest
\end{verbatim}
\noindent and observce its output.
\subsection{x86 function calling conventions}
When a 32-bit x86 program is about to call a function, the parameters to the
function are first pushed onto the stack.  The function is called and
the function references its parameters from the stack.  In the first printf, there
are two parameters: the format string and the {\tt var1} variable. 

Since in the 32-bit x86 the
stack pointer register {\tt esp} decreases as the stack ''grows``, the
figure 1 diagram has low memory at the top of the diagram.


\begin{verbatim}
low memory

        [stuff used by printf]

esp ->  pointer to the format string
        var1 value

        [stuff from calling function]
high memory

       Figure 1: Stack prior to call to printf
\end{verbatim}

In figure 1, we see the var1 value has been pushed on the
stack, followed by the pointer of the format string.  

\subsection{Behavior of printf}
When the printf function is called, it expects to find the pointer to
the format string at the top of the parameters on the stack.
It then reads the format string and interprets the conversion specifications.
In the case of our first printf, it only sees the {\tt \%d}, which causes
printf to treat the next parameter on the stack as an integer, and display its
value as such along with the rest of the format string literals.

The second printf function call will have three parameters.  This time, the printf function sees a {\tt \%x} conversion 
specification and looks at the next parameter,
which is now the {\tt var2} value and it displays that as a hexadecimal value per the {\tt \%x}.  It then sees the {\tt \%s} and treats the
next parameter as a pointer to some string, which it then displays.

\subsection{Observe calling conventions with gdb}
Run the  program in gdb:
\begin{verbatim}
   gdb printTest
\end{verbatim}
\noindent List the program with the {\tt list} command at set a breakpoint at the line of 
the first printf statement and run:
\begin{verbatim}
   break <line number>
   run
\end{verbatim}
\noindent The program will break just before the call to printf.  But not close enough for our
purposes, so we will view the disassembly of the machine instructions so that we can advance
execution to just before the actual call.  Use this gdb command to display the disassembly of
the current instruction:
\begin{verbatim}
   display/i $pc
\end{verbatim}
\noindent Then use the {\tt nexti} instruction to advance execution to the next instruction.
Repeatedly press the Return key to keep stepping until you reach the call to {\tt printf@plt}
Now the program is really just about to call printf.  Look at twenty words on the stack
as hexidecimal values:
\begin{verbatim}
   x/20xw $esp
\end{verbatim}
The {\tt esp} register is pointing to the top of the stack, which contains the first parameter
to printf, i.e., the pointer to the
format string.  Confirm that by examining memory at that address (i.e., the first displayed word)
as a string:\footnote{cut/paste by highlighting the desired text and pressing {\tt ctl shift c}
and then paste that with {\tt ctl shift v}}
\begin{verbatim}
   x/s <address>
\end{verbatim}
\noindent You should see the format string.  The word at the next parameter on the stack is our {\tt val1} value of
13 (hex 0x0d).

Look at the content of subsequent addresses.  You see some address values and such, but a bit further in
you will see the two values of {\tt var1} and {\tt var2} within adjacent words.  That memory is where the printTest
program has stored those two values.  You previously observed a copy of the {\tt var1} value near the top
of the stack.  The values at the higher addresses are the original values of those variables.

Our next step will be to fool printf into displaying those values from their original locations.

If you'd like to review what you've seen a bit more, set a breakpoint at the 2nd printf, step through
its disassembly until that call, and look at the stack to identify the three parameters to printf.

\subsection{User input in format strings}
Look at the source code of the testPrint.c program again, and find the line that reads:
\begin{verbatim}
    printf(user_input);
\end{verbatim}
\noindent In this case, the format string is supplied by the user, and there are no other parameters to be
displayed.  What if the user supplies a format string that contains conversion specifications?  The printf
function has no way of knowing the providence of the format string, nor does it have any way of knowing the
number of parameters provided in the function call -- it simply assumes parameters have been pushed onto the
stack.  Thus, if printf encounters a {\tt \%x} in the format string, it will look at the next parameter on the
stack, and since there were no other parameters, it will find whatever happened to be at that address.
Lets expand our repertoire of conversion specifications to include:
\begin{verbatim}
    %8x
\end{verbatim}
\noindent which directs printf to display the word as an 8 digit hexadecimal value.  We'll combine a raft of
those format conversions and provide that as input when the program prompts us for a string
\begin{verbatim}
AAAA%8x.%8x.%8x.%8x.%8x.%8x.%8x.%8x.%8x.%8x.%8x.%8x.%8x.%8x.%8x.%8x.
\end{verbatim}
Run the program (without gdb) and provide the above string as input.  Where do the displayed values come from?
Run the program in gdb again, this time set a break at line number of the vulnerable call to printf and use {\tt run}
to start the program.  Before the program reaches your breakpoint, it will primpt you to enter the string.
Paste the above string and the program will then break at the (almost) call to printf.  Use
\begin{verbatim}
   display/i $pc
   nexti
   <return>....
\end{verbatim}
\noindent to step to the call to {\tt printf@plt} and then display the stack content.
\begin{verbatim}
   x/20x2 $esp
\end{verbatim}
\noindent  Find the first (and only) parameter to the printf statement and confirm it is the address of your user-provided format string:
\begin{verbatim}
   x/s <address>
\end{verbatim}
\noindent The use the {\tt c} command to continue, allowing the program to output the results of the printf statement.
Compare that output to what you see in memory just past the address of the format string.\footnote{You may notice the content of memory
changes between each run of the program.  This is due to Address Space Layout Randomization.  Google it.}

\subsection{More detail}
See the formatstring lab to further explore printf vulnerabilities, including a method for modifying the content of memory.


\section{Submission}
After finishing the lab, go to the terminal on your Linux system that was used to start the lab and type:
\begin{verbatim}
    stoplab 
\end{verbatim}
When you stop the lab, the system will display a path to the zipped lab results on your Linux system.  Provide that file to 
your instructor, e.g., via the Sakai site.

\copyrightnotice

\end{document}
