%\section{Lab Environment}

This lab runs in the Labtainer framework,
available at http://my.nps.edu/web/c3o/labtainers.
That site includes links to a pre-built virtual machine
that has Labtainers installed, however Labtainers can
be run on any Linux host that supports Docker containers.

From your labtainer-student directory start the lab using:
\begin{verbatim}
    labtainer webtrack
\end{verbatim}
Links to this lab manual and to an empty lab report will be displayed.  If you create your lab report on a separate system,
be sure to copy it back to the specified location on your Linux system.

\subsection{Environment Configuration}
This lab includes two networked computers, one running a the browser
and the other hosting each of the websites used in the lab. The computer
hosting websites runs the apache server, and each site is allocated its 
own resources, as if each site ran on an independent web server.
The Firefox browser includes the  \texttt{Web Developer / Network} tools for 
to inspect the HTTP requests and responses.  


\paragraph{Starting the Apache Server.}
The Apache web server will be running when the lab
commences.  If you need to restart the web server, use
the following command:
\begin{verbatim}
   % sudo systemctl restart httpd
\end{verbatim}

\paragraph{The {\tt Elgg} Web Application.}
We use an open-source web application called {\tt Elgg} in this lab.
{\tt Elgg} is a web-based social-networking application. 
It is already set up in on the vuln-server.
We have also created several user accounts on the {\tt Elgg} server and the credentials are given below.


\vspace{0.1in}
\begin{tabular}{|l|l|l|}
\hline
User 	& UserName 	& Password\\
\hline
Admin 	& admin 	& seedelgg \\
Alice 	& alice 	& seedalice \\
Boby 	& boby 		& seedboby \\
Charlie & charlie 	& seedcharlie \\
Samy 	& samy 		& seedsamy \\
\hline
\end{tabular}
\vspace{0.1in}


\paragraph{Configuring DNS.}
We have configured the following \urlorurls needed for this lab: 
