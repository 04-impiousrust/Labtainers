\documentclass[11pt]{article}

\usepackage{times}
\usepackage{epsf}
\usepackage{epsfig}
\usepackage{amsmath, alltt, amssymb, xspace}
\usepackage{wrapfig}
\usepackage{fancyhdr}
\usepackage{url}
\usepackage{verbatim}
\usepackage{fancyvrb}

\usepackage{subfigure}
\usepackage{cite}
%\usepackage{cases}
%\usepackage{ltexpprt}
%\usepackage{verbatim}

%\topmargin      -0.70in  % distance to headers
%\headheight     0.2in   % height of header box
%\headsep        0.4in   % distance to top line
%\footskip       0.3in   % distance from bottom line

% Horizontal alignment
\topmargin      -0.50in  % distance to headers
\oddsidemargin  0.0in
\evensidemargin 0.0in
\textwidth      6.5in
\textheight     8.9in 


%\centerfigcaptionstrue

%\def\baselinestretch{0.95}


\newcommand\discuss[1]{\{\textbf{Discuss:} \textit{#1}\}}
%\newcommand\todo[1]{\vspace{0.1in}\{\textbf{Todo:} \textit{#1}\}\vspace{0.1in}}
\newtheorem{problem}{Problem}[section]
%\newtheorem{theorem}{Theorem}
%\newtheorem{fact}{Fact}
\newtheorem{define}{Definition}[section]
%\newtheorem{analysis}{Analysis}
\newcommand\vspacenoindent{\vspace{0.1in} \noindent}

%\newenvironment{proof}{\noindent {\bf Proof}.}{\hspace*{\fill}~\mbox{\rule[0pt]{1.3ex}{1.3ex}}}
%\newcommand\todo[1]{\vspace{0.1in}\{\textbf{Todo:} \textit{#1}\}\vspace{0.1in}}

%\newcommand\reducespace{\vspace{-0.1in}}
% reduce the space between lines
%\def\baselinestretch{0.95}

\newcommand{\fixmefn}[1]{ \footnote{\sf\ \ \fbox{FIXME} #1} }
\newcommand{\todo}[1]{
\vspace{0.1in}
\fbox{\parbox{6in}{TODO: #1}}
\vspace{0.1in}
}

\newcommand{\mybox}[1]{
\vspace{0.2in}
\noindent
\fbox{\parbox{6.5in}{#1}}
\vspace{0.1in}
}


\newcounter{question}
\setcounter{question}{1}

\newcommand{\myquestion} {{\vspace{0.1in} \noindent \bf Question \arabic{question}:} \addtocounter{question}{1} \,}

\newcommand{\myproblem} {{\noindent \bf Problem \arabic{question}:} \addtocounter{question}{1} \,}


\newcommand{\copyrightnoticeA}[1]{
\vspace{0.1in}
\fbox{\parbox{6in}{\small Copyright \copyright\ 2006 - 2014\ \ Wenliang Du, Syracuse University.\\ 
      The development of this document is partially funded by 
      the National Science Foundation's Course, Curriculum, and Laboratory 
      Improvement (CCLI) program under Award No. 0618680 and 0231122. 
      Permission is granted to copy, distribute and/or modify this document
      under the terms of the GNU Free Documentation License, Version 1.2
      or any later version published by the Free Software Foundation.
      A copy of the license can be found at http://www.gnu.org/licenses/fdl.html.}}
\vspace{0.1in}
}


\newcommand{\copyrightnotice}[1]{
\vspace{0.1in}
\fbox{\parbox{6in}{\small Copyright \copyright\ 2006 - 2014\ \ Wenliang Du, Syracuse University.\\
      The development of this document is/was funded by three grants from
      the US National Science Foundation: Awards No. 0231122 and 0618680 from
      TUES/CCLI and  Award No. 1017771 from Trustworthy Computing.
      This lab was imported into the Labtainer framework by the Naval Postgraduate 
      School, Center for Cybersecurity and Cyber Operations under National Science 
      Foundation Award No. 1438893.
      Permission is granted to copy, distribute and/or modify this document
      under the terms of the GNU Free Documentation License, Version 1.2
      or any later version published by the Free Software Foundation.
      A copy of the license can be found at http://www.gnu.org/licenses/fdl.html.}}
\vspace{0.1in}
}

\newcommand{\copyrightnoticeB}[1]{
\vspace{0.1in}
\fbox{\parbox{6in}{\small Copyright \copyright\ 2006 - 2014\ \ Wenliang Du, Syracuse University.\\
      The development of this document is/was funded by the following grants from
      the US National Science Foundation: No. 0231122, 0618680, and 1303306.
      Permission is granted to copy, distribute and/or modify this document
      under the terms of the GNU Free Documentation License, Version 1.2
      or any later version published by the Free Software Foundation.
      A copy of the license can be found at http://www.gnu.org/licenses/fdl.html.}}
\vspace{0.1in}
}


\newcommand{\nocopyrightnotice}[1]{
\vspace{0.1in}
\fbox{\parbox{6in}{\small  
      The development of this document is funded by 
      the National Science Foundation's Course, Curriculum, and Laboratory 
      Improvement (CCLI) program under Award No. 0618680 and 0231122. 
      Permission is granted to copy, distribute and/or modify this document.
      }}
\vspace{0.1in}
}

\newcommand{\idea}[1]{
\vspace{0.1in}
{\sf IDEA:\ \ \fbox{\parbox{5in}{#1}}}
\vspace{0.1in}
}

\newcommand{\questionblock}[1]{
\vspace{0.1in}
\fbox{\parbox{6in}{#1}}
\vspace{0.1in}
}


\newcommand{\minix}{{\tt Minix}\xspace}
\newcommand{\unix}{{\tt Unix}\xspace}
\newcommand{\linux}{{\tt Linux}\xspace}
\newcommand{\ubuntu}{{\tt Ubuntu}\xspace}
\newcommand{\selinux}{{\tt SELinux}\xspace}
\newcommand{\freebsd}{{\tt FreeBSD}\xspace}
\newcommand{\solaris}{{\tt Solaris}\xspace}
\newcommand{\windowsnt}{{\tt Windows NT}\xspace}
\newcommand{\setuid}{{\tt Set-UID}\xspace}
%\newcommand{\smx}{{\tt Smx}\xspace}
\newcommand{\smx}{{\tt Minix}\xspace}
\newcommand{\relay}{{\tt relay}\xspace}
\newcommand{\isys}{{\tt iSYS}\xspace}
\newcommand{\ilan}{{\tt iLAN}\xspace}
\newcommand{\iSYS}{{\tt iSYS}\xspace}
\newcommand{\iLAN}{{\tt iLAN}\xspace}
\newcommand{\iLANs}{{\tt iLAN}s\xspace}
\newcommand{\bochs}{{\tt Bochs}\xspace}

\newcommand\FF{{\mathcal{F}}}

\newcommand{\argmax}[1]{
\begin{minipage}[t]{1.25cm}\parskip-1ex\begin{center}
argmax
#1
\end{center}\end{minipage}
\;
}

\newcommand{\bm}{\boldmath}
\newcommand  {\bx}    {\mbox{\boldmath $x$}}
\newcommand  {\by}    {\mbox{\boldmath $y$}}
\newcommand  {\br}    {\mbox{\boldmath $r$}}


%\pagestyle{fancyplain}
%\lhead[\thepage]{\thesection}      % Note the different brackets!
%\rhead[\thesection]{SEED Laboratories}
%\lfoot[\fancyplain{}{}]{Syracuse University} 
%\cfoot[\fancyplain{}{}]{\thepage} 

\newcommand{\tstamp}{\today}   
%\lhead[\fancyplain{}{\thepage}]         {\fancyplain{}{\rightmark}}
%\chead[\fancyplain{}{}]                 {\fancyplain{}{}}
%\rhead[\fancyplain{}{\rightmark}]       {\fancyplain{}{\thepage}}
%\lfoot[\fancyplain{}{}]                 {\fancyplain{\tstamp}{\tstamp}}
%\cfoot[\fancyplain{\thepage}{}]         {\fancyplain{\thepage}{}}
%\rfoot[\fancyplain{\tstamp} {\tstamp}]  {\fancyplain{}{}}

\pagestyle{fancy}
%\lhead{\bfseries Computer Security Course Project}
\lhead{\bfseries SEED Labs}
\chead{}
\rhead{\small \thepage}
\lfoot{}
\cfoot{}
\rfoot{}

\usepackage{listings}
\usepackage{color}

\definecolor{dkgreen}{rgb}{0,0.6,0}
\definecolor{gray}{rgb}{0.5,0.5,0.5}
\definecolor{mauve}{rgb}{0.58,0,0.82}

\lstset{frame=tb,
  language=C,
  aboveskip=3mm,
  belowskip=3mm,
  showstringspaces=false,
  columns=flexible,
  basicstyle={\small\ttfamily},
  numbers=none,
  numberstyle=\tiny\color{gray},
  keywordstyle=\color{blue},
  commentstyle=\color{dkgreen},
  stringstyle=\color{mauve},
  breaklines=true,
  breakatwhitespace=true,
  tabsize=3
}




%\documentclass{article} 
%\usepackage{graphicx}
%\usepackage{color}
%\usepackage[latin1]{inputenc}
%\usepackage{lgrind}
%\input {highlight.sty}

\lhead{\bfseries SEED Labs -- Remote DNS Cache Poisoning Attack Lab}


\def \code#1 {\fbox{\scriptsize{\texttt{#1}}}}

\begin{document}

\begin{center}
{\LARGE Remote DNS Cache Poisoning Attack Lab}
\end{center}

\copyrightnotice

\section{Lab Overview}

The objective of this lab is for students to gain the first-hand experience
on the remote DNS cache poisoning attack, also called the Kaminsky
DNS attack~\cite{Kaminsky}. DNS~\cite{bib1} (Domain Name System) is the
Internet's phone book; it  
translates hostnames to IP addresses and vice versa.
This translation is through DNS resolution, which happens behind
the scene. DNS Pharming~\cite{bib4} attacks manipulate this resolution process
in various ways, with an intent to misdirect users to
alternative destinations, which are often malicious. 
This lab focuses on a particular DNS Pharming attack technique, called 
{\em DNS Cache Poisoning attack}. 



\section{Lab Environment}
\label{sec:environment}
This lab runs in the Labtainer framework,
available at http://my.nps.edu/web/c3o/labtainers.
That site includes links to a pre-built virtual machine
that has Labtainers installed, however Labtainers can
be run on any Linux host that supports Docker containers.

From your labtainer-student directory start the lab using:
\begin{verbatim}
    start.py remote-dns
\end{verbatim}
Links to this lab manual and to an empty lab report will be displayed.  If you create your lab report on a separate system,  
be sure to copy it back to the specified location on your Linux system. 



\begin{figure}[!htb]
\centering
\includegraphics*[ width=0.9\textwidth,natwidth=621,natheight=403]{Figs/remote-dns.jpg}
\caption{The Lab Environment Setup} 
\label{fig:environment}
\end{figure}


Figure~\ref{fig:environment} illustrates the setup of the lab environment. 
In this lab the user machine's IP address is {\tt 192.168.0.100}, 
the DNS Server's IP is {\tt 192.168.0.10} and the attacker machine's IP is {\tt 203.0.113.3}.


\paragraph {Note for Instructors:} 
For this lab, a lab session is desirable, especially if students are
not familiar with the tools and the environments. If an instructor
plans to hold a lab session (by himself/herself or by a TA), it
is suggested that the following topics are covered in the
lab session~\footnote{We assume that the instructor has already covered
the concepts of the attacks in the lecture, so we do not include
them in the lab session.}:
\begin{enumerate}
  \item The use of Labtainers. 

  \item The use of {\tt Wireshark}.

  \item Configuration of {\tt BIND 9} DNS server~\cite{bib2}.
\end{enumerate}


\subsection{Review the Local DNS server {\tt Apollo} Configuration} 

The {tt BIND 9} server program is installed on the Apollo DNS server.
The DNS server reads a configuration file named
{\tt /etc/bind/named.conf} when it starts. This configuration file includes an option 
file, which is called {\tt /etc/bind/named.conf.options}.  Please 
review that file and note this entry:
\begin{verbatim}
options {
       dump-file       "/var/cache/bind/dump.db";
};
\end{verbatim}

\noindent which instructs the DNS server to dump its cache into
a file named: \texttt{/var/cache/bind/dump.db} 
whenever you run the command:
\begin{verbatim}
% sudo rndc dumpdb -cache 	// Dump the cache to dump.db  
\end{verbatim}
\noindent You may delete the cache using:
\begin{verbatim}
% sudo rndc flush         	// Flush the DNS cache
\end{verbatim}


If a change is made to a configuration file, the DNS server must be
restarted:
\begin{verbatim}
% sudo /etc/init.d/bind9 restart
\end{verbatim}


\subsection{User Machine Configuration} 
\label{subsec:user_machine}

On the user machine {\tt 192.168.0.100}, we need to use 
{\tt 192.168.0.10} as the default DNS server. This is achieved by 
setting the file \texttt{/etc/resolv.conf} of the user machine to contain:

\begin{verbatim}
  nameserver 192.168.0.10 # the ip of the DNS server you just setup
\end{verbatim}

\noindent
This is already set on the user machine.

\subsection{The Wireshark Tool}

{\tt Wireshark} is a very important tool for this lab, and you 
probably need it to learn how exactly DNS works, as well as 
debugging your attacks.  This tool is installed in the Apollo 
DNS Server for your convenience.


\section{Lab Tasks}


The main objective of Pharming attacks is to redirect the user
to another machine $B$ when the user tries to get to machine $A$ using
$A$'s host name. For example, assuming {\tt www.example.com} is an online banking 
site.  When the user tries to access this site using the
correct URL {\tt www.example.com}, if the adversaries can redirect the user 
to a malicious web site that looks very much like 
{\tt www.example.com}, the user might be fooled and give away 
his/her credentials to the attacker. 

In this task, we use the domain name {\tt www.example.com}
as our attacking target. It should be noted that the {\tt example.com} 
domain name is reserved for use in documentation, not for 
any real company. The authentic IP address of {\tt www.example.com} is 
{\tt 93.184.216.34}, and its name server is managed by
the Internet Corporation for Assigned Names and Numbers (ICANN).
When the user runs the {\tt dig} command 
on this name or types the name in the browser, 
the user's machine sends a DNS query to its local DNS 
server, which will eventually ask for the IP address 
from {\tt example.com}'s name server. 


The goal of the attack is to launch the DNS cache poisoning attack
on the local DNS server, such that 
when the user runs the {\tt dig} command to find out {\tt
www.example.com}'s IP address, the local DNS server will end
up going to the attacker's name server {\tt ns.dnslabattacker.net} 
to get the IP address, so the IP address returned can be 
any number that is decided by the attacker. As results, the 
user will be led to the attacker's web site,
instead of the authentic {\tt www.example.com}.



There are two tasks in this attack: cache poisoning and result
verification.  In the first task, 
students need to poison the DNS cache of the user's local DNS server {\tt
Apollo}, such that, in {\tt Apollo}'s DNS cache,
{\tt ns.dnslabattacker.net} is set as the name server for 
the {\tt example.com} domain, instead of the domain's 
registered authoritative name server. 
In the second task, students need to demonstrate the impact of the attack.
More specifically, they need to run the command {\tt "dig
www.example.com"} from the user's machine, and the returned 
result must be a fake IP address. 




\begin{figure}[!htb]
\centering
\includegraphics*[width=0.9\textwidth,natwidth=621,natheight=403]{Figs/DNS_Remote_Flow1.pdf}
\caption{The complete DNS query process} 
\label{fig:flow_diagram1}
\end{figure}


\begin{figure}[!htb]
\centering
\includegraphics*[width=0.9\textwidth,natwidth=621,natheight=403]{Figs/DNS_Remote_Flow2.pdf}
\caption{The DNS query process when {\tt example.com}'s name server is cached}
\label{fig:flow_diagram2}
\end{figure}


\subsection{Task 1: Remote Cache Poisoning}

In this task, the attacker sends a DNS query request to the victim
DNS server ({\tt Apollo}), triggering a DNS query from {\tt Apollo}. The
query may go through one of the root DNS servers, the {\tt .COM} DNS server, and 
the final result will come back from {\tt example.com}'s DNS server. This 
is illustrated in Figure~\ref{fig:flow_diagram1}. In case that 
{\tt example.com}'s name server information is already cached by 
{\tt Apollo}, the query will not go through the root or the 
{\tt .COM} server; this is illustrated in Figure~\ref{fig:flow_diagram2}.
In this lab, the situation depicted in  Figure~\ref{fig:flow_diagram2} is 
more common, so we will use this figure as the basis to describe 
the attack mechanism.

While {\tt Apollo} waits for the DNS reply from {\tt example.com}'s name
server, the attacker can send forged replies to {\tt Apollo}, pretending 
that the replies are from {\tt example.com}'s name server. If the forged 
replies arrive first, it will be accepted by {\tt Apollo}. The attack will
be successful.


One reason that cache poisoning attacks are difficult
is that the transaction ID
in the DNS response packet must match with that 
in the query packet. Because the transaction ID in the query is 
usually randomly generated, without seeing the query packet,
it is not easy for the attacker to know the correct ID.


Obviously, the attacker can guess the transaction ID. Since the
size of the ID is only 16 bits, if the attacker can forge $K$ 
responses within the attack window (i.e. before the legitimate
response arrives), the probability of success is $K$ over $2^{16}$.
Sending out hundreds of forged responses is not impractical, so
it will not take too many tries before the attacker can succeed. 


However, the above hypothetical attack has overlooked the cache effect.
In reality, if the attacker is not fortunately enough to make a correct guess before
the real response packet arrives, correct information will be cached 
by the DNS server for a while. This caching effect makes it impossible
for the attacker to forge another response regarding the same 
domain name, because the DNS server will not send out another DNS query for 
this domain name before the cache times out.
To forge another response on the same domain name, the attacker has to 
wait for another DNS query on this domain name, which means he/she has to
wait for the cache to time out. The waiting period can be hours or days.


\paragraph{The Kaminsky Attack.} 
Dan Kaminsky came up with an elegant technique to defeat the caching effect~\cite{Kaminsky}.
With the Kaminsky attack, attackers will be able to continuously attack
a DNS server on a domain name, without the need for waiting, so
attacks can succeed within a very short period of time.
Details of the attacks are described in~\cite{Kaminsky}. 
In this task, we will try this attack method. The following steps with reference to 
Figure~\ref{fig:flow_diagram2} outlines the attack. 

\begin{enumerate}
\item The attacker queries the DNS Server {\tt Apollo} for a non-existing name in 
{\tt example.com}, such as {\tt twysw.example.com},
where {\tt twysw} is a random name. 

\item Since the mapping is unavailable in {\tt Apollo}'s DNS cache, 
{\tt Apollo} sends a DNS query to the name server of
the {\tt example.com} domain.

\item While {\tt Apollo} waits for the reply, 
the attacker floods {\tt Apollo} with a stream of spoofed DNS response~\cite{bib6}, 
each trying a different transaction ID, hoping one is correct.
In the response, not only does the attacker provide an IP resolution
for {\tt twysw.example.com}, the attacker 
also provides an ``Authoritative Nameservers'' record, indicating 
{\tt ns.dnslabattacker.net} as the name server for the {\tt example.com} domain.
If the spoofed response beats the actual responses and
the transaction ID matches with that in the query, 
{\tt Apollo} will accept and cache the spoofed answer, and
and thus {\tt Apollo}'s DNS cache is poisoned.  

\item Even if the spoofed DNS response fails (e.g.
the transaction ID does not match or it comes too late),
it does not matter, because the next time, the attacker will query
a different name, so {\tt Apollo} has to send out another query, 
giving the attack another chance to do the spoofing attack. 
This effectively defeats the caching effect.


\item If the attack succeeds, in {\tt Apollo}'s DNS cache, the
name server for {\tt example.com} will be replaced by the attacker's
name server {\tt ns.dnslabattacker.net}.
To demonstrate the success of this attack, students need to show that such a record 
is in {\tt Apollo}'s DNS cache. Figure~\ref{fig:cache_screenshot} shows 
an example of poisoned DNS cache.

\end{enumerate}




\paragraph{Attack Configuration.} The following configuration settings
facilitate your ability to perform the DNS cache poisoning attack. 

\begin{enumerate}

\item {\em Attack Machine Configuration.} 
The attack machine uses the targeted 
DNS server (i.e., {\tt Apollo}) as its default DNS server. 
The attacker {\tt /etc/resolv.conf} file is configured with the external
address of the site, exported by the gateway, i.e., 203.0.113.10.
The gateway forwards DNS requests to this address to the {\tt Apollo} DNS server.

\item {\em Source Ports.} Some DNS servers now randomize the source port number 
in the DNS queries; this makes the attacks much more difficult. Unfortunately, 
many DNS servers still use predictable source port number.  
For the sake of simplicity in this lab, we assume that the source port 
number is a fixed number. The Apollo DNS server is configured to set
the source port for all DNS queries 
to {\tt 33333}. This was done by adding the following option to the file {\tt /etc/bind/named.conf.options}
on {\tt Apollo}:
\begin{verbatim}
   query-source port 33333
\end{verbatim}

\item {\em DNSSEC.}
Most DNS servers now adopt a protection scheme called "DNSSEC", which is
designed to defeat the DNS cache poisoning attack.  If you do not turn
it off, your attack would be extremely difficult, if not impossible.
In this lab, we have disabled DNSSEC by changing 
the file {\tt /etc/bind/named.conf.options} on {\tt Apollo}. Please find the line 
{\tt "dnssec-validation auto"}, and note it is commented out, and a new line was added. See
the following:
\begin{verbatim}
 //dnssec-validation auto;
   dnssec-enable no;
\end{verbatim}


\item {\em Flush the Cache.}
Flush {\tt Apollo}'s  DNS cache, and restart its DNS server. 


\end{enumerate}


\begin{figure}[!htb]
\centering
\includegraphics*[viewport=0 50 600 750,width=1.0\textwidth,natwidth=621,natheight=403]{Figs/screenshot_packet.pdf}
\caption{A Sample DNS Response Packet}
\label{fig:response_packet}
\end{figure}



\paragraph{Forge DNS Response Packets.}
In order to complete the attack, the attacker first needs to send 
DNS queries to {\tt Apollo} for some random host names in
the {\tt example.com} domain. Right after each query is sent out, 
the attacker needs to forge a large number of DNS response packets in a
very short time window,
hoping that one of them has the correct transaction ID and it reaches the target before 
the authentic response does. It is better to write C code to achieve this. 
To make your life easier, we have provided a sample code called {\tt “udp.c”}. 
This program can send a large number of DNS packets. Feel free to 
use this sample code when writing your attack programs. It is in the HOME directory
of the attacker machine.

\begin{enumerate}

\item When modifying the {\tt udp.c} program, you need to fill 
each DNS field with the correct value.  To understand the value in each
field, you can use {\tt Wireshark} to capture a few DNS query and response packets. 

\item DNS response packet details: it is not easy to construct a correct DNS
response packet. We made a sample packet to help you.
Figure~\ref{fig:response_packet} is the screen shot of an example response packet:
{\tt 10.0.2.6} is the local DNS server address, and 
{\tt 199.43.132.53} is the real name server for {\tt example.com}. 
The highlighted bytes are the raw UDP payload data, and you need to figure
out what they are.
The details about how each byte works are explained clearly in 
Appendix~\ref{sec:response_detail}.
There are several techniques used in the response packet, such as  
the string pointer offset to shorten the packet length. 
You may not have to use that technique but it is
very common in real packets.

\end{enumerate} 


Check the {\tt dump.db} file to see whether your spoofed DNS
response has been successfully accepted by the DNS server. 
See an example in Figure~\ref{fig:cache_screenshot}.

\begin{figure}[!htb]
\centering
%\includegraphics*[viewport=0 0 600 680,width=1.0\textwidth]{Figs/cache_screenshot.pdf}
\includegraphics*[viewport=0 180 600 630,width=0.85\textwidth,natwidth=621,natheight=403]{Figs/cache_screenshot.pdf}
\caption{A Sample of Successfully Poisoned DNS Cache}
\label{fig:cache_screenshot}
\end{figure}


\subsection{Task 2: Result Verification}


If your attack is successful, {\tt Apollo}'s DNS cache will look like 
that in Figure~\ref{fig:cache_screenshot}, i.e., the {\tt NS} record 
for {\tt example.com} becomes {\tt ns.dnslabattacker.net}. To make sure
that the attack is indeed successful, we run the {\tt dig} command 
on the user machine (see Figure~\ref{fig:environment}) to 
ask for {\tt www.example.com}'s IP address. 

When {\tt Apollo} receives the DNS query, it searches
for {\tt example.com}'s {\tt NS} record in its cache,
and finds {\tt ns.dnslabattacker.net}.
It will therefore send a DNS query to {\tt ns.dnslabattacker.net}.
However, before sending the query, it needs to know the IP address of 
{\tt ns.dnslabattacker.net}. This is done by issuing a separate DNS query. 
That is where we get into trouble.


The domain name {\tt dnslabattacker.net} does not exist in reality.
We created this name for the purpose of this lab. {\tt Apollo} will soon
find out about that, and mark the {\tt NS} entry invalid, essentially recovering 
from the poisoned cache. One may say that when forging the DNS response, we
can use an additional record to provide the IP address for 
{\tt ns.dnslabattacker.net}. The sample response packet in 
Figure~\ref{fig:response_packet} actually does that. Unfortunately,
this additional record will not be accepted by {\tt Apollo}. Please think
about why and give your explanation in your lab report (hint: think
about the {\em zones}).

We will demonstrate the impact
of our successful cache-poisoning attack by using a fake domain name.
Note the changes made to the Apollow DNS server described below are simply
to facilitate the demonstration. If we controlled a real domain, we would achive
this by using that domain in the {\tt NS} record in place of {\tt dnslabattacker.net}.

\paragraph{Use A Fake Domain Name.}
We have configured {\tt Apollo} so
it recognizes {\tt dnslabattacker.net} as a real domain. We added
the {\tt ns.dnslabattacker.net}'s IP address to {\tt Apollo}'s DNS configuration,
so {\tt Apollo} does not need to go out asking for the IP address of this
hostname from a non-existing domain. The configuration was achieved as described
below.

We configured the victim's DNS server {\tt Apollo}
file named {\tt named.conf.default-zones} in
the {\tt /etc/bind/} folder, to include the following entry:

\begin{verbatim}
zone "ns.dnslabattacker.net" {
                type master;
                file "/etc/bind/db.attacker";
};
\end{verbatim}

And we added the file {\tt /etc/bind/db.attacker}, with the following
content:

\begin{verbatim}
$TTL 604800
@		IN		SOA		localhost. root.localhost. (
                2; Serial
                604800 ; Refresh
                86400 ; Retry
                2419200 ; Expire
                604800 ) ; Negative Cache TTL;
@		IN		NS		ns.dnslabattacker.net.
@		IN		A		203.0.113.3
@		IN		AAAA ::1
\end{verbatim}
\noindent Note, we let the attacker's machine and the malicious DNS, 
{\tt ns.dnslabattacker.net} share the machine ({\tt 192.168.0.200}). 

If your cache poisoning attack is successful, any 
DNS query sent to {\tt Apollo} for the hostnames 
in {\tt example.com} will be sent to {\tt 203.0.113.3}, which is 
attacker's machine. 


The DNS server on {\tt 203.0.113.3} has been configured so it answers the 
queries for the domain {\tt example.com}.  This was achieved by
the following 
entry in {\tt /etc/bind/named.conf.local} on {\tt 203.0.113.3}:

\begin{verbatim}
zone "example.com" {
                type master;
                file "/etc/bind/example.com.db";
};
\end{verbatim}

\noindent and a file called {\tt /etc/bind/example.com.db}, containing:

\begin{verbatim}
$TTL 3D
@               IN         SOA ns.example.com. admin.example.com. (
                2008111001
                8H
                2H
                4W
                1D)	
@               IN          NS          ns.dnslabattacker.net.
@               IN          MX          10 mail.example.com.
www             IN          A           1.1.1.1	
mail            IN          A           1.1.1.2
*.example.com   IN          A           1.1.1.100
\end{verbatim}

If everything is done properly, you can use the command like {\tt
"dig www.example.com”} on the user machine. The reply would be {\tt
1.1.1.1}, which is exactly we put in the above file. 



\section{Submission}

Students need to submit a detailed lab report to describe what they have done and
what they have observed. Report should include the evidences to support 
the observations. Evidences include packet traces, screen dumps, etc.

\paragraph{Note:} Please do not forget to answer the question asked in
Task 2, regarding why the IP address for 
{\tt ns.dnslabattacker.net} in the additional field is not
accepted by the victim DNS server. 

If you edited your lab report on a separate system, copy it back to the Linux system at the location
identified when you started the lab, and do this before running the stop.py command.
After finishing the lab, go to the terminal on your Linux system that was used to start the lab and type:
\begin{verbatim}
    stop.py remote-dns
\end{verbatim}
When you stop the lab, the system will display a path to the zipped lab results on your Linux system.  Provide that file to
your instructor, e.g., via the Sakai site.


\begin{thebibliography}{10}

\bibitem {Kaminsky}
\newblock D. Schneider.
\newblock Fresh Phish, How a recently discovered flaw in the Internet's Domain Name
System makes it easy for scammers to lure you to fake Web sites.
\newblock {\em IEEE Spectrum}, 2008  
\newblock{\url{http://spectrum.ieee.org/computing/software/fresh-phish}}

\bibitem{bib1}
RFC 1035 Domain Names - Implementation and Specification :
\newblock http://www.rfc-base.org/rfc-1035.html

\bibitem{bib2}
DNS HOWTO :
\newblock http://www.tldp.org/HOWTO/DNS-HOWTO.html

\bibitem{bib4}
Pharming Guide :
\newblock http://www.technicalinfo.net/papers/Pharming.html
%\newblock http://www.ngssoftware.com/papers/ThePharmingGuide.pdf

\bibitem{bib5}
DNS Cache Poisoning:
\newblock http://www.secureworks.com/resources/articles/other\_articles/dns-cache-poisoning/

\bibitem{bib6}
DNS Client Spoof:
\newblock http://evan.stasis.org/odds/dns-client\_spoofing.txt

%\bibitem{bib6}
%Phishing:
%\newblock http://en.wikipedia.org/wiki/Phishing

\end{thebibliography}


\appendix

\section{Details of DNS Response Packet}
\label{sec:response_detail}


{\footnotesize
\begin{Verbatim}[frame=single]
0x8e 0x01 transaction ID
0x84 0x00 flags:means a no-error answer
0x00 0x01 Questions No.     (1 question session)
0x00 0x01 Answer No.        (1 answer session)
0x00 0x01 Authority No.     (1 authority session)
0x00 0x02 Additional No.    (2 additional sessions)
query session: eggdd.example.com:type A, class IN
0x05 5 characters follow
0x74 t
0x77 w
0x79 y
0x73 s
0x77 w
0x07 7 characters follow
0x65 e
0x78 x
0x61 a
0x6d m
0x70 p
0x6c l
0x65 e
0x03 3 characters
0x63 c
0x6f o
0x6d m
0x00 end of the string
0x00 0x01 type:A(address)
0x00 0x01 Class:IN
the Answer session:
0xc0 first two bits set to 1 to notify this is a pointer for a name string,
not a standard
string as before
0x0c the offset of the start point: here from transaction ID field to the
name string
12 bytes. The string will shows from the offset point to the end of the
string
0x00 0x01 type:A
0x00 0x01 Class:IN
0x02 0x00 0x00 0x00 time to live
0x00 0x04 DataLength:4 bytes
0x01 0x01 x01 0x01 1.1.1.1
Authoritative Nameservers session:
0xC0 first two bits set to 1 to notify this is a pointer for a name string,
not a standard
string as before
0x12 Offset 18 the string should be "/7example/3com/0"
0x00 0x02 type:NS
0x00 0x01 Class:IN
0x02 0x00 0x00 0x00 time to live
0x00 0x17 DataLength:23 bytes
The string represent "/2ns/14dnslabattacker/3net"
0x02 2 characters follow
0x6e n
0x73 s
0x0e 14 characters
0x64 d
0x6e n
0x73 s
0x6c l
0x61 a
0x62 b
0x61 a
0x74 t
0x74 t
0x61 a
0x63 c
0x6b k
0x65 e
0x72 r
0x03 3 characters
0x6e n
0x65 e
0x74 t
0x00 end of the string
****************************additional session***************************************
first session :example.com:type NS,class IN ns ns.dnslabattacker.net

notice: you can use the same pointer technique we
talked before to shorten
the packet, this is just to show you both ways work.
The string represent "/2ns/14dnslabattacker/3net"
0x02 2 characters follow
0x6e n
0x73 s
0x0e 14 characters
0x64 d
0x6e n
0x73 s
0x6c l
0x61 a
0x62 b
0x61 a
0x74 t
0x74 t
0x61 a
0x63 c
0x6b k
0x65 e
0x72 r
0x03 3 characters
0x6e n
0x65 e
0x74 t
0x00 end of the string
0x00 0x01 type:A
0x00 0x01 Class:IN
0x02 0x00 0x00 0x00 time to live
0x00 0x04 DataLength:4 bytes
0x01 0x01 0x01 0x01 1.1.1.1
second session: not related to the lab. Just set a rule that during the DNS
communication, the server won't accept
the packet which is larger than a certain size
0x00 0x00 0x29 0x10 0x00 0x00 0x00 0x88 0x00 0x00 0x00
\end{Verbatim}
}
\end{document}

