

\documentclass[11pt]{article}

\usepackage{times}
\usepackage{epsf}
\usepackage{epsfig}
\usepackage{amsmath, alltt, amssymb, xspace}
\usepackage{wrapfig}
\usepackage{fancyhdr}
\usepackage{url}
\usepackage{verbatim}
\usepackage{fancyvrb}

\usepackage{subfigure}
\usepackage{cite}
%\usepackage{cases}
%\usepackage{ltexpprt}
%\usepackage{verbatim}

%\topmargin      -0.70in  % distance to headers
%\headheight     0.2in   % height of header box
%\headsep        0.4in   % distance to top line
%\footskip       0.3in   % distance from bottom line

% Horizontal alignment
\topmargin      -0.50in  % distance to headers
\oddsidemargin  0.0in
\evensidemargin 0.0in
\textwidth      6.5in
\textheight     8.9in 


%\centerfigcaptionstrue

%\def\baselinestretch{0.95}


\newcommand\discuss[1]{\{\textbf{Discuss:} \textit{#1}\}}
%\newcommand\todo[1]{\vspace{0.1in}\{\textbf{Todo:} \textit{#1}\}\vspace{0.1in}}
\newtheorem{problem}{Problem}[section]
%\newtheorem{theorem}{Theorem}
%\newtheorem{fact}{Fact}
\newtheorem{define}{Definition}[section]
%\newtheorem{analysis}{Analysis}
\newcommand\vspacenoindent{\vspace{0.1in} \noindent}

%\newenvironment{proof}{\noindent {\bf Proof}.}{\hspace*{\fill}~\mbox{\rule[0pt]{1.3ex}{1.3ex}}}
%\newcommand\todo[1]{\vspace{0.1in}\{\textbf{Todo:} \textit{#1}\}\vspace{0.1in}}

%\newcommand\reducespace{\vspace{-0.1in}}
% reduce the space between lines
%\def\baselinestretch{0.95}

\newcommand{\fixmefn}[1]{ \footnote{\sf\ \ \fbox{FIXME} #1} }
\newcommand{\todo}[1]{
\vspace{0.1in}
\fbox{\parbox{6in}{TODO: #1}}
\vspace{0.1in}
}

\newcommand{\mybox}[1]{
\vspace{0.2in}
\noindent
\fbox{\parbox{6.5in}{#1}}
\vspace{0.1in}
}


\newcounter{question}
\setcounter{question}{1}

\newcommand{\myquestion} {{\vspace{0.1in} \noindent \bf Question \arabic{question}:} \addtocounter{question}{1} \,}

\newcommand{\myproblem} {{\noindent \bf Problem \arabic{question}:} \addtocounter{question}{1} \,}


\newcommand{\copyrightnoticeA}[1]{
\vspace{0.1in}
\fbox{\parbox{6in}{\small Copyright \copyright\ 2006 - 2014\ \ Wenliang Du, Syracuse University.\\ 
      The development of this document is partially funded by 
      the National Science Foundation's Course, Curriculum, and Laboratory 
      Improvement (CCLI) program under Award No. 0618680 and 0231122. 
      Permission is granted to copy, distribute and/or modify this document
      under the terms of the GNU Free Documentation License, Version 1.2
      or any later version published by the Free Software Foundation.
      A copy of the license can be found at http://www.gnu.org/licenses/fdl.html.}}
\vspace{0.1in}
}


\newcommand{\copyrightnotice}[1]{
\vspace{0.1in}
\fbox{\parbox{6in}{\small Copyright \copyright\ 2006 - 2014\ \ Wenliang Du, Syracuse University.\\
      The development of this document is/was funded by three grants from
      the US National Science Foundation: Awards No. 0231122 and 0618680 from
      TUES/CCLI and  Award No. 1017771 from Trustworthy Computing.
      This lab was imported into the Labtainer framework by the Naval Postgraduate 
      School, Center for Cybersecurity and Cyber Operations under National Science 
      Foundation Award No. 1438893.
      Permission is granted to copy, distribute and/or modify this document
      under the terms of the GNU Free Documentation License, Version 1.2
      or any later version published by the Free Software Foundation.
      A copy of the license can be found at http://www.gnu.org/licenses/fdl.html.}}
\vspace{0.1in}
}

\newcommand{\copyrightnoticeB}[1]{
\vspace{0.1in}
\fbox{\parbox{6in}{\small Copyright \copyright\ 2006 - 2014\ \ Wenliang Du, Syracuse University.\\
      The development of this document is/was funded by the following grants from
      the US National Science Foundation: No. 0231122, 0618680, and 1303306.
      Permission is granted to copy, distribute and/or modify this document
      under the terms of the GNU Free Documentation License, Version 1.2
      or any later version published by the Free Software Foundation.
      A copy of the license can be found at http://www.gnu.org/licenses/fdl.html.}}
\vspace{0.1in}
}


\newcommand{\nocopyrightnotice}[1]{
\vspace{0.1in}
\fbox{\parbox{6in}{\small  
      The development of this document is funded by 
      the National Science Foundation's Course, Curriculum, and Laboratory 
      Improvement (CCLI) program under Award No. 0618680 and 0231122. 
      Permission is granted to copy, distribute and/or modify this document.
      }}
\vspace{0.1in}
}

\newcommand{\idea}[1]{
\vspace{0.1in}
{\sf IDEA:\ \ \fbox{\parbox{5in}{#1}}}
\vspace{0.1in}
}

\newcommand{\questionblock}[1]{
\vspace{0.1in}
\fbox{\parbox{6in}{#1}}
\vspace{0.1in}
}


\newcommand{\minix}{{\tt Minix}\xspace}
\newcommand{\unix}{{\tt Unix}\xspace}
\newcommand{\linux}{{\tt Linux}\xspace}
\newcommand{\ubuntu}{{\tt Ubuntu}\xspace}
\newcommand{\selinux}{{\tt SELinux}\xspace}
\newcommand{\freebsd}{{\tt FreeBSD}\xspace}
\newcommand{\solaris}{{\tt Solaris}\xspace}
\newcommand{\windowsnt}{{\tt Windows NT}\xspace}
\newcommand{\setuid}{{\tt Set-UID}\xspace}
%\newcommand{\smx}{{\tt Smx}\xspace}
\newcommand{\smx}{{\tt Minix}\xspace}
\newcommand{\relay}{{\tt relay}\xspace}
\newcommand{\isys}{{\tt iSYS}\xspace}
\newcommand{\ilan}{{\tt iLAN}\xspace}
\newcommand{\iSYS}{{\tt iSYS}\xspace}
\newcommand{\iLAN}{{\tt iLAN}\xspace}
\newcommand{\iLANs}{{\tt iLAN}s\xspace}
\newcommand{\bochs}{{\tt Bochs}\xspace}

\newcommand\FF{{\mathcal{F}}}

\newcommand{\argmax}[1]{
\begin{minipage}[t]{1.25cm}\parskip-1ex\begin{center}
argmax
#1
\end{center}\end{minipage}
\;
}

\newcommand{\bm}{\boldmath}
\newcommand  {\bx}    {\mbox{\boldmath $x$}}
\newcommand  {\by}    {\mbox{\boldmath $y$}}
\newcommand  {\br}    {\mbox{\boldmath $r$}}


%\pagestyle{fancyplain}
%\lhead[\thepage]{\thesection}      % Note the different brackets!
%\rhead[\thesection]{SEED Laboratories}
%\lfoot[\fancyplain{}{}]{Syracuse University} 
%\cfoot[\fancyplain{}{}]{\thepage} 

\newcommand{\tstamp}{\today}   
%\lhead[\fancyplain{}{\thepage}]         {\fancyplain{}{\rightmark}}
%\chead[\fancyplain{}{}]                 {\fancyplain{}{}}
%\rhead[\fancyplain{}{\rightmark}]       {\fancyplain{}{\thepage}}
%\lfoot[\fancyplain{}{}]                 {\fancyplain{\tstamp}{\tstamp}}
%\cfoot[\fancyplain{\thepage}{}]         {\fancyplain{\thepage}{}}
%\rfoot[\fancyplain{\tstamp} {\tstamp}]  {\fancyplain{}{}}

\pagestyle{fancy}
%\lhead{\bfseries Computer Security Course Project}
\lhead{\bfseries SEED Labs}
\chead{}
\rhead{\small \thepage}
\lfoot{}
\cfoot{}
\rfoot{}

\usepackage{listings}
\usepackage{color}

\definecolor{dkgreen}{rgb}{0,0.6,0}
\definecolor{gray}{rgb}{0.5,0.5,0.5}
\definecolor{mauve}{rgb}{0.58,0,0.82}

\lstset{frame=tb,
  language=C,
  aboveskip=3mm,
  belowskip=3mm,
  showstringspaces=false,
  columns=flexible,
  basicstyle={\small\ttfamily},
  numbers=none,
  numberstyle=\tiny\color{gray},
  keywordstyle=\color{blue},
  commentstyle=\color{dkgreen},
  stringstyle=\color{mauve},
  breaklines=true,
  breakatwhitespace=true,
  tabsize=3
}




\lhead{\bfseries SEED Labs -- Environment Variable and \setuid Program Lab}


\begin{document}


\begin{center}
{\LARGE Environment Variable and \setuid Program Lab}
\end{center}

\copyrightnotice


\section{Overview}

The learning objective of this lab is for students to understand how
environment variables affect program and system behavior. Environment
variables are a set of dynamic named values that can affect the way running processes
will behave on a computer. They are used by most operating systems, since
they were introduced to Unix in 1979. Although environment variables affect
program behavior, how they achieve that is not well understood by many
programmers. As results, if a program uses  environment
variables,  but the programmer do not know that they are used, the program
may have vulnerabilities. In this lab, students will understand how
environment variables work, how they are propogated from parent process to
child, and how they affect system/program bahivors. We are particularly
interested in how environment variables affect the behavior of \setuid
programs, which are usually privileged programs. 

\section{Lab Environment}

This lab runs in the Labtainer framework,
available at http://nps.edu/web/c3o/labtainers.
That site includes links to a pre-built virtual machine
that has Labtainers installed, however Labtainers can 
be run on any Linux host that supports Docker containers.

From your labtainer-student directory start the lab using:
\begin{verbatim}
    labtainer setuid-env
\end{verbatim}
Links to this lab manual and to an empty lab report will be displayed.  If you create your lab report on a separate system, 
be sure to copy it back to the specified location on your Linux system. 


\section{Lab Tasks}

\subsection{Task 1: Manipulating environment variables}

In this task, we study the commands that can be used to set and unset
environment variables. We are using the Bash in this lab. The default
shell that a user uses is set in  the {\tt /etc/passwd} file (the last
field of each entry). You can change this to another shell program using
the command {\tt chsh} (please do not do it for this lab). Please 
do the following tasks:

\begin{itemize}
\item Use {\tt printenv} or {\tt env} command to print out the 
environment variables. If you are interested in some particular 
environment variables, such as {\tt PWD}, you can use {\tt "printenv PWD"}
or {\tt "env | grep PWD"}. 


\item Use {\tt export} and {\tt unset} to set or unset environment
variables. It should be noted that 
these two commands are not separate programs; they are two of 
the Bash's internal commands (you will not be able to find them
outside of Bash).

\end{itemize}


\subsection{Task 2: Inheriting environment variables from parents}

In this task, we study how environment variables are inherited by 
child processes from their parents. In Unix,
{\tt fork()} creates a new process by duplicating the calling process.
The new process, referred to as the child, is an exact duplicate of the calling 
process, referred to as the parent; however, several things 
are not inherited by the child (please see the manual of {\tt fork()} by
typing the following command: {\tt man fork}). In this task,
we would like to know whether the parent's environment variables
are inherited by the child process or not.


\paragraph{Step 1.} Please compile and run the {\tt printenv.c}
program (the listing is below),  and describe your observation. Because the output 
contains many strings, you should save the output into a file, such as
using {\tt a.out > child} (assuming that {\tt a.out} is your 
executable file name).


\begin{Verbatim}[frame=single]
#include <unistd.h>
#include <stdio.h>
#include <stdlib.h>

extern char **environ;

void printenv()
{
  int i = 0;
  while (environ[i] != NULL) {
     printf("%s\n", environ[i]);
     i++;
  }
}

void main()
{
  pid_t childPid;

  switch(childPid = fork()) {
    case 0:  /* child process */
        printenv();
	exit(0);
    default:  /* parent process */
        //printenv();
	exit(0);
  }
}
\end{Verbatim}

\paragraph{Step 2.} Now comment out the {\tt printenv()} statement in
the child process case, and uncomment the {\tt printenv()} statement in
the parent process case. Compile and run the code, and describe your 
observation. Save the output in another file.


\paragraph{Step 3.} Compare the difference of these two files using 
the {\tt diff} command. Please draw your conclusion.



\subsection{Task 3: Environment variables and {\tt execve()}}

In this task, we study how environment variables are affected
when a new program is 
executed via {\tt execve()}. The 
function {\tt execve()} calls a system call to load 
a new command and execute it; this function never returns. 
No new process is created; instead, the calling 
process's text,  data, bss, and stack  are overwritten by that of 
the program loaded. Essentially, {\tt execve()} runs the new program inside
the calling process. We are interested in what happens to the 
environment variables; are they automatically inherited by the 
new program?


\paragraph{Step 1.} Please compile and run the {\tt execve.c} program,
(the listing is below), and describe your observation. This program simply executes
a program called {\tt /usr/bin/env}, which prints out the environment
variables of the current process.

\begin{Verbatim}[frame=single]
#include <stdio.h>
#include <stdlib.h>

extern char **environ;

int main()
{
  char *argv[2];

  argv[0] = "/usr/bin/env"; 
  argv[1] = NULL;

  execve("/usr/bin/env", argv, NULL);

  return 0 ;
}
\end{Verbatim}


\paragraph{Step 2.} Now, change the invocation of {\tt execve()} to the following, and 
describe your observation.
\begin{verbatim}
  execve("/usr/bin/env", argv, environ);
\end{verbatim}


\paragraph{Step 3.} Please draw your conclusion regarding how the new
program gets its environment variables. 


\subsection{Task 4: Environment variables and {\tt system()}}

In this task, we study how environment variables are affected 
when a new program is executed via the {\tt system()} function. This 
function is used to execute a command, but unlike
{\tt execve()}, which directly executes a command, {\tt system()}
actually executes {\tt "/bin/sh -c command"}, i.e., it
executes {\tt /bin/sh}, and asks the shell to execute the command.

If you look at the implementation of the {\tt system()} function, you will
see that it uses {\tt execl()} to execute {\tt /bin/sh}; {\tt excel()}
calls {\tt execve()}, passing to it the environment variables array. 
Therefore, using {\tt system()},  the environment variables of the calling process 
are passed to the new program {\tt /bin/sh}. Please compile and run the following 
{\tt system.c} program, (listing is below) to verify this. 


\begin{Verbatim}[frame=single]
#include <stdio.h>
#include <stdlib.h>

int main()
{
  system("/usr/bin/env");

  return 0 ;
}
\end{Verbatim}


\subsection{Task 5: Environment variable and \setuid Programs}

{\tt Set-UID} is an important security mechanism in Unix operating systems.
When a {\tt Set-UID} program runs, it assumes the owner's privileges. For
example, if the program's owner is root, then when anyone runs this
program, the program gains the root's privileges during its execution. {\tt
Set-UID} allows us to do many interesting things, but it escalates the
user's privilege when executed, making it quite risky. Although the
behavior of a \setuid programs is decided by its program logic, not by
users, users can indeed affect the behavior via environment variables. 
To understand how \setuid programs are affected, let us first figure out
whether environment variables are inherited by the \setuid program's
process from the user's process.


\paragraph{Step 1.} We will use the {\tt printall.c} program, 
(listing below) that prints out all the environment variables in the current process.

\begin{Verbatim}[frame=single]
#include <stdio.h>
#include <stdlib.h>

extern char **environ;

void main()
{
  int i = 0;
  while (environ[i] != NULL) {
     printf("%s\n", environ[i]);
     i++;
  }
}
\end{Verbatim}


\paragraph{Step 2.} Compile the {\tt printall.c} program, change its ownership to {\tt
root}, and make it a \setuid program. 
\begin{Verbatim}
    sudo chown root:root a.out
    sudo chmod a+s a.out
\end{Verbatim}
\paragraph{Step 3.} In your Bash shell 
use the {\tt export} command to set the
following environment variables (they may have already exist):
\begin{itemize}
\item {\tt PATH}
\item {\tt LD\_LIBRARY\_PATH}
\item {\tt ANY\_NAME} (this is an environment variable defined by you, so
pick whatever name you want).
\end{itemize}

These environment variables are set in the user's shell process.
Now, run the \setuid program from Step 2 in your shell. After you type the
name of the program in your shell, the shell forks a child process,
and uses the child process to run the program. Please check whether all the
environment variables you set in the shell process (parent) get into
the \setuid child process.  Describe your observation. If there are 
surprises to you, describe them.  



\subsection{Task 6: The system() function and \setuid Programs}
One of many changes made to the setuid feature over the years relates
to its interaction with the system() function.  Since the results
of system() are dependent on environment variables, it was a source
of increased risk.  For example, use of the {\tt PATH} environment
variable will determine which of potentially several instances of a
given program will be invoked.  The following command:
\begin{verbatim}
    export PATH=~/:$PATH
\end{verbatim}
\noindent would cause the shell to first look in your home directory for a program
of a given name.

The setuid {\tt path-suid.c} program list below is supposed to execute the {\tt /bin/ls} command; 
however, the programmer only uses the relative path for the {\tt ls} 
command, rather than the absolute path:

\begin{Verbatim}[frame=single] 
# path-suid.c
#include <stdlib.h>
#include <stdio.h>
#include <unistd.h>
#include <sys/types.h>
int main()
{
   uid_t euid = geteuid();
   printf("euid %d\n", euid);
   system("ls");
   return 0;
}
\end{Verbatim}

Please compile the {\tt path-suid.c} program, and change its owner to {\tt root}, and 
make it a \setuid program.  

Can you cause the path-suid.c \setuid program run the ls.c program instead of 
{\tt /bin/ls}?  If you can, is your code running with the root privilege?
Describe and explain your observations. 
\begin{Verbatim}[frame=single] 
# ls.c
#include <stdio.h>
#include <unistd.h>
#include <sys/types.h>
int main(){
    uid_t euid = geteuid();
    printf("my ls prog, euid is %d\n", euid);
    return 0;
}
\end{Verbatim}



\subsection{Task 7: The {\tt LD\_PRELOAD} environment variable and \setuid
Programs}

In this task, we study how \setuid programs deal
with some of the environment variables.
Several environment variables, including {\tt LD\_PRELOAD}, 
{\tt LD\_LIBRARY\_PATH}, and other {\tt LD\_*} influence the 
behavior of dynamic loader/linker.
A dynamic loader/linker is the part of an operating system (OS) that 
loads (from persistent storage to RAM) and links the shared libraries 
needed by an executable at run time. 

In Linux, {\tt ld.so} or {\tt ld-linux.so}, are the dynamic 
loader/linker (each for different types of binary).
Among the environment variables that affect their behavior,
{\tt LD\_LIBRARY\_PATH} and {\tt LD\_PRELOAD} are the two
that we are concerned in this lab. 
In Linux, {\tt LD\_LIBRARY\_PATH} is a colon-separated set
of directories where libraries should be searched for first, before the
standard set of directories. 
{\tt LD\_PRELOAD} specifies a list of additional, user-specified, shared libraries to
be loaded before all others. In this task, we will only 
study {\tt LD\_PRELOAD}.


\paragraph{Step 1.} 
First, we will see how these environment variables influence the 
behavior of dynamic loader/linker when running a normal program. 
Please follow these steps:


  \begin{enumerate}
  \item Let us build a dynamic link library. The following program listing,
  is in {\tt mylib.c}. It basically overrides the {\tt sleep()} function 
  in {\tt libc}:
  \begin{verbatim}
  #include <stdio.h>
  void sleep (int s)
  {
    /* If this is invoked by a privileged program, 
       you can do damages here!  */
    printf("I am not sleeping!\n");
  }
  \end{verbatim}

  \item We can compile the {\tt mylib.c} program using the following commands (in the 
  {\tt -lc} argment, the second character is $\ell$):
  \begin{verbatim}
  % gcc -fPIC -g -c mylib.c
  % gcc -shared -o libmylib.so.1.0.1 mylib.o -lc
  \end{verbatim}

  

  \item Now, set the {\tt LD\_PRELOAD} environment variable:  
  \begin{verbatim}
  % export LD_PRELOAD=./libmylib.so.1.0.1  
  \end{verbatim}

  \item Finally, compile the {\tt myprog.c} program
  in the same directory as the above dynamic link library {\tt
  libmylib.so.1.0.1}:
  \begin{verbatim}
  /* myprog.c */
  int main()
  {
    sleep(1);
    return 0;
  }
  \end{verbatim}
  \end{enumerate}


\paragraph{Step 2.} 
After you have done the above, please run {\tt myprog} under the following
conditions, and observe what happens. 

  \begin{itemize}
  \item Make {\tt myprog} a regular program, and run it as a normal user.
  \item Make {\tt myprog} a \setuid root program, and run it as a normal user.
  \item Become root with {\tt sudo su}, export the {\tt LD\_PRELOAD}
  environment variable again and run the {\tt myprog} program again.

  \item Make {\tt myprog} a \setuid user1 program (i.e., the owner is user1, which 
        is another user account), export the {\tt LD\_PRELOAD} environment variable 
	again as the user1 user and run it.
  \end{itemize}


\paragraph{Step 3.} 
You should be able to observe different behaviors in the scenarios
described above, even though you are running the same program.  You need 
to figure out what causes the difference. Environment variables 
play a role here. Please design an experiment to figure out the 
main causes, and explain why the behavior in Step 2 is 
different. (Hint: the child process 
may not inherit the {\tt LD\_*} environment variables). 


\subsection{Task 8: Capability Leaking}

To follow the Principle of Least Privilege, \setuid programs often
permanently relinquish their root privileges if such privileges are not
needed anymore. Moreover, sometimes, the program needs to hand over its 
control to the user; in this case, root privileges must be revoked.
The {\tt setuid()} system call can be used to revoke the privileges. 
According to the manual, ``\texttt{setuid()} sets the effective user ID of
the calling process. If the effective UID of the caller is root, the real
UID and saved set-user-ID are also set''. Therefore, if a \setuid program
with effective UID 0 calls \texttt{setuid(n)}, the process will become a
normal process, with all its UIDs being set to \texttt{n}. 

When revoking the privilege, one of the common mistakes is capability
leaking. The process may have gained some privileged capabilities when it
was still privileged; when the privilege is dropped, if the program
does not clean up those capabilities, they may still be accessible by the
non-privileged process. In other words, although the effective user ID of
the process becomes non-privileged, the process is still privileged because
it possesses privileged capabilities, e.g., access to a protected file.


Compile the {\tt leak.c} program, change its owner to root, and
make it a \setuid program. Run the program as a normal user,
and describe what you have observed.
Will the file {\tt /etc/zzz} be modified? Please explain
your observation.

\begin{Verbatim}[frame=single]
#include <stdio.h>
#include <stdlib.h>
#include <fcntl.h>

void main()
{ int fd;

  /* Assume that /etc/zzz is an important system file,
   * and it is owned by root with permission 0644.
  fd = open("/etc/zzz", O_RDWR | O_APPEND);
  if (fd == -1) {
     printf("Cannot open /etc/zzz\n");
     exit(0);
  }

  /* Simulate the tasks conducted by the program */
  sleep(1);

  /* After the task, the root privileges are no longer needed,
     it's time to relinquish the root privileges permanently. */
  setuid(getuid());  /* getuid() returns the real uid */

  if (fork()) { /* In the parent process */
    close (fd);
    exit(0);
  } else { /* in the child process */
    /* Now, assume that the child process is compromised, malicious
       attackers have injected the following statements
       into this process */

    write (fd, "Malicious Data\n", 15);
    close (fd);
  }
}
\end{Verbatim}



\section{Submission}


You need to submit a detailed lab report to describe what you have done and
what you have observed, including screenshots and code snippets.
You also need to provide explanation to the
observations that are interesting or surprising. You are encouraged to
pursue further investigation, beyond what is required by the lab
description. Your can earn bonus points for extra efforts (at the
discretion of your instructor).

If you edited your lab report on a separate system, copy it back to the Linux system at the location
identified when you started the lab, and do this before running the stoplab command.
After finishing the lab, go to the terminal on your Linux system that was used to start the lab and type:
\begin{verbatim}
    stoplab setuid-env
\end{verbatim}
When you stop the lab, the system will display a path to the zipped lab results on your Linux system.  Provide that file to
your instructor, e.g., via the Sakai site.



\end{document}
